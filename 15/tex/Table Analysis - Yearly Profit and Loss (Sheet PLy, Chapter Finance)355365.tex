

\subsection{Yearly Profit and Loss}\label{sec:title}
\nonumsidenote{Summary: This report provides an analysis of the yearly profit and loss statement for Alpha Project. It shows that revenue has increased significantly over the past five years, with gross profit margins remaining relatively steady. Operating expenses have also increased over this period, but not as quickly as revenues, resulting in a gradual increase in EBITDA margin. The net income margin has also increased steadily due to an effective tax strategy.}

The table data provided presents a comprehensive overview of Alpha Project's yearly financial performance over the past five years (mm US$). Revenues have grown rapidly from 601,200 in year 0 to 6,500,492 in year 5. Product sales account for the majority of this growth while other services remain stagnant at 0 throughout all five years. Cost of goods sold (COGS) show a similar trend as revenues with an increase from 237,000 to 2,800,188 over this period. 

Gross profit is calculated by subtracting COGS from revenues and is indicative of how efficiently Alpha Project utilizes its resources and production capabilities. Over these five years gross profit has grown from 364,200 to 3,700,303 showing that their operations are becoming increasingly efficient. Operating expenses follow a similar pattern with an increase from 337143 to 2279333 which is slightly slower than revenue growth leading to a gradual increase in EBITDA margin from 3.16\% - 21.85\%. 

Labor cost accounts for the majority of operating expenses increasing from 264000 - 1746750 over these five years while rent remains relatively stable at 60k-120k per year followed by material cost (11124-70891), maintenance cost (11124-70891), other costs (1503-17758) IT costs (3800-4700) and sales/marketing costs (33372-212673). Lease fee remains constant at 670 per year throughout all 5 years indicating consistent rental agreements or ownership status over this time frame. 

EBITDA is calculated by subtracting operating expenses from gross profits which results in 19017 - 1420300 showing that Alpha Project is able to generate positive returns on its investments despite increasing operational costs associated with scaling up their business activities such as labor and materials cost etc.. After deducting depreciation & amortization expense ($8000 each year) we get 11017 - 1412300 showing that Alpha project generates positive earnings before taxes each year which further increases after taking into account interest payments ($0-$37400). Finally after deducting current taxes ($2203-$282453) we get 8814 - 1129810 showing that net income increases steadily each year due to effective tax planning strategies employed by Alpha project�s management team  

In conclusion it can be seen that overall financial performance of Alpha Project has improved significantly over the past 5 years primarily