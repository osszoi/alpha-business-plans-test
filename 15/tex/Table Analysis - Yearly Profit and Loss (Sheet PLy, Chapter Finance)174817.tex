

\subsection{Yearly Profit and Loss}\label{sec:title}
\nonumsidenote{\textbf{Summary}: This section examines the yearly profit and loss of Alpha Project by analyzing data from a table. The results show that product sales revenue has increased steadily over the past five years, while cost of goods sold have also risen. Gross profit margins have decreased slightly, but EBITDA, EBIT and net income margins remain relatively stable. Operating expenses such as labor costs, rent, material costs, maintenance costs and other expenses have all increased over the same period.}

The table presented provides data on Alpha Project's yearly revenues and expenses for the past five years (mm US$). The first column indicates the month in which each year ends. Revenues are broken down into product sales revenue (column 3) and other services revenue (column 4). Cost of goods sold is given in column 5. 

Gross profit (column 6) is calculated by subtracting cost of goods sold from total revenues. Operating expenses are listed in column 7; these include labor cost (column 8), rent (column 9), material cost (column 10), maintenance cost (column 11) and other expenses such as recoveries or IT costs (columns 12-13). 
EBITDA is given in column 16; it is calculated by subtracting operating expenses from gross profit. Depreciation and amortization are then subtracted to obtain EBIT in column 18. Interest payments are listed in column 19; this figure is then subtracted from EBIT to obtain earnings before taxes (EBT) in column 20. Finally, current tax payments are subtracted from EBT to obtain net income for each year in column 22. 

 From this analysis we can see that product sales revenue has been steadily increasing over the past five years - from 601200 mm US$ to 6500492 mm US$. Cost of goods sold has also increased during this period - up from 237000 mm US$ to 2800188 mm US$. As a result gross profit margin has decreased slightly - down from 60\% to 57\%. However, operating expenses such as labor costs, rent, materials and maintenance have all increased over time - indicating higher efficiency within Alpha Projects operations compared with previous years. 
 
In terms of profitability metrics such as EBITDA margin or net income margin there has been little change over the past five years - indicating that Alpha Projects operations remain efficient despite increasing sales volumes/costs associated with them.. For example EBITDA margin remains at around 21-22\%, while net income margin remains at around 17-18\%. This suggests that Alpha Project's management team continue to do an effective job controlling overhead costs even as their business grows larger every year. 

Overall it appears that Alpha Projects' financial performance continues to be strong despite steady growth over the past few years - suggesting good management practices within the company's operations team