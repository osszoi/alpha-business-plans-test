

\subsection{Assumptions}\label{sec:assumptions}
\nonumsidenote{This section provides an analysis of the assumptions made in the Alpha Project business plan. It includes a discussion of growth, inflation, and costs associated with packing and shipping, materials, maintenance, sales and marketing and taxes.}

The Alpha Project business plan has made several assumptions that are key to its success. In terms of growth, it is assumed that the company will grow by 80\% over five years. This is an ambitious goal but one that could be achievable given the right circumstances. Inflation is assumed to be 3\%, which is below the current rate for most countries. 

In terms of costs associated with packing and shipping, materials, maintenance, sales and marketing and taxes; these are assumed to be 1\%, 1\%, 0.5\%, 3\% and 20\% respectively. These percentages may vary depending on different factors such as location or type of product being shipped or sold but they provide a good starting point for estimating costs associated with these areas. 

The results summary provides insight into how successful this project could potentially be over five years in terms of both total revenue (13 834 026 US$) as well as maximum revenue (6 500 492 US$). The same can also be said for total NIAT (2 069 283 US$) as well as maximum NIAT (1 129 810 US$). Lastly there is also a minimum investment amount required which stands at 148 353 US$. All these figures provide useful information when making decisions about whether or not to invest in this project. 

Finally there are two important measures used to evaluate projects such as this one - NPV@10% ($2 543 467 76) and IRR (186%). NPV stands for net present value which takes into account all future cash flows from this project discounted back to today's date so we can compare them against other investments available today while IRR stands for internal rate of return which looks at how quickly an investor will earn back their initial investment plus any additional returns generated from this project over time taking into account all future cash flows discounted back to today's date again. Both these measures help us decide if investing in this project would make financial sense compared to other investments available today. 

Summary: This section analyses the assumptions made in the Alpha Project business plan regarding growth rate, inflation rate, cost percentages associated with packing/shipping/materials/maintenance/sales&marketing/taxes; results summary regarding total & max revenue & NIAT over 5 years; minimum investment required; NPV@10%; IRR measure used to evaluate potential return on investment compared against other investments available today