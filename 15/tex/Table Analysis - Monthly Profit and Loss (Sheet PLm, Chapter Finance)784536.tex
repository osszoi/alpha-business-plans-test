

\subsection{Monthly Profit and Loss (m = month, y = year)}\label{sec:title}

Esta sección presenta un análisis de los resultados financieros mensuales del Proyecto Alpha. Se muestra el ingreso, los costos de bienes vendidos, los gastos operativos, el margen bruto de ganancias, el EBITDA y el margen neto de ganancias. El análisis se basa en la información recopilada desde enero hasta diciembre del año 2020. 

\nonumsidenote{\textbf{Resumen}: Los resultados financieros mensuales del Proyecto Alpha muestran que durante todo el año 2020, su ingreso fue constante con 50 mil dólares cada mes; sin embargo, sus gastos operativos variaron entre 28 mil y 22 mil dólares al mes. Esta variación puede ser atribuida principalmente a la fluctuación en los costes laborales y alquileres. Su margen bruto promedio fue del 60 por ciento con un EBITDA promedio de 3 por ciento y un margen neto promedio de 1 por ciento.} 

La tabla anterior muestra que durante todo el año 2020 el Proyecto Alpha tuvo ingresos constantes de 50 mil dólares cada mes. Esta consistencia es positiva para la empresa ya que indica que están logrando generar ingresos estables e inclusivo crecientes durante este período. Sin embargo, también es importante analizar los gastos operativos totales para determinar si hay áreas en las que puedan reducir costes o mejorar su eficiencia para maximizar sus ganancias netas. 

Los resultados financieros mensuales del Proyect Alpha muestran que sus gastos operativos variaron entre 28 mil y 22 mil dólares al mes durante todo el año 2020. La mayor parte de esta variación puede ser atribuida principalmente a la fluctuación en los costes laborales y alquileres asociados con la producción y ventas; además también hay otros factores comunes comunes tales comocostes materiales , mantenimientoy otros (rec