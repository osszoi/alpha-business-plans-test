

\subsection{Yearly Profit and Loss}\label{sec:title}
\nonumsidenote{This section provides an analysis of the yearly profit and loss of Alpha Project. The data shows that revenues, product sales, gross profit, EBITDA margin, EBIT margin and net income margin have all increased over the past five years. Additionally, labor costs, rent expenses, material costs, maintenance costs and other expenses have all decreased over the same period.}

Alpha Project has experienced a steady increase in its revenues over the past five years. In mm US$, their revenues rose from 601200 in m0y0 to 6500492 in m5y5. This indicates that Alpha Project�s operations are becoming more profitable as time goes on. 

Product sales also increased for Alpha Project during this same time frame. Starting with 601200 in m0y0 and increasing to 6500492 in m5y5. This suggests that Alpha Project is successfully selling more products as time goes on which is leading to higher overall profits for the company. 

The cost of goods sold was also lower than expected for Alpha Projects during this period of time. Starting at 237000 in m0y0 and decreasing to 2800188 by m5y5 indicates that they are able to keep their costs down while still maintaining high levels of production efficiency which leads to higher profits overall for the company. 

Gross profit is another area where Alpha Projects saw increases over this five year period as well. Gross profits started at 364200 in m0y0 and increased steadily until reaching 3700303 by m5y5 indicating that their production methods are becoming increasingly efficient as time passes resulting in higher profits overall for the company. 
EBITDA margin also saw increases over this same period of time starting at 3% (19017/601200)inm 0 y 0and increasing steadily until reaching 21% (1420300/6500492)inm 5 y 5 suggesting that they are successfully reducing their operating expenses while still increasing their level of output leading to improved profitability across all aspects of operations for Alpha Projects during this period of time . 

 Additionally , labor cost , rent expense , material cost , maintenance cost , other expenses (negative recoveries ) IT sales & marketing lease fee were all lower than expected when compared with previous years . Labor cost decreased from 264000mm US$inm 0 y 0to 1746750mm US$inm 5 y 5which indicates a reduction in overhead associated with labor . Rent expense decreased from 30000mm US$inm 0 y 0to 155000mm US$inm 5 y 5indicating a decrease in rental related expenditure . Material cost decreased from 6000mm US$inm 0 y 0to 70891mm US$inm 5 y 5suggesting better management when it comes to material procurement . Maintenance cost decreased from 6000mm US$inm 0