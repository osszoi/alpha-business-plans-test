
                El producto o servicio descrito en este documento es una solución de software para la gestión de proyectos. Está diseñado para ayudar a las empresas a administrar y controlar sus proyectos desde el inicio hasta el final. La solución ofrece herramientas útiles para simplificar el seguimiento de los avances, los recursos involucrados, los plazos y los costes asociados a cada proyecto. Además, incluye funciones como alertas por correo electrónico, informes personalizables y análisis gráficos que permiten evaluar la eficacia de cada proyecto. 
            
                \section{Ventajas del Producto o Servicio} \label{sec:ventajas_producto_servicio} 
            
                Nuestro producto o servicio ofrece numerosas ventajas respecto al uso tradicional de herramientas manuales para la gestión de proyectos. Entre ellas destacamos: 

                \begin{itemize} 
                    \item Reduce considerablemente el tiempo necesario para llevar un seguimiento detallado del progreso de todos los proyectos. 
                    \item Proporciona información precisa sobre cualquier aspect relacionado con un proyect

                Venezuelan Hot Dogs es una empresa de comida rápida que ofrece hot dogs gourmet. Esta fue fundada por Lucas hace cuatro años con la intención de llevar el sabor venezolano a la zona de Doral, Florida. La empresa está estructurada legalmente como una Compañía de Responsabilidad Limitada y actualmente cuenta con cinco empleados.

                La empresa se ha destacado por su innovadora receta para preparar los hot dogs, además de ofrecer un gran servicio al cliente. Estas características han permitido que Venezuelan Hot Dogs tenga éxito en el mercado local, logrando ventas anuales por valor de 4 millones y contando con 5 sucursales en la zona.

                El mercado meta para Venezuelan Hot Dogs son las personas que habitan en Doral, pero también existen otros factores externos que influyen directamente sobre el negocio como la competencia, regulaciones gubernamentales y cambios económicos. Los principales competidores son los grandes restaurantes como McDonalds y Wendys, quienes presionan al negocio con precios bajos y productos diferenciados. El poder de compra por parte del consumidor es bajo debido a la cantidad considerable de restaurantes disponibles; sin embargo, el poder de compra por parte del proveedor es alto ya que hay escasez en la cadena suministradora. Finalmente, existe una amenaza constante por parte nuevos entrantes así como sustitutos (opciones saludables).

Los productos ofrecidos por Venezuelan Hot Dogs satisfacen las necesidades del consumidor gracias a sus sabores únicos, rapidez en el servicio y conveniencia para los clientes localizados dentro del área metropolitana. Estas características hacen que se distingan frente a sus competidores e incrementen su ventaja competitiva. Las fortalezas internas incluyen su receta única así como su fuerte marca reconocida dentro del sector; mientras que sus debilidades son principalmente relacionadas con limitaciones financieras y falta de diferenciación respecto a otros productores similares. Por otra parte, existen muchas oportunidades externas tales como nuevos mercados abiertos así como avances tecnológicos; sin embargo, también hay amenazantes externamente tales como alta competencia y crisis económicas globales.

                Para capitalizar sobre sus fortalezas internas (receta única) junto con las oportunidades externamente (mercado venezolano), Venezuelan Hot Dogs implementará estrategias tales como campañas publicitarias dirigidas hacia la Comunidad Venezolana así mismo programando descuentos en eventos particulares durante fines de semana. Además, se planea construir 6 sucursales situadas dentro del área metropolitana durante el próximo año.

Respecto a la aplicación de valor de precios, se considera que esta será más elevada de lo establecido normalmente, de lo que ofrecen similares restaurantes rápidos en la misma zona. La forma de distribución será mediante la ubicación del negocio dentro de una tienda minorista ubicada en el Sawgrass Mall. Finalmente, al momento de formular una estrategia marketing se considera trabajar enfocarse principalmente en la Comunidad Venezolana y promover los beneficios e innovadoras presentadas por los productos a través de medios digitales y boca-a-boca.

                En cuanto a la gestión humana, se contratarán personal calificado y amigable para garantizar el buen servicio al cliente y cumplir con las normas y reglamentaciones presentadas por la ley laboral venezolana. Además, se implementará un plan detallado de desarrollo y capacitación para los trabajadores existentes y potenciales que integren la empresa.

                Desdemateriales e infraestructura actualmente dispone una pequeña materia prima para preparar las hot dogs y una pequeña tienda para expenderlas mismas situadas dentro del Sawgrass Mall; sin embargo, existe proyecto para ampliar la infraestructura existente con 6 tiendas adicionales que se situarán en el Doral Area durante el próximo.
