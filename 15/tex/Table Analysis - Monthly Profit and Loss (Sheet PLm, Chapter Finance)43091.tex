\nonumsidenote{\textbf{Resumen:}} El presente capítulo del plan de negocios se enfoca en el análisis de los resultados obtenidos por la empresa Alpha Project durante el año 2020, específicamente en su rentabilidad. Para ello, se utilizaron los datos contenidos en la tabla anterior para calcular los principales indicadores financieros y determinar si el proyecto está generando las ganancias esperadas.
El Proyecto Alpha inició sus operaciones en enero de 2020, logrando un ingreso mensual promedio de 50.100 dólares, provenientes principalmente de la venta de productos. Esta cifra permitió obtener un margen bruto promedio del 60,58%, con un costo asociado de 19.750 dólares al mes. Por otro lado, se evidencia que los costos operativos totales (28.095 dólares) representan el 56% del ingreso total, siendo los principales costos los laborales (22.000 dólares), arrendamientos (2500 dólares), materiales (500 dólares) y mantenimiento (500 dólares).
Esta situación genera un EBITDA promedio mensual de 1.585 dólares, lo cual equivale a un margen EBITDA del 3,16%. Siendo este número menor al margen bruto debido a los gastos operativos antes mencionados; sin embargo, todavía hay cierto potencial para mejorarlo reduciendo estas últimas o incrementando las ventas realizadas durante el periodo analizado.
Adicionalmente, se puede apreciar que el Proyecto Alpha tuvo un EBIT promedio mensual de 918 dólares, lo cual equivale a un 1,83% sobre su ingreso total. Sin embargo, este resultado fue afectado por la presencia de intereses financieros asociados al proyecto ($0). Finalmente, gracias a esta situación, lograron obtener 734 dólares como margen neto de ingresos en promedio mensual (1,47%).
En conclusión, podemos examinar que el Proyecto Alpha estuvo operando con un margen bruto promedio mensual del 60%, controlando adecuadamente los costos y gastos operativos totales que supusieron el 56% del ingreso total generado. Además, obtuvieron un EBITDA promedio mensual de 1.585 dólares y un neto promedio de 734 dólares mensual, lo que constituye buenos resultados para un proyecto de reciente enfoque inversor como el del Proyecto Alpha.
