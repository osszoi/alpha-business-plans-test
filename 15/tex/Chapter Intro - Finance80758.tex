

Venezuelan Hot Dogs is a limited liability company owned by Lucas, which has been in operation for four years. The company offers two types of hot dogs and its target market is the people living in Doral area. Venezuelan Hot Dogs� competitive advantage lies in its unique recipe and excellent customer service. This chapter covers the financial aspects of Venezuelan Hot Dogs including the current size and capacity of the business, pricing strategy, profit margins, and long-term goals. 

The short-term goal of Venezuelan Hot Dogs is to sell 4 million hot dogs within one year while its long-term goal is to open five branches across Miami area within three years. To reach this target market, the company plans to use social media campaigns, digital marketing campaigns, word-of-mouth advertising, fliers at malls etc. The pricing strategy will be high compared to other fast food restaurants but still affordable for customers looking for convenience and quality product. To achieve these goals an estimated capital investment of $200 000 is needed which includes costs associated with renting facilities; purchasing raw material; paying salaries; marketing expenses etc., before it can reach its short term goals of selling four million hot dogs within one year . This capital investment should also cover costs associated with opening five branches across Miami area over next three years so that long term goals can be achieved successfully as well . 

Yearly Profit and Loss analysis shows that revenues generated through product sales continue to rise while associated operational costs remain relatively constant or decrease slightly resulting in steady increases across all profitability metrics including gross profit margins (6058\%-5692\%), EBITDA Margins (316\%-2185\%) ,EBIT Margins(183\%-2173\)and Net Income Margins(147\-1738\%). This indicates that despite increasing operational costs profits are still on an upward trajectory due mainly to increasing revenues generated through product sales.