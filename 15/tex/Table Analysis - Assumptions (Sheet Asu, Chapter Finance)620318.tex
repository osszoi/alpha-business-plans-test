\section{Supuestos} Los supuestos del proyecto Alpha se basan en datos históricos y estimaciones realistas para los próximos cinco años. Estas estimaciones incluyen el crecimiento, la inflación, los costos de embalaje y envío, los costos de materiales, los costos de mantenimiento, otros (recuperaciones negativas) e impuestos. Los resultados previstos se muestran en la tabla 1.
En primer lugar, se espera un crecimiento promedio anual del 80\% durante el período de cinco años. Esta tasa de crecimiento refleja el potencial del mercado objetivo y las estrategias implementadas por Alpha para alcanzarlo. Además, se espera que la inflación sea del 3\%, lo que significa que el poder adquisitivo disminuirá con el tiempo debido a la devaluación monetaria.
Además, se esperan gastos relacionados con el empaque y envío equivalentes al 1\% del total de ingresos cada mes; así como costos materiales equivalentes al 1\%. Se esperan gastos de mantenimiento relativamente bajos representando un 0.5\% del total mensualmente; así como otros gastos (recuperaciones negativas) representando un 0.25\%. Finalmente, Alpha Project tendrá que pagar impuestos por sus ganancias equivalentes a un 20\%.
La tabla 1 muestra también los resultados financieros previstos para Alpha Project durante los próximos 5 años. El total previsto de ingresos durante este periodo es de 13 834 026 USD, con un máximo anual de 6 500 492 USD, así como un monto total previsto del NIAT de 2 069 283 USD con un máximo anual de 1 129 810 USD. Por otro lado, se necesita una inversión inicial de 148 353 USD para iniciar el proyecto y obtener un NPV@10\% de 2 543 467 76 USD y un IRR del 186\%.
En resumen, los supuestos empleados para Alpha Project son realistas y optimistas. Las estimaciones se basan en datos históricos y las estrategias planteadas por el equipo de gestión. Se ha estimado un crecimiento promedio del 80%, una inflación media anual del 3%, costes mensuales por servicios relacionados con embalaje y envío equivalentes al 1%, costes mensuales por materiales equivalentes al 1%, gastos por mantenimiento equivalentes al 0.5% y otros gastos equivalentes al 0.25%. Además, se ha considerado una paga de impuestos al 20%. Estos supuestos tienen como resultado una inversión inicial necesaria de 148353 USD con un NPV@10% de 2543467.76 USD e IRR del 186%.
