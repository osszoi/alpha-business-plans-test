\\nonumsidenote\\{resumen} En este capítulo se analizará el balance mensual del Proyecto Alpha. Se examinarán los activos y pasivos de la empresa, así como los ingresos y patrimonio de los accionistas.
El balance mensual del Proyecto Alpha muestra una tendencia ascendente en sus activos totales durante los 12 meses del año. Los activos circulantes incluyen efectivo, cuentas por cobrar (A/R) e inventarios, que aumentaron significativamente desde enero hasta diciembre. El efectivo aumentó de $1,647 a $17,060 durante el período de 12 meses, lo cual es un incremento significativo para la empresa. Además, el inventario también se incrementó desde $4,608 hasta $4,608 durante el mismo período. Por otro lado, los activos fijos disminuyeron ligeramente desde $149,333 hasta $142,000 durante el período mencionado anteriormente.
Los pasivos totales también se mantuvieron estables durante todo el año, con un total de 155,589 al principio del periodo y 163,668 al final del mismo. Los pasivos corrientes incluyen pagarés comerciales y otros pagarés que permanecieron constantes en 812 y 3,925, respectivamente, para todos los meses considerados en este estudio. Además, las provisiones fueron igualmente estables con 118 para cada mes considerado en este estudio.
En cuanto al patrimonio de los accionistas, éste mostró un crecimiento constante desde 150,000 hasta 158,814 durante el periodo mencionado anteriormente. Asimismo, las ganancias generadas por la compañía también mostraron un crecimiento sostenido desde 734 hasta 8,814 para todos los meses considerados en este estudio. Estas ganancias son resultado directamente de la gestión financiera exitosa realizada por la compañía Alpha Project.
Con base en lo anteriormente expuesto, podemos afirmar que la administración financiera exitosa ha sido clave para lograr resultados positivos dentro del Proyecto Alpha. Sin embargo, es importante recalcar que estas decisiones deben ser tomadas con responsabilidad, ya que cualquier movimiento puede poner en riesgo las finanzas futuras de la empresa si no son bien evaluadas previamente.
