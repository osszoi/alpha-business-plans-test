

\subsection{Monthly Profit and Loss}\label{sec:title}
\nonumsidenote{This section provides an analysis of the monthly profit and loss data for the Alpha Project. Revenue, product sales, cost of goods sold, gross profit, operating expenses, labor costs, rent, material costs, maintenance costs, other expenses (negative recoveries), IT costs, sales and marketing expenses, lease fees are all analyzed to calculate EBITDA (earnings before interest taxes depreciation and amortization) margin as well as EBIT (earnings before interest taxes) margin. Net income margin is also calculated to determine the profitability of the project.} 

The table data provided in this section provides a detailed analysis of the financial performance of the Alpha Project over a period of 12 months. The first row indicates that Month 0 corresponds to January while Month 11 corresponds to December. The second row indicates that total revenue for each month was $50 100. Product Sales were also equal to $50 100 per month indicating that there were no significant variations in sales during this period. Other services rendered during this period were not significant with a value of zero across all months. 

Costs associated with goods sold amounted to $19 750 per month indicating that these figures remained constant throughout the year-long period under consideration. This resulted in a gross profit figure of $30 350 for each month which is equivalent to a gross profit margin of 60\%. Operating expenses included labor costs at $22 000 per month as well as rent ($2 500), material costs ($500), maintenance costs ($500), other expenses (negative recoveries) at $125 and IT related expenses at $300 per month respectively. Sales and marketing expenditures amounted to $1 500 while lease fees totaled at 670 dollars per month resulting in an EBITDA figure amounting to 1 585 dollars for each month or 3 16\% when expressed as a percentage figure relative to total revenue generated during each respective month under consideration.. 

EBIT was determined by subtracting Depreciation & Amortization (D&A) from EBITDA resulting in 918 dollars or 1 83 \% when expressed relative to total revenues generated during each respective period under consideration.. Interest payments were minimal amounting only 0 00 during this 12-month period under consideration thus leaving Earnings Before Tax (EBT) equal 918 dollars or 1 83 \% relative total revenues generated throughout this particular time frame.. Current tax payments amounted 184 dollars resulting in net income margins totaling 734 dollars or 1 47 \% when expressed relative total revenues generated throughout this particular time frame.. 

Overall these figures indicate that although profits remain relatively consistent across all twelve months considered here they are still quite low suggesting potential areas where improvements can be made such as reducing operating expenses increasing product sales etcetera order improve overall profitability levels.. In conclusion it appears that although profits remain stable they could likely be improved upon given some additional effort on behalf of