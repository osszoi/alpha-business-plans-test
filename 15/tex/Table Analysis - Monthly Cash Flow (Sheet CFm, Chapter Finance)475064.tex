

\subsection{Monthly Cash Flow}\label{sec:title}
\nonumsidenote{\textbf{Summary}: This section provides an analysis of the cash flow of Alpha Project for twelve months, starting from m=0. It shows that the initial capital expenditure is high but decreases over time, while net income and D\&A remain constant throughout. Operating cash flow increases from month to month, with a peak in m=11. Net cash decreases significantly in m=1 due to a large capital expenditure payment for shares, but then slowly increases until it reaches a balance of 17,060 mm US$ by m=12.}

This section provides an analysis of the monthly cash flow of Alpha Project from m = 0 to 12. The data presented in Table 1 reveals several key trends regarding Alpha Project's financial performance over this period. 

First, it is evident that the initial capital expenditure payment is quite high at 150,000 mm US$, which is paid out entirely in shares during m = 1 (row 6). This amount decreases significantly over time and eventually reaches zero by m = 12 (row 5). On the other hand, net income and D\&A remain constant throughout at 734 mm US$ (rows 1 and 2). 

Second, operating cash flow increases steadily each month as all other values remain roughly constant (row 4). By m = 11 it has reached its peak value at 14258 mm US$, before falling slightly to 1401 mm US$ by the end of the year (m = 12). This suggests that although there was a large initial investment cost when paying for shares during m = 1, Alpha Project's financial performance improved steadily over time as more money came into its coffers. 

Thirdly, net cash follows a different pattern than operating cash flow. Although both values start off relatively low at -148353 and 1647 respectively during m = 0 (rows 8 and 10), they diverge quickly after this point due to a large capital expenditure payment for shares during m = 1 (row 6). As such net cash falls sharply from 1647 mm US$ to just 1401mm US$ by the end of this month (row 8), whereas operating cash flow remains roughly unchanged at 1401mm US$. However after this point both values increase steadily until they reach their respective peaks at 17052mm US$ for netcash and 14258mmUS$ for operatingcashflowbym=11(rows8and4respectively)beforefallingslightlyinm=12to1706and1401respectively(rows10and4respectively). 

Overall these figures suggest that despite incurring a significant upfront cost when purchasing shares duringm=1AlphaProject�sfinancialperformanceimprovedsteadilyoverthecourseoftheyearasthemoneycomingintoitscoffersincreasedaswellasitsoperatingcashflowreachingitspeakvalueat14258mmUS$.By