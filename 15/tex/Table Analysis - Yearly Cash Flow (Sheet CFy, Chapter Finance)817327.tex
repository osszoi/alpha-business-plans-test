\\nonumsidenote\\{resumen} Este documento se centra en el análisis del flujo de caja anual de la empresa Alpha Project. Se examinan los ingresos netos, los gastos operativos y de capital, el pago de préstamos bancarios y la inversión externa como principales fuentes de financiación. También se evalúa el saldo final alcanzado durante el período analizado.
El flujo de caja es una herramienta fundamental para las empresas, ya que permite conocer su situación financiera actual y futura, así como también determinar si cuenta con los recursos necesarios para afrontar sus compromisos financieros. En este sentido, la tabla anterior muestra los datos correspondientes al flujo de caja anual para la empresa Alpha Project durante un periodo comprendido entre enero del año 1 hasta diciembre del año 5.
Los ingresos netos representan el principal componente en el flujo de caja para esta empresa. Estas ganancias se incrementaron considerablemente desde enero del año 1 (8 814) hasta diciembre del año 5 (1 129 810). Esto refleja un crecimiento significativo en la producción y ventas realizadas por parte de Alpha Project durante este periodo, lo que resultó ser clave para mejorar su situación financiera.
Además, es importante mencionar que existen dos tipos principales de gastos: operacionales y capitales (CAPEX). Los primeros incluyen todos aquellos relacionados con los costes laborales y administrativos; mientras que los segundos son aquellas inversiones hechas con fines productivos o comerciales tales como maquinaria o software. En este caso particular, ambas categorías tuvieron un impacto negativo sobre el flujo neto; sin embargo, resultó ser menor comparado con las ganancias obtenidas durante este periodo.
Por otra parte, también fue relevante considerar el pago efectuado por préstamo bancario, así como las inversiones externas recibidas por parte de accionistas externos a esta empresa (CF from Bank Payment / Loan (=) y CF from Investment (+)). Respectivamente, estas cantidades ascendieron a 0$ y 150000$ durante enero del primer año, y 25000$ durante febrero del mismo año. Sin embargo, no hubo más pagos ni nuevas inversiones durante el resto del periodo analizado (3-5). Esto es debido a que la empresa AlphaProject no requirió ningún préstamo adicional para financiar sus operaciones a lo largo del periodo analizado o para realizar desarrollos posteriores durante este tiempo.
Finalmente, en cuanto al saldo final observado durante cada uno de estos años, se detecta que hay un crecimiento progresivo en las cifras observadas desde enero del primer año hasta diciembre del quinto (55,969$ - 2,198,403$), lo que indica que Alpha Project había logrado gestionar sus recursos habilitando un mejor manejo de efectivo y reduciendo los gastos en general para lograr un aumento en las ganancias sostenibles durante el extenso periodo analizado (1-5).
En conclusión, se observa que la firma de Alpha Project logró mantener un flujo de efectivo positivo tanto en el corto plazo como en el largo plazo gracias a un buen gestionamiento financiero y una adecuada administración de recursos capitales y operacionales que permitieron fortalecer el saldo efectivo disponible para la empresa al final del periodo analizado (2198403$).
