

\subsection{Monthly Balance Sheet}\label{sec:title}
\nonumsidenote{Summary: This section provides an analysis of the monthly balance sheet data for the Alpha Project company. It looks at the assets, liabilities and equity of the company over a 12 month period. The analysis reveals that total assets have increased steadily over time while liabilities have remained relatively stable. Additionally, earnings and equity shareholders have also grown significantly during this period.}

The monthly balance sheet data for Alpha Project Company shows a steady increase in total assets from 155,589 to 163,668 over 12 months. Current assets such as cash and accounts receivable (A/R) experienced significant growth in this period with cash increasing from 1,647 to 17,060 and A/R remaining at 0 throughout this timeframe. Inventories had a static value of 4608 across all months. Fixed Assets decreased slightly from 149333 to 142000 with long term debt remaining zero throughout this period. 

Total liability plus equity also increased steadily from 155 589 to 163 668 during this time frame confirming that total asset growth was not due to any increase in liabilities or equity but rather due to genuine asset accumulation by the company. Current Liability values remained constant at 4855 while Trade Payables and Other Payables were 812 and 3925 respectively across all months indicating no significant changes in these areas during the 12 month period analyzed here. Provisions also remained stable at 118 throughout this time frame suggesting that there has been no major change in risk exposure for Alpha Project Company since last year. 

Equity shareholders increased significantly from 150 000 to 158 814 over 12 months indicating strong investor confidence in Alpha Project Company's operations as well as its ability to generate returns on their investments through dividends or share price appreciation or both. Earnings also grew significantly from 734 to 8814 further reinforcing investor confidence in Alpha Projects operations as well as its ability to generate profits which can be reinvested into business operations or distributed among shareholders through dividends or share buybacks etc.. 

Overall, it appears that Alpha Projects has experienced strong growth over the past year due largely to its ability to accumulate assets without taking on too much additional debt or increasing shareholder equity significantly thus allowing them more control over their finances while still generating returns for investors who are willing take risks investing into their operations. This is a promising sign for future prospects of Alpha Projects going forward into 2020 and beyond!