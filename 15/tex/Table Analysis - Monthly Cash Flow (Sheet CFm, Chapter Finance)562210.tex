

\subsection{Monthly Cash Flow}\label{sec:title}
\nonumsidenote{This section provides an analysis of the monthly cash flow of Alpha Project. It includes a summary of net income, operating cash flow, capital expenditure, and cash balance for each month. The analysis concludes that the company is in a strong financial position with steady growth.} 

The table above shows the monthly cash flow for Alpha Project over twelve months. Net income is represented by row 1, D&A (depreciation and amortization) by row 2, working capital (WC) by row 3, operating cash flow (OCF) by row 4, capital expenditure from bank payment/loan by row 7, net cash by row 8, CF from investment by row 9 and finally the cash balance at the end of each month shown in row 10.  

The net income for all twelve months remains constant at \$734 million US dollars. The D&A also remains constant throughout all twelve months at \$667 million US dollars. The WC decreases to \$0 after month 0 due to an initial outflow of \$246 million US dollars in month 0 which is not repeated in subsequent months. This results in an OCF of \$1,647 million US dollars for month 0 which then steadily increases to \$1706 million US dollars in month 12 due to the lack of WC outflows after month 0 as well as increasing revenues from investments made during this period. 

Capital expenditure from bank loan payments remains constant throughout all twelve months at zero while capital expenditure from share issuance was only incurred once during this period; a one-time outflow of \$150000 million US dollars occurred during month 1 with no further outlays occurring afterwards. As such there is no change seen in Row 7 - CF from Bank Payment/Loan - throughout this period as no payments were made towards any loans taken out during this time frame. 

Net Cash calculated using rows 1-3 minus rows 4-6 gives us a total inflow or outflow depending on whether it is positive or negative respectively; this figure can be seen on Row 8 with an initial outflow of -\$148353 followed by inflows ranging between +\$1401 and +\$17060 over the next eleven months respectively due to increased OCFs offsetting any remaining costs associated with investments made during this period such as share issuance expenses incurred during Month 1 previously mentioned earlier on Row 6 . Finally we have CF from Investment shown on Row 9 which has a single entry representing one-time share issuance expenses incurred within Month 1 as previously discussed before; thereafter there are no further entries indicating that no other investments were made within these twelve months outside those already accounted for within Rows 5 & 6 . 

 Lastly we have Cash Balance shown on Row 10 which starts off low but gradually increases up until its peak value at Month 12 when it