

\subsection{Assumptions}\label{sec:assumptions}
The Alpha Project has a projected growth of 80\% in the next five years. The rate of inflation is assumed to be 3\%, while packing and shipping, material cost, and maintenance cost are assumed to account for 1.00\%, 1.00\%, and 0.50\% respectively of total revenue in the same period. Other negative recoveries are expected to account for 0.25\%. Sales and marketing expenses will amount up to 3.00\% of total revenue, while taxes are estimated at 20\%. 

With these assumptions, the Alpha Project is expected to achieve a total revenue of 13,834,026 US$ over five years with a maximum annual revenue amounting to 6,500,492 US$. Total Net Income After Taxes (NIAT) during this period is estimated at 2,069,283 US$, with maximum NIAT over one year reaching 1,129,810 US$. Minimum investment required for this project is 148353 US$. Finally the Net Present Value (NPV)@10\% amounts to $2 543 467 76 and Internal Rate of Return (IRR) 186 \%. 

These assumptions provide us with an indication that the Alpha Project could be profitable in the long term if all conditions remain stable as anticipated by management. It should also be noted that any changes in external economic conditions or internal operations can affect these projections significantly and thus must be monitored closely throughout the project's duration. Furthermore it should be noted that although NPV@10 \% appears positive it may not necessarily represent an accurate picture due to other factors such as inflation rates or market volatility which cannot always be accurately predicted but can have a significant effect on profitability in both short-term or long-term scenarios. 

Summary: This section provides an analysis of assumptions made by management regarding growth rate; inflation; packing/shipping costs; material costs; maintenance costs; sales/marketing expenses; taxes etc., which indicate that Alpha Project could potentially generate significant profits over five years given all conditions remain stable as anticipated by management taking into account potential external economic factors which could affect profitability either positively or negatively depending on their magnitude and directionality..