

\subsection{Yearly Profit and Loss}\label{sec:title}
\nonumsidenote{This section provides a summary of the yearly profit and loss statement for the company Alpha Project. The table below displays revenue, cost of goods sold, gross profit, operating expenses, labor cost, rent, material cost, maintenance cost, other expenses (negative recoveries), IT costs, sales and marketing costs as well as lease fees. Additionally it shows EBITDA (earnings before interest taxes depreciation and amortization), EBIT (earnings before interest and taxes) Interest Expense and Earnings Before Tax (EBT). Lastly there is net income which is broken down into gross profit margin, EBITDA margin ,EBIT margin and net income margin.} 

The data from the table shows that overall revenues for Alpha Project increased from mm US$601200 in month 0 to mm US$6500492 in month 5. This increase can be attributed to an increase in product sales which also increased from mm US$601200 to mm US$6500492 over the same period. Other services however remained at zero throughout this period indicating that they are not a major source of revenue for Alpha Project. 

Costs of goods sold also increased over this period from mm US$237000 to mm US$2800188 indicating that more resources were needed to produce products for sale during this time frame. Labor costs also increased significantly from mm US$264000 in month 0 to mm US$1746750 in month 5 indicating an increase in personnel or wages paid out during this time frame. Rent expense was relatively constant throughout the five months with values ranging between 30000-60000 while material costs ranged between 6000-11124 over this same period. Maintenance costs were similar with values ranging between 6000-11124 while other expenses such as IT costs ranged between 3600-4700 over these five months. Sales and marketing expenditure saw a significant jump from 18000mmUS $in month 0 to 212673mmUS $in month 5 indicating that more resources were allocated towards promotion activities during this time frame . Lease fees on the other hand remained constant at 8040mmUS$. 

Gross profit decreased slightly from 60% in Month 0 to 56%in Month 5 due mainly to an increase in Cost of Goods Sold as mentioned earlier on . Operating expenses also increased slightly by 4% over these five months resulting in a decrease in EBITDA margins which decreased by 1% over these five months . EBT margins decreased even further by 2 % while net income margins decreased marginally by 1%. This indicates that despite increasing revenues profitability has not been affected positively due mainly to increases operating expenditures 

Overall Alpha Projects financial performance has been positive with increasing revenues , however profits have been affected negatively due mainly increases operating expenditures such as labor ,materials ,maintenance etc.. As such it would be advisable if management could focus on reducing operational expenditures without