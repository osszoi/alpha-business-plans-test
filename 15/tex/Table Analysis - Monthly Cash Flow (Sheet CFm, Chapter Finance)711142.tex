

\subsection{Flujo de Caja Mensual}\label{sec:title}
\nonumsidenote{\textbf{Resumen}: El presente documento analiza el flujo de caja mensual del proyecto Alpha. La tabla muestra los ingresos netos, los gastos por depreciación y amortización (D\&A), el capital de trabajo (WC) y el flujo de efectivo operacional resultante. Además, se muestran los gastos de capital relacionados con la financiación bancaria y la financiación a través de acciones, así como el flujo neto de efectivo resultante. Por último, se incluyen los ingresos generados por las inversiones realizadas durante el periodo considerado y una descripción detallada del saldo global en efectivo para cada mes.}

El análisis del flujo de caja mensual permite evaluar la capacidad financiera a corto plazo que tiene una empresa para cumplir sus obligaciones financieras. Esta información es fundamental para determinar si una compañía cuenta con suficiente liquidez para cubrir sus pagos corrientes sin tener que recurrir a nuevas fuentes externas o endeudamiento adicional. 

En primer lugar, se observan los ingresos netos obtenidos durante todos los meses considerados en la tabla ($734$ mm US$). A continuación, se pueden ver los gastos por D\&A ($667$ mm US$), lo que indica que están destinando recursos significativamente al desarrollo tecnológico y/o innovador. Además, hay un coste importante debido al capital de trabajo (-246 mm US$). Si bien no es necesario hacer frente a este tipo de pagos periódicamente, son indispensables para mantener la operatividad normal dentro del negocio. 

Los siguientes dos renglones corresponden al Flujo Neto Operacional ($1 401$ mm US$) y al Gasto en Capital relacionado con la financiación bancaria ($0 $mm US$, ya que no hay ninguna transferencia entre ambas partes). Después se ve el Gasto en Capital relacionado con