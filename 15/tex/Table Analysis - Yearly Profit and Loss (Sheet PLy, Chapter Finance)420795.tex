

\subsection{Yearly Profit and Loss}\label{sec:title}
\nonumsidenote{This section provides a summary of the yearly profit and loss for the Alpha Project. It includes an analysis of revenues, product sales, cost of goods sold, gross profit, operating expenses, labor cost, rent, material cost, maintenance cost, other expenses (negative recoveries), IT costs, sales and marketing costs and lease fees. Additionally it also shows Earnings Before Interest Taxes Depreciation and Amortization (EBITDA), Earnings Before Interest Taxes (EBIT) interest payments and net income. The results show that revenues increased steadily over the 5 year period while product sales followed a similar trend. Cost of goods sold also increased in line with revenue growth while gross profits remained relatively stable at around 60\%. Operating expenses rose slightly but EBITDA margins stayed consistent at around 20-22\% throughout the period. Net income margin was lower than EBITDA margin but still remained steady between 17-18\%.}

The Alpha Project's yearly profit & loss statement for mm US$ is provided in Table \ref{tab:pnl}. This table shows that revenues have grown steadily from 601200 in month 0 to 6500492 in month 5. Product Sales follow a similar pattern with 601200 in month 0 to 6500492 in month 5 showing growth across all months. Cost of Goods Sold has also risen from 237000 to 2800188 over the same time frame indicating increased production costs for Alpha Project products as demand grows. Gross Profit remains relatively stable at around 60\%, ranging from 364200 to 3700303 over five years showing efficient utilization of resources by Alpha Project management team to maintain profitability despite increasing production costs due to rising demand for their products. 

Operating Expenses have also risen significantly from 337143 mm US$ in month 0 to 2279333 mm US$ in month 5 due mainly to increases in Labor Costs which account for most expenditure on Operating Expenses ranging from 264000 mm US$in Month 0to 1746750 mm US$in Month 5 as well as Rent which has risen steadily from 30000mmUS$inMonth0to155000mmUS$inMonth5andMaterialCostwhichhasrisenfrom6000mmUS $inMonth0to70891mmUS $inMonth5aswellasMaintenanceCostwhichhasrisenfrom6000mmUS $inMonth0to70891mmUS $inMonth5andOtherExpenses(NegativeRecoveries)whichhasrisenfrom1503mmUS $in Month 0to17758mmUS $in Month 5aswellasITcostswhichhaveincreasedsteadilyfrom3600m m US$in Month 0to4700 m m US$i n M o n t h 5a s w e l l a sSale sandMarketingcostswhic hhav ei n