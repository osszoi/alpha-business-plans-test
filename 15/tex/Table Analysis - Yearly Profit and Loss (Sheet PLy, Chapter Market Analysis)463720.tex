

\subsection{Yearly Profit and Loss}\label{sec:title}
\nonumsidenote{This section provides an analysis of the yearly profit and loss of Alpha Project, focusing on revenues, product sales, cost of goods sold, gross profit, operating expenses, labor costs, rent expense, material costs, maintenance costs, other expenses (negative recoveries), IT expense, sales and marketing expense and lease fee. It also covers EBITDA margin and net income margin.}

Alpha Project has experienced a steady increase in revenue over the past five years. In month 0 (January) of year 1 their total revenue was \$601200. This increased steadily to \$1.11 million by month 4 (April), \$2 million by month 8 (August) and then more than doubled to \$3.6 million by month 12 (December). By the end of year 5 their total revenue had reached \$6.5 million. 

Product sales have mirrored these increases in revenue with a corresponding rise from \$601200 in January to a peak of \$6.5 million by December as well as a steady increase throughout each year for all five years studied here. Other services provided by Alpha Project have been minimal or nonexistent; there were no reported other services for any months during the period studied here. 

Costs associated with producing goods have also risen steadily over this period from \$237000 in January to nearly three times that amount at its peak in November at almost 3 million dollars before falling slightly back down to 2.8 million dollars in December where it remained until the end of year 5 when it decreased slightly again to 2.7 million dollars per month on average for that entire year.. The largest portion of these costs are labor-related with an average monthly cost rising from just under $265000 per month up to $174750 per month towards the end of this period while rent expense is relatively constant at around $60 000-$120 000 per month throughout this time frame.. Material costs are much lower than labor-related expenses but still show an overall upward trend increasing from just under $6000 each month up to around $70891 near the end of this study's period.. Maintenance costs follow a similar pattern increasing from about 6000 dollars each month up to around 70891 near the end of this study's period.. Other expenses such as IT remain relatively constant throughout this time frame while sales and marketing expenses see a significant jump beginning in August when they reach 61 872 dollars before peaking at 212 673 dollars towards the end  . Lease fee remains fairly consistent through out all five years averaging 8040 dollars per month . 

 Gross profits show an overall upward trend for all five years studied here starting off at 364 200 US Dollars for January before peaking at 3 700 303 US Dollars during December when compared against product sales which peaked during that same time frame .. Operating Expenses also rise steadily through