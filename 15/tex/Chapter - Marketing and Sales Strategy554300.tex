\section{Estrategia de Marketing y Ventas para la Compañía Proyecto Alpha}\label{sec:marketing}

Venezuelan Hot Dogs es una empresa que se dedica a la venta de hot dogs gourmet. La empresa está ubicada en Caracas, específicamente en Calle la Colina Edificio 32 Av Fuerzas Armadas. El objetivo principal de esta empresa es ofrecer hot dogs con un sabor único y de alta calidad. La empresa ofrece dos tipos de hot dogs diferentes, el hot dog 1 y el hot dog 2.

El mercado objetivo para Venezuelan Hot Dogs son las personas que viven en Doral. La ventaja competitiva que ofrece la empresa es su calidad y servicio al cliente. Los objetivos a corto plazo son vender 4 millones de dólares y tener cinco sucursales. 

La estructura legal de Venezuelan Hot Dogs es una compañía de responsabilidad limitada (LLC). La compañía ha estado en operación durante cuatro años y cuenta con cinco empleados actualmente.

Los principales competidores son todos los restaurantes de comida del centro comercial Sawgrass Mall, ya que están ubicados cerca del área donde se encuentra Venezuelan Hot Dogs. El tamaño del mercado es estimado en $30 billones, con una alta demanda debido a la gran cantidad de personas que visitan el centro comercial.

Las tendencias en la industria indican una mayor preocupación por comer alimentos saludables pero también rápidos. Los principales desafíos incluyen alta competencia, bajos precios y gestión adecuada del suministro.

La fuerza negociadora de los proveedores es alta debido a que hay solo un proveedor para el pan utilizado para los hot dogs gourmet. No hay poder negociador por parte del comprador porque no hay opciones alternativas para comprar productos similares. La amenaza de nuevos competidores es alta debido a que cada día hay nuevas empresas que ingresan al mercado. La amenaza de sustitutos también es alta, ya que cada vez hay más opciones saludables disponibles.

Los productos ofrecidos por Venezuelan Hot Dogs son hot dogs gourmet con un sabor único y conveniencia para el cliente. Las fortalezas internas incluyen una receta única y distintiva. Las debilidades internas incluyen la falta de capital para expandirse.

Las oportunidades externas disponibles para la empresa son las personas que buscan sabores innovadores y la gran comunidad venezolana en el área local. Las amenazas externas incluyen una recesión económica y competencia fuerte.

Para capitalizar en las fortalezas y oportunidades, se implementará una campaña publicitaria enfocada en la comunidad venezolana a través de redes sociales, campañas digitales, publicidad boca a boca y volantes distribuidos en el centro comercial Sawgrass Mall. El precio será alto debido al sector de comida rápida gourmet donde compite Venezuelan Hot Dogs.

El producto se distribuirá desde un punto de venta ubicado en el centro comercial Sawgrass Mall. Para diferenciarse de los competidores, se enfocará en la comunidad venezolana local con empaques únicos y conveniencia del servicio rápido.

La empresa cuenta actualmente con diez empleados, pero planea contratar amigos y familiares dentro de la comunidad venezolana local para su expansión futura. Los planes futuros incluyen abrir seis tiendas más en Doral durante el próximo año, seguido por seis tiendas adicionales por año en todo Miami.

Los procesos operativos actuales implican tres proveedores diferentes para las salchichas utilizadas en los hot dogs gourmet, un proveedor para el pan y la compra de salsas e ingredientes en Costco. La gestión del inventario se realiza semanalmente para analizar las ventas de la semana.

En conclusión, Venezuelan Hot Dogs tiene una oportunidad única para capitalizar en la comunidad venezolana local y ofrecer hot dogs gourmet con un sabor único y conveniencia al cliente. Con una estrategia sólida de marketing y ventas, así como planes futuros de expansión, esta empresa tiene un gran potencial para crecer en el mercado competitivo de comida rápida gourmet.