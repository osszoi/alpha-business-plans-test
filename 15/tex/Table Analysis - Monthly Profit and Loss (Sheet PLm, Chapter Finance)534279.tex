

\nonumsidenote{\textbf{Summary:} This chapter provides an analysis of the monthly profit and loss data for Alpha Project. It includes a discussion of the revenue, cost of goods sold, operating expenses, EBITDA margin, EBIT margin and net income margin. The results show that the company has achieved consistent profitability throughout the year.}

The Monthly Profit and Loss for Alpha Project is presented in this chapter. This report contains data from January to December of 2020. Data was collected on revenue, product sales, other services, cost of goods sold (COGS), gross profit (GP), operating expenses (OPEX), labor costs, rent costs, material costs, maintenance costs, IT costs and sales & marketing costs. In addition to these metrics it also shows Earnings Before Interest Taxes Depreciation & Amortization (EBITDA) margins as well as Earnings Before Interest & Taxes (EBIT) margins and Net Income Margins. 

Revenue for Alpha Project was steady throughout 2020 at $50 100 each month with no significant fluctuations or changes in trend over time. Product sales were also constant at $50 100 per month which indicates that there is a strong demand for their products in the market place. Other services did not generate any revenue during this period which is expected given that they are not offering any additional services outside of product sales at this time. 

Cost of Goods Sold was consistent throughout 2020 with an average value of $19 750 per month indicating that they have been able to maintain a high level of efficiency when producing their products while keeping COGS low enough to ensure profitability on each sale made by Alpha Project. Gross Profit was also steady at $30 350 each month showing a healthy GP margin across all months at 60%. 

Operating Expenses totaled $28 095 each month with Labor Cost accounting for 78% ($22 000) followed by Rent Costs ($2 500), Material Costs ($500), Maintenance Costs ($500) and Others ($125). IT expenses accounted for 1% ($300) while Sales & Marketing accounted for 5% ($1 500). Lease Fees account for 2% ($670). These figures indicate that Alpha Projects OPEX are relatively low compared to its peers in the industry which has enabled them to maximize their profits despite having lower revenues than competitors who may be spending more on overhead costs such as labor or rent etc.. 

EBITDA Margin averaged 3.16% across all months indicating that Alpha Project is able to generate positive cash flows even after taking into account depreciation and amortization expenses associated with running its business operations effectively during 2020. EBIT Margin averaged 1.83%, slightly lower than EBITDA due mainly to interest payments being incurred during some months due to financing activities undertaken by Alpha project during certain periods within 2020 while Net Income Margin averaged 1