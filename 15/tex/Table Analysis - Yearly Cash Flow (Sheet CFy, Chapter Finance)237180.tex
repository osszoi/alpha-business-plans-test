

\subsection{Yearly Cash Flow} \label{sec:title}
\nonumsidenote{This section provides an analysis of the yearly cash flow for Alpha Project. The data shows a positive net income from year 1 to 5, with an increase in operating cash flow over the same period. Capital expenditures have been financed mainly through share issues, with no bank payments or loans being taken out during this time. This has resulted in a net cash increase and a corresponding rise in the company's cash balance.} 

The table provided displays Alpha Project's yearly financial performance from year 1 to 5. In terms of net income, there was a steady increase from year 1-5, starting at 8,814 and increasing to 1,129,810 by year 5. This is likely due to increased sales or improved operational efficiency within the company. Additionally, depreciation and amortization (D&A) expenses remained constant throughout this period at 8000 per annum, indicating that these costs were relatively stable over time. 

In terms of working capital (WC), there was a decrease in WC expenses over this five-year period from 39155 in year 1 to 28042 by year 5. This could be attributed to improved inventory management strategies or better accounts receivable collection practices within the company during this time frame. As a result of both increased revenues and decreased WC expenses during this period, operating cash flow rose significantly from 55,969 in year one to 1,165,852 by year five - representing more than 10 times growth over that time span. 

Capital expenditures (CAPEX) were financed mainly through share issues rather than bank payments or loans during this five-year period as evidenced by the zero values for CAPEX - fbank (-) and CAPEX - fshare (-). As such there were no CFs generated from bank payment/loans (=). However there was still an overall net increase in cash due to CFs generated through investments (+). Specifically 150000 was invested into Alpha Project at years one and two while 25000 was invested into Alpha Project at year three resulting in total net cash inflows of 328477 for years 3-5 respectively as well as total net cash inflows of 94031 for years 1-2 respectively . These investments helped fuel further growth within Alpha Project resulting in an overall rise of their respective Cash Balance which began at 55999 in Year One before eventually reaching 2198403 by Year Five - representing almost 40 times growth over that same timeframe . 

Overall it can be seen that Alpha Projects financial performance has been quite strong between Years One and Five with significant increases seen across all metrics measured here including Net Income , Operating Cash Flow , Net Cash , Investment Capital Inflows and ultimately Cash Balance . Such results are indicative of sound financial planning on behalf of Alpha Projects Management team who have managed to capitalize on opportunities presented while also controlling costs where possible . Going forward it would be beneficial if similar trends continue so that