\nonumsidenote{Summary}
The table displays the profit and loss statement for Alpha Project over five years, with monthly data. The company's revenues have been increasing steadily, reaching 6.5 million US dollars in the last year, mainly driven by product sales. However, cost of goods sold has also increased significantly, reducing gross profit margin from 60.58% to 56.92%. Operating expenses have also risen, particularly labor cost and sales and marketing expenses. As a result, EBITDA margin has fluctuated between 3.16% and 22.22%, while net income margin has ranged from 1.47% to 17.38%.

Alpha Project Yearly Profit and Loss where m = month and y = year

The profit and loss statement of Alpha Project shows that the company has experienced steady growth in revenues over the past five years, with a total increase of more than ten times from the first to the fifth year (601,200 US dollars to 6,500,492 US dollars). This growth is mainly due to an increase in product sales which account for all of Alpha Project's revenue.

However, this growth in revenue is accompanied by significant increases in cost of goods sold (COGS), which rose from 237,000 US dollars in year one to almost three million US dollars in year five. As a result of these increases in COGS as well as other operating expenses such as labor costs ($1,746,750) and sales/marketing expenses ($212673), Alpha Project's gross profit margin decreased from its initial value of around sixty percent (60%) to around fifty-seven percent (57%). 

In terms of profitability measures such as EBITDA margin or net income margin; they show fluctuations that are partly due to changes within different expense categories like rent or maintenance costs but mostly because COGS was rising faster than revenues were growing during some periods. The EBITDA margin ranged from a low of 3.16% to a high of 22.22%, while the net income margin varied between 1.47% and 17.38%.

One area where Alpha Project could improve its profitability is by reducing operating expenses, particularly labor costs and sales/marketing expenses, which have been increasing steadily over the years. Another potential strategy would be to increase product prices in order to maintain or even improve gross profit margins despite rising COGS.

In conclusion, Alpha Project has experienced steady revenue growth over the past five years but has also seen significant increases in cost of goods sold and operating expenses leading to a decrease in gross profit margin. Profitability measures such as EBITDA margin and net income margin fluctuate due to changes within different expense categories but indicate that reducing operating expenses could improve overall profitability for the company.