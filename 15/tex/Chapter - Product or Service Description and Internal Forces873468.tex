\section{Descripción del Producto o Servicio} \label{sec:descripcion_producto}Venezuelan Hot Dogs es una empresa creada por Lucas, con la dirección en Calle La Colina Edificio 32 Av Fuerzas Armadas Caracas. El objetivo de la empresa es vender hot dogs gourmet a la comunidad de Doral. Esta empresa cuenta con una ventaja competitiva basada en la calidad y el servicio que ofrece a sus clientes. Los objetivos a corto y largo plazo son vender 4 millones y tener 5 sucursales.
La estructura legal de Venezuelan Hot Dogs es una Compañía de Responsabilidad Limitada (LLC). Lleva en operación 4 años, con 5 miembros clave en su equipo de gestión: Lucas - Presidente, Raúl - Vicepresidente y Andrés - Gerente de Operaciones. Actualmente, cuenta con 5 empleados y su historia se remonta al pequeño puesto que Lucas abrió en Nueva York, el cual fue todo un éxito.
El mercado al que se dirige Venezuelan Hot Dogs cuenta con 30 billones y demanda productos o servicios similares que satisfagan las necesidades de sus clientes gracias al sabor, rapidez y comodidad que ofrecen. Las tendencias en el sector apuntan a consumir alimentos saludables y rápidos, lo que representa grandes desafíos como una competencia altísima, precios bajos e incertidumbres en la cadena logística. Además, los principales competidores son las grandes cadenas como McDonalds o Wendys, que ejercen presión sobre el precio. Por otro lado, el poder negociador entre proveedores es muy elevado debido a la escasez existente de insumos básicos, como pan. Por otro lado, no hay poder negociador entre los compradores debido a los altísimos precios existentes en este tipo de productividad rápida. Finalmente, existe una amenaza constante por parte de nuevos entrantes, así como sustitutivos saludables, lo que representa un gran reto para Venezuelan Hot Dogs.
Los productos son hot dogs gourmet preparados con recetas únicas adaptadas para satisfacer los gustos locales del mercado venezolano. Además, contarán con variedad en sabores, así como mayor facilidad para ser servidos en menor tiempo, comparándolo con otros restaurantes de hot dogs tradicionales, lo cual le dará mayor valor agregado al producto final, diferenciándose así de sus competidores directamente con respecto a calidad, precio y velocidad de servicio. 

Además, las fortalezas internas son la receta única localizada al mercado venezolano y las debilidades internas son la falta de capital, mientras que las oportunidades externas son la existencia de gran cantidad de personas potencialmente que buscan innovación y el mercado venezolano, mientras que las amenazas externas son la economía en recesión, alta competición y precio bajo en los insumos para producción del producto final. 

Para proveer una solución estratégica, Venezuelan Hot Dogs cuenta con una estrategia de marketing y comunicación enfocada a la comunidad venezolana de la zona, por lo que se utilizarán medios tales como redes sociales, manteniendo una transparencia absoluta entre la comunicación y los clientes. Se realizarán campañas continuas y promociones publicitarias antiguas y diferentes que utilice el equipo de ventas y marketing para lograr un mayor número de clientes y mejoras de la venta y la experiencia al cliente final.
La estrategia de ventas se basará en su precio alto en frente al sector de comida rápida, lo que significa que habrá mayores precios con garantía de calidad y velocidad. En vista del sector al que se encuadra Venezuelan Hot Dogs, se utilizarán medios tales como publicaciones sobre digital, campañas de voz, boca a boca, folletos en las zonas malls y eventos en lugares relevantes por el mercado meta objetivo. Finalmente, para la repartición, se harán envíos directos a la zona de mall Sawgrass y por último se utilizará una estrategia de diferenciación definida sobre todo en la presentación de paquete y la conveniencia al servicio que ofrece Venezuelan Hot Dogs con respecto a otros restaurantes fast.
