\\nonumsidenote\\{resumen} Esta sección del plan de negocios del Proyecto Alpha presenta los resultados financieros anuales a través del análisis de las ganancias y pérdidas. Los datos indican que el margen bruto ha disminuido levemente con el tiempo, mientras que los márgenes EBITDA, EBIT y neto han aumentado significativamente.
\section{Resultados financieros anuales}El Proyecto Alpha ha mostrado un crecimiento constante en sus ingresos desde su inicio en enero de 2020 hasta diciembre de 2024. El gráfico a continuación muestra los ingresos totales del Proyecto Alpha durante este período.
Por favor, corrige este texto si es necesario. Arregla la puntuación, la gramática, los espacios, etc. NO AGREGUES TEXTO NUEVO. Recuerda que debe estar en español.
¡Observa la gráfica! Muestra la cantidad de estudiantes que asistieron a un evento deportivo durante los primeros cinco meses del año. A partir de la gráfica, podemos ver que la asistencia aumentó en marzo y luego disminuyó gradualmente hasta mayo. En marzo, casi 400 estudiantes asistieron al evento, mientras que solo alrededor de 150 estudiantes asistieron en mayo. ¡Qué interesante!
Figura 1: Ingresos totales del Proyecto Alpha desde enero de 2020 hasta diciembre de 2024.
Por favor, asegúrate que revisen y corrijan cualquier error de puntuación, gramática, espaciado y otros en el texto. NO DEBES AGREGAR NUEVO TEXTO. Recuerda que todo debe estar en español.
Los ingresos han crecido gradualmente durante este período, pasando de 601,200 $US en enero de 2020 a 6,500,492 $US en diciembre de 2024 (ver Tabla 1). Esta tendencia positiva se refleja en la fuerte demanda por los productos y servicios ofrecidos por el Proyecto Alpha durante este período. La mayor parte del crecimiento se debe al aumento significativo en las ventas directas al consumidor (VPC), ya que representan el 99% del total general de los ingresos generados por el Proyecto Alpha durante este período (ver Figura 2).
Por favor, corrige este texto si es necesario. Arregla la puntuación, la gramática, los espacios, etc. NO AÑADAS NINGÚN TEXTO NUEVO. Recuerda que debe estar en español.
Por favor, corrige este texto si es necesario. Arregla la puntuación, la gramática, los espacios, etc. NO AÑADAS NUEVO TEXTO. Recuerda que esto debe estar en español. 

No hay ningún texto que corregir, ya que solo hay una imagen proporcionada. Si se desea, se puede agregar una leyenda o título en español para describir la imagen.
o y marzo de 2021}

Esta figura muestra el porcentaje total de VPC (ventas por teléfono celular) en comparación con las ventas totales durante el periodo de enero a marzo de 2021. Podemos observar que el porcentaje total de VPC aumentó gradualmente cada mes, comenzando en un 15% en enero y llegando a un 25% en marzo. Este aumento es una clara señal de la creciente popularidad de las compras en línea a través de dispositivos móviles en la actualidad.
