Venezuelan Hot Dogs es una empresa que se dedica a la venta de hot dogs gourmet, ofreciendo productos de alta calidad y un servicio excepcional a sus clientes. La empresa tiene como objetivo vender 4 millones de dólares en el primer año y tener cinco sucursales en los próximos tres años. A largo plazo, esperan seguir expandiendo su negocio para convertirse en líderes del mercado.

La estructura legal de Venezuelan Hot Dogs es una Sociedad de Responsabilidad Limitada (LLC) y ha estado operando durante cuatro años. Los miembros clave del equipo directivo son Lucas, presidente; Raul, vicepresidente; y Andres, gerente general. Actualmente cuentan con cinco empleados.

El mercado meta son las personas que viven en Doral y buscan un sabor único e innovador en sus comidas rápidas. El tamaño del mercado alcanza los 30 mil millones de dólares anuales. La demanda actual por los productos ofrecidos por Venezuelan Hot Dogs es alta debido a que hay mucha gente visitando centros comerciales buscando opciones rápidas para comer.

Los principales competidores son todos los restaurantes ubicados en el centro comercial Sawgrass Mall. Las tendencias muestran que las personas están buscando opciones de comida rápida saludable y sabrosa. Los principales desafíos que enfrenta la industria son la alta competencia, los precios bajos y la cadena de suministro.

Para capitalizar sus fortalezas y oportunidades mientras aborda sus debilidades y amenazas, se implementará una campaña de marketing enfocada en la comunidad venezolana utilizando redes sociales, publicidad digital, boca a boca y volantes en el centro comercial Sawgrass Mall.

El mercado meta son los miembros de la comunidad venezolana en Doral. La estrategia de precios será alta para el sector de comida rápida. El producto se distribuirá desde un puesto ubicado dentro del centro comercial Sawgrass Mall con un empaque atractivo enfocado en comodidad.

En cuanto al recurso humano actualmente trabajan diez empleados entre tiempo completo y parcial. En el futuro se planea contratar amigos y familiares pertenecientes a la comunidad venezolana local. Las políticas actuales incluyen lo establecido por ley para empleados horarios.

Los planes futuros incluyen abrir seis sucursales más en el área de Doral durante el próximo año y luego expandirse a seis nuevas sucursales por año en todo Miami. La cadena de suministro actual se compone de tres proveedores para las salchichas, un proveedor para el pan y todos los demás ingredientes se compran en Costco. El inventario se administra semanalmente según las ventas registradas.

En conclusión, Venezuelan Hot Dogs es una empresa que tiene como objetivo ofrecer productos de alta calidad y servicio excepcional a sus clientes, con una estrategia enfocada en la comunidad venezolana en Doral. A través de su campaña de marketing, esperan capitalizar sus fortalezas mientras abordan sus debilidades y amenazas para convertirse en líderes del mercado.