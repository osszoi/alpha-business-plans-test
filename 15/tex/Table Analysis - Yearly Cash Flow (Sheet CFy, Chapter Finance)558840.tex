

\subsection{Yearly Cash Flow}\label{sec:title}
\nonumsidenote{The yearly cash flow analysis of Alpha Project shows that the net income has increased steadily over a five-year period. Additionally, the operating cash flow and cash balance have also grown significantly in this time. This growth is due to an increase in D&A expenses, as well as a decrease in WC expenses.}

The table data provided provides a detailed look into Alpha Project's financial performance over the past five years. The first column shows the year, with columns two through six showing the net income, D&A expenses, WC expenses, operating cash flow (OCF), CAPEX - fbank (CAPEX from bank payment or loan) and CAPEX - fshare (CAPEX from shareholder payment). 

Looking at column two of the table reveals that Alpha Project's net income has been increasing steadily since year 1. In year 1 it was 8,814; by year 5 it had increased to 1,129,810. This trend indicates that Alpha Project is doing well financially and is likely to continue its upward trajectory in future years. 

D&A expenses have remained relatively constant throughout this period; they were 8000 for each of the five years shown in the table. This suggests that Alpha Project is managing its assets efficiently without needing to invest heavily in new equipment or technologies every year. 

WC expenses have decreased significantly over this time frame; they were 39155 for year one but had dropped to 28042 by year 5. This suggests that Alpha Project has become more efficient at managing its workforce costs as it has grown larger and more established over time. 

Operating cash flow (OCF) reflects both changes in net income and changes in WC expenses; while OCF was only 55,969 for Year 1 it had increased significantly by Year 5 when it reached 1,165,852. The combination of increasing net income and decreasing WC expense suggest that Alpha project is becoming increasingly profitable as time goes on which bodes well for future growth potentials. 

 CAPEX - fbank remained 0 throughout these five years indicating that no additional capital was needed from external sources such as banks or loans during this period; however CAPEX - fshare did increase from 0 to 150000 between Years 4 and 5 suggesting some additional investments may be necessary going forward if further expansion plans are made reality . 

 Finally CF from Bank Payment / Loan (=) was always 0 during these five years meaning there were no payments received or made related to loans during this timeframe which suggests either there were no loans taken out or any existing ones have already been paid off prior to Year 1 .  

 Overall ,the data provided indicates steady financial growth for Alpha project over a five-year period with significant increases seen across all relevant metrics including Net Income , Operating Cash Flow ,and Cash Balance . These trends suggest continued success going forward if proper investments are