

\subsection{Monthly Profit and Loss}\label{sec:title}
The following table provides a detailed analysis of the revenue, product sales, cost of goods sold, gross profit, operating expenses, labor costs, rent, material costs, maintenance costs and others (negative recoveries), IT expenses, sales and marketing expenses and lease fees for Alpha Project from January to December. The table also provides insights into the company's EBITDA margin, EBIT margin and net income margin. 

\nonumsidenote{Summary: This section looks at the monthly financial performance of Alpha Project from January to December. It includes an analysis of revenue sources such as product sales and other services along with an analysis of operational expenses like labor cost, rent ,material cost etc. Further it includes an insight into the profitability metrics such as EBITDA margin ,EBIT margin and net income margin.}

From the table we can see that Revenue from product sales stays constant throughout all 12 months while other services do not generate any revenue in any month. Cost of Goods Sold (COGS) is also consistent across all months with 19750 being spent each month on it. Gross profit stands at 30350 which is a good sign as it indicates that there is enough money left after paying for COGS to cover operational expenses & other costs incurred by Alpha Project. 

Operating Expenses are made up primarily by Labor Costs (22000), Rent (2500), Material Cost(500) & Maintenance Cost(500). Other Operating Expenses include IT(300), Sales & Marketing(1500) & Lease Fee (670). These add up to 28095 leaving us with 1 585 in EBITDA which is 3% of total revenues generated by Alpha Project during this period. After deducting Depreciation & Amortization charges we get 918 in EBIT which is 1 83%of total revenues generated by Alpha Project during this period .After deducting Interest expense we get 918in Earnings Before Tax(EBT). On further deduction for Current Taxes due we get 734 in Net Income which is 1 47 %of total revenues generated by Alpha Project during this period .Gross Profit Margin remains consistent at 60 58 %for all 12 months indicating that Alpha project has been able to maintain its pricing strategy effectively over these months . 

 Overall ,the data suggests that Alpha project has been able to maintain a healthy level of profits despite increasing operating expenses over these 12 months .However more focus needs to be given on reducing operating expenditure while simultaneously increasing top line growth so as to increase profitability margins going forward .