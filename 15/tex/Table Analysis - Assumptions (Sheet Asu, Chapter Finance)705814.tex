

\subsection{Assumptions}\label{sec:assumptions}
\nonumsidenote{This section contains an analysis of the assumptions used in the Alpha Project business plan, including growth, inflation, packing and shipping costs, material costs, maintenance costs, other (negative) recoveries, sales and marketing costs, taxes incurred and results summary.} 
The Alpha Project business plan made several assumptions about their expected revenue growth over a five-year period. The assumption was that revenue would grow at 80\% annually. Inflation was assumed to be 3\%, as were packing and shipping costs. Material cost was also assumed to be 1.00\%, with maintenance cost at 0.50\%. Other (negative) recoveries were estimated at 0.25\%. Sales and Marketing expenses were estimated at 3.00 \%, along with taxes incurred at 20 \%. 

Using these assumptions over five years resulted in total revenue of $13,834,026 US dollars and maximum revenue of $6,500,492 US dollars during that time period. Total Net Income After Tax (NIAT) for the same period was estimated to be $2,069,283 US dollars with maximum NIAT being $1129 810 US dollars during the same period. The minimum investment required for this project is estimated to be $148 353 USD  . Lastly , NPV@10 % is calculated as 2 543 467 76 USD while IRR is 186 \%. 


The Alpha Project business plan has taken into account various factors such as inflationary pressure on pricing as well as operational expenses such as packing/shipping costs which will have a direct impact on profitability margins over the long term duration of this project . Additionally , it is important to note that sales & marketing efforts will play an important role in driving revenues while taxes incurred need to be taken into account when calculating net income after tax figures . It can also be observed that there is a large difference between total revenue earned versus maximum revenue earned within a given year – indicating potential fluctuations in demand which could cause large variations in earnings from one year to another . Finally , it should also be noted that although NPV@10 % returns are high , they do not take into account any potential risks associated with this project such as currency exchange rate fluctuations or political instability within target markets .  

Overall , the Alpha Project business plan has taken into consideration various factors which are necessary for successful implementation of any venture . These include considerations related to pricing pressures due to inflationary trends , operational expenses associated with running this venture & taxation implications amongst others . Furthermore , by taking into account potential market variations & risk factors associated with international ventures – this business plan provides a comprehensive framework for success within its chosen industry sector .