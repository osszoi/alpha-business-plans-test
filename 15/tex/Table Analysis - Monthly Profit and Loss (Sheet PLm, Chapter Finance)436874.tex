

\subsection{Monthly Profit and Loss (m/y)}\label{sec:title}
\nonumsidenote{This section provides a summary of the monthly profit and loss for the Alpha Project. The data shows that revenue, product sales, cost of goods sold, gross profit, operating expenses, labor costs, rent, material costs, maintenance costs and other expenses remain consistent throughout the year. However, IT expenses increase in December by 5\%, while Sales and Marketing increases by 10\%. Additionally, lease fees remain consistent throughout the year. EBITDA margins are 3.16\% with EBIT margins at 1.83\% and Net Income margins at 1.47\%. Gross Profit Margin is 60.58 \%.}

The Alpha Project's monthly profit and loss report shows that revenue from product sales remains stable throughout the year at \$50 100 per month as does Cost of Goods Sold (COGS) which stands at \$19 750 per month on average for all 12 months of the year. This results in a Gross Profit of \$30 350 per month or 60.58 \% margin on average for all 12 months of the year; this indicates that there is potential to increase profitability through increasing sales or reducing COGS if needed in order to meet targets set out in any business plan for this project. 

Operating Expenses stand at an average of \$28 095 per month with Labor Costs being a major expense at 22 000 per month on average; Rent follows closely behind with an average monthly cost of 2500 while Material Costs also remain constant over all 12 months standing at 500 each month on average as do Maintenance Costs which also stand at 500 each month on average as well as Other Expenses such as negative recoveries standing at 125 each month on average over all 12 months reported here; however IT expenses increase slightly during December by 5\%, while Sales and Marketing increases by 10\%. Lease Fees also remain constant over all 12 months standing at 670 each month on average over all 12 months reported here indicating no additional overhead costs associated with leasing property or equipment for this project during these twelve months reported here.. 

EBITDA stands at 1585 after accounting for Operating Expenses whilst EBIT stands slightly lower than EBITDA due to Depreciation & Amortization (D&A) adding 667 to Operating Expenses resulting in an EBIT margin of 1.83 \%; Interest charges are not applicable during these twelve months reported here so Earnings Before Tax (EBT) is equal to EBIT; Current Tax charges then add 184 resulting in Net Income margins averaging 1 47 \% across these twelve reported here showing potential room for improvement if desired - either through increased sales or reduced operating expenses - when creating a business plan for this project going forward into future periods beyond those shown here.. 

