 

\section{Análisis de los datos de la tabla} \label{sec:title}

El proyecto Alpha requiere un análisis cuidadoso de los datos para evaluar su viabilidad. Primero, se han hecho algunas suposiciones básicas sobre el crecimiento, que se estima en un 80 por ciento, así como la inflación del 3 por ciento. Además, hay otros costos involucrados como el embalaje y el envío (1 por ciento), los costos de materiales (1 por ciento) y los costos de mantenimiento (0.50 por ciento). También hay otros gastos no recuperables (0.25 por ciento) y gastos en ventas y marketing (3 por ciento). Finalmente, hay impuestos aplicables del 20%.

Con base en estas suposiciones, se ha calculado que las ganancias totales durante los próximos 5 años serán 13 834 026 dólares americanos con un máximo de 6 500 492 dólares americanos durante este período. Esta cantidad es mayor que el monto inicial necesario para iniciar el proyecto Alpha ($148 353). Los ingresos netos antes de impuestos durante este mismo período serán 2 069 283 dólares americanos con un máximo de 1 129 810 dólares americanos. Además, se ha calculado que el Valor Presente Neta (VPN) del proyectoa 10% es $2 543 467.76 y su tasa interna de retorno es 186%. 

En resumen, los resultados obtenidose indican que el Proyect Alpha es viable económicamente ya que generaría beneficios significativossin necesidad deponer muchocapital inicial. El VPN indicaqueelproyectopodría generar gananciasmayoresqueelcapitalinvertidoenelmismo si seconsideraelvalordeltiempoyserealizaunainversiónadecuadaenmarketingparalograrunmayornúmerodedventasygananciastotalesdurantecadaejercicioeconómicoqueseplaneeenelproyectodeAlphaProject.  

Resumen: El análisis realizado sobre los datosen