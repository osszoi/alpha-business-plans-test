

\subsection{Monthly Cash Flow Analysis for Alpha Project}\label{sec:title}
\nonumsidenote{This section provides an analysis of the monthly cash flow data for Alpha Project. It begins with a summary of the key points, followed by an overview of the cash flow and capital expenditure, a discussion of the net income and operating cash flow, and a conclusion.}

Summary: This section provides an analysis of the monthly cash flow data for Alpha Project. The cash flow from bank payment/loan is zero in all months, indicating that no loans were taken out during this period. Capital expenditure was only incurred at month 0 in the form of funds raised through equity issuance. Net income remained steady throughout all months due to depreciation and amortization being equal to net income each month. Operating cash flow increased steadily over time as WC decreased from month 0 to month 12. 

The table provided shows the monthly cash flows for Alpha Project from January (month 0) to December (month 12). At month 0, Alpha Project raised $150 million through equity issuance, which is reflected in both its capital expenditure and net cash columns. As expected, there was no entry in either column at any other time during this period since no further capital expenditures or loans were taken out by Alpha project during this period. 

Net income remained constant throughout all months due to depreciation and amortization being equal to net income each month ($734 million). Working capital also remained steady at negative 246 million dollars until December (month 12), when it became zero due to debt repayment or other factors not specified in the table data. As such, operating cash flow increased steadily over time as WC decreased from month 0 to month 12 ($1,647 million - $1706 million). 

In conclusion, this analysis has shown that while there was some variation in working capital over time for Alpha Project's monthly cash flows, overall these figures remained relatively stable throughout all twelve months given that no additional loans or expenditures were taken out by the company during this period. The data provided also indicates that despite incurring significant initial costs associated with raising equity funds at Month 0 ($150 million), these costs are likely outweighed by what will be gained through higher operating profits over time as WC decreases going forward into 2021 and beyond.