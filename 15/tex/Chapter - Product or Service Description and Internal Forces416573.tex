

\section{Alpha Project} \label{sec:alpha-project}
Alpha Project is a Venezuelan Hot Dog business located in Caracas, Venezuela, founded by Lucas. The company provides two types of hot dogs: Hot Dog 1 and Hot Dog 2. The target market consists of people living in the Doral area. Alpha Project's competitive advantage lies in its high quality products and services. Its short-term goal is to sell 4 million hot dogs and open 5 branches, while its long-term goal is to become a regional leader in the fast food industry. 

The legal structure of the business is that of a limited liability company (LLC). It has been operating for four years now with five employees on board; Lucas as President, Raul as Vice President and Andres as Operations Manager. The current size of the business includes annual revenue from sales of hot dogs which have been increasing consistently over time since its foundation due to customer satisfaction with their products and services. 

The main competitors are all other food restaurants in the Sograss Mall, where Alpha Project operates from. The market size for this kind of product is estimated at 30 billion dollars annually, with a high demand for such services due to convenience and taste provided by Alpha Project�s hot dogs. Major trends within the industry include customers looking for healthier options than traditional fast food items such as burgers or pizza, leading to higher prices being charged by companies like Alpha Project who offer healthier alternatives while still providing convenience and speed when ordering their product. 

Alpha Projects offers two types of hotdogs: Hot Dog 1 & 2 which meet customer needs through taste, speed & convenience while also providing local adaptations based on Venezuelan culture & community preferences through unique recipes developed by their team members over time. Their internal strengths include unique recipes created by their team members combined with strong brand recognition within their target demographic along with skilled workforce trained according to standards set out by management resulting in consistent quality across all orders placed with them over time . Internal weaknesses include lack of capital preventing further expansion into new markets as well external opportunities available such as partnerships or technological advancements allowing them access into new markets they would not have had access before due to financial constraints . External threats faced by Alpha Projects include competition from major chains like McDonalds or Wendies along with regulatory changes or economic downturns reducing consumer spending power leading to lower overall profits earned throughout any given year . 

 All these factors have been taken into consideration when formulating marketing strategies targeting Venezuelan Community living in Doral Area using social media campaigns , digital advertising , word-of mouth marketing & fliers distributed within Sawgrass Mall itself . Pricing strategy employed uses higher prices compared to traditional fast food outlets but still manageable enough so customers can afford it without feeling burdened financially . Distribution follows standard model used throughout fast food industry where orders are prepared at one location then sold directly from another location at Sawgrass Mall itself allowing customers direct access without having wait times associated when ordering online or