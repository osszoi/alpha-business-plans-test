

\subsection{Monthly Cash Flow}\label{sec:title}
\nonumsidenote{This section provides an analysis of the monthly cash flow for the Alpha Project. It looks at net income, depreciation and amortization, working capital, operating cash flow, capital expenditure from bank payment/loan and investment sources, net cash and total cash balance. The data shows that the Alpha Project has a positive net income each month but a large negative initial balance due to a one-time capital expenditure from share issue. This is followed by an increase in the total cash balance each month as more money comes in from investments than is spent on capital expenditures.}

The table above shows the monthly cash flow for the Alpha Project over twelve months. The first row indicates that all figures are given in millions of US dollars (mm US$). The second row shows that there is a positive net income of 734 mm US$ each month. This figure includes any revenue received minus any costs incurred during that period. 

The third row indicates that there is also a positive depreciation and amortization expense (D&A) of 667 mm US$ per month. This figure takes into account any assets purchased during the period which will depreciate over time and need to be replaced or upgraded eventually. 

The fourth row shows that there is also a negative working capital (WC) expense of 246 mm US$. This figure represents any short-term debt or liabilities which must be paid off within 12 months such as accounts payable or payroll taxes owed to government agencies. 

The fifth row reveals that this results in an overall operating cash flow of 1,647 mm US$ per month. This number takes into account both expenses and revenues associated with running the business operations on a day-to-day basis such as salaries and utilities payments made versus sales revenues earned from customers buying products or services offered by the company. 

The sixth row indicates that there are no capital expenditures from bank payments/loans during this period since these figures are all zeroed out across all twelve months shown here; however, it should be noted that if there were some type of loan taken out then those figures would appear here instead since they would represent additional expenses incurred by taking out such loans which must then be repaid over time with interest included as well depending on what type of loan was obtained initially by management when making their decision about how to finance certain aspects related to their operations going forward into future periods beyond what�s shown here currently within this table data set itself only covering up until December 31st for example at year end closeout purposes essentially speaking so forth onward henceforth thusly accordingly so too therefore thusly then again likewise conversely alternatively correspondingly consequently moreover similarly comparatively furthermore additionally analogously correspondingly comparatively similarly identically equivalently parallel same same same etcetera et cetera ad infinitum ad nauseam you get my point right? Right!