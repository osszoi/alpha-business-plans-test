\section{Flujo de Caja Anual}El flujo de caja anual del proyecto Alpha es una herramienta importante para comprender la capacidad financiera de la compañía. Estos datos muestran tanto los ingresos como los gastos totales del proyecto durante los últimos cinco años, desde 2015 hasta 2019. En la tabla se pueden identificar las principales fuentes de ingresos y gastos, incluyendo ingresos netos, depreciación y amortización (D&A), costos laborales (WC), y gastos de capital (CAPEX).
Los resultados muestran que el Proyecto Alpha ha obtenido un flujo de caja operativo positivo en todos los años. En 2015, este fue de 55.969 euros; mientras que en 2019 ascendió a 1.165.852 euros. Esto significa que la compañía generó suficientes fondos para cubrir sus gastos operativos durante todo ese periodo sin necesidad de recurrir a financiamiento externo o fondearse con inversiones propias.
En cuanto al CAPEX, se puede observar que éste fue mayor durante los dos primeros años del Proyecto Alpha: 150 mil euros en ambas ocasiones; luego disminuyó drásticamente hasta ser nulo en los últimos tres años siguientes (2017-2019). Este descenso significativo podría indicar un menor nivel de inversión por parte del proyecto para mantener su actividad productiva o bien un mejor uso de recursos en aquellas inversiones y equipamiento, ya que el proyecto se ha vuelto autofinanciable con el flujo de caja positivo creado por el ingreso neto operacional durante todos esos periodos posteriores al 2015 (2016-2019).
Por otro lado, no hay evidencia dentro del flujo de caja anual sobre préstamos en forma de ingreso o egreso por el fondo bancario (CF from Bank Payment/Loan = 0). Esto es importante, pues indica que el proyecto no ha optado por financiarse mediante préstamos externos en ninguno de los cinco años siguientes a su constitución (2015-2019). De esta manera, se confirma el alto grado de autofinanciabilidad del proyecto y su capacidad de generar ingresos suficientes para cubrir sus gastos cotidianos transcurridos durante el periodo analizado en esta tabla (2015-2019). En resumen, el flujo de caja anual mostrado por la tabla demuestra que el Proyecto Alpha ha sido capaz de generar ingresos suficientes para cubrir sus gastos cotidianos transcurridos durante el periodo analizado en esta tabla (2015-2019), lo que confirma el alto grado de autofinanciabilidad del proyecto y su ausencia de recurso a financiamiento externo mediante préstamo bancario durante todo este periodo temporal analizado.
