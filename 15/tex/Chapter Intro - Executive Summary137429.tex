Venezuelan Hot Dogs is a small business that aims to sell gourmet hot dogs in the Doral area. The company's unique recipe sets it apart from its competitors, and its target market is the Venezuelan community living in Doral. The size of the market for fast food restaurants is estimated at around 30 billion dollars, with high competition and low prices being major challenges facing the industry.

Venezuelan Hot Dogs offers two types of Venezuelan-style gourmet hot dogs that meet customer needs for taste, convenience, and speed of service compared to other fast-food restaurants' products. The company's internal strengths include its unique recipe, while its internal weaknesses are a lack of capital. External opportunities for the company include people looking for innovative tastes and a huge Venezuelan market that wants these flavors.

The company plans to differentiate itself from competitors by focusing on the Venezuelan community with great packaging and convenience while serving faster than other fast-food restaurants. Its pricing strategy involves charging higher prices than other fast-food restaurants in the sector.

Currently, there are ten employees working for Venezuelan Hot Dogs, but in the future, the company plans to hire friends and family members from within the Venezuelan community. Basic HR policies as per law are followed for hourly employees. The company also has plans to train its personnel based on its standards.

Venezuelan Hot Dogs currently has one small facility in Doral where it prepares food along with a small store in Sawgrass Mall. Its plans involve opening six more stores within a year in Doral followed by six stores per year across Miami areas. The operational processes involve three suppliers for sausage, one supplier for bread, while sauces and other ingredients are bought from Costco. Inventory management is done weekly based on sales analysis during that week.

In conclusion, Venezuelan Hot Dogs aims to sell gourmet hot dogs to the Venezuelan community living in Doral by offering unique recipes that set it apart from competitors' products. With an effective marketing and sales strategy, the company plans to differentiate itself by providing great packaging and convenience while serving faster than other fast-food restaurants. Its operational processes involve managing inventory weekly based on sales analysis during that week.