

\nonumsidenote\{The yearly profit and loss of Alpha Project shows that revenues, product sales, gross profits and EBITDA have increased steadily over the five-year period. Costs of goods sold and labor costs have also grown in line with revenues. Operating expenses have been kept under control, with rent, material cost, maintenance cost, other (negative recoveries) and IT costs increasing at a slower rate than revenue. Sales and marketing costs as well as lease fees remain relatively stable. The net income margin has seen an increase from 1.47\% to 17.38\%. This indicates that Alpha Project is making good progress towards achieving its financial goals}\label{sec:summary}

The yearly profit and loss of Alpha Project for the past five years provides insight into the company's overall financial performance. Revenues have steadily increased from \$601200 in m0y0 to \$6 500 492 in m5y5; a growth of 883\%. Product sales have also grown at a similar rate as revenues indicating consistent demand for Alpha Project's products or services during this time period. Other services provided by the company however remain practically unchanged throughout the five-year period with no significant contribution to total revenues or profits. 

Costs of goods sold (COGS) has risen from \$237000 in m0y0 to \$2 800 188 in m5y5; an increase of 1179\%. Labor costs have increased even more significantly over this time frame rising from \$264 000 in m0y0 to \$1 746 750 in m5y5; representing a 568\% increase over five years. Rent expenses also rose significantly during this time frame increasing by 400\%, while material cost saw an increase of 1050\%, maintenance cost 1050\% and other (negative recoveries) 1075\. IT expenses were kept relatively constant throughout this period with only a 28\% rise between m0y0 and m5y5\. 

Sales and marketing costs remained relatively stable during this time frame increasing by only 84\. Lease fee was kept constant at \$670 for all months within the year except for one instance where it reached \$804 0 due to additional rental fee incurred on premises owned by Alpha Projects' subsidiary companies\. 

Gross profit margins remained consistent over these five years hovering around 60\. Similarly EBITDA margins stayed between 3\-22 percent during these five years indicating that operating expenses were kept under control despite increases in labor costs\. EBIT margins followed suit remaining between 1\-21 percent over these five years while net income margin grew substantially from 1\-17 percent suggesting improved efficiency within operations\. Interest payments were not applicable throughout this period so there was no effect on earnings before tax (EBT). Current taxes paid out rose gradually reaching 28245 3inm 5 y 5 representing an effective tax rate of 26 4percent which is inline