La empresa venezolana Venezuelan Hot Dogs se dedica a la venta de hot dogs gourmet en Caracas. Ofrecen dos tipos de hot dogs diferentes y su mercado objetivo son las personas que viven en Doral, con una ventaja competitiva basada en la calidad y el servicio al cliente. La estructura legal es una compañía de responsabilidad limitada (LLC) y los principales competidores son los restaurantes del centro comercial Sawgrass Mall. El tamaño del mercado se estima en $30 billones.

Las tendencias actuales indican una mayor preocupación por comer alimentos saludables pero también rápidos, lo que representa un desafío para la empresa debido a la alta competencia, bajos precios y gestión adecuada del suministro. La fuerza negociadora de los proveedores es alta debido a que hay solo un proveedor para el pan utilizado para los hot dogs gourmet, mientras que la amenaza de nuevos competidores y sustitutos es alta.

Los productos ofrecidos por Venezuelan Hot Dogs son hot dogs gourmet con un sabor único y conveniencia para el cliente. Las fortalezas internas incluyen una receta única y distintiva, mientras que las debilidades internas incluyen la falta de capital para expandirse. Las oportunidades externas disponibles para la empresa son las personas que buscan sabores innovadores y la gran comunidad venezolana en el área local, mientras que las amenazas externas incluyen una recesión económica y competencia fuerte.

Para capitalizar en las fortalezas y oportunidades, se implementará una campaña publicitaria enfocada en la comunidad venezolana a través de redes sociales, campañas digitales, publicidad boca a boca y volantes distribuidos en el centro comercial Sawgrass Mall. El precio será alto debido al sector de comida rápida gourmet donde compite Venezuelan Hot Dogs. El producto se distribuirá desde un punto de venta ubicado en el centro comercial Sawgrass Mall, con empaques únicos y conveniencia del servicio rápido.

La empresa cuenta actualmente con diez empleados, pero planea contratar amigos y familiares dentro de la comunidad venezolana local para su expansión futura. Los planes futuros incluyen abrir seis tiendas más en Doral durante el próximo año, seguido por seis tiendas adicionales por año en todo Miami. Los procesos operativos actuales implican tres proveedores diferentes para las salchichas utilizadas en los hot dogs gourmet, un proveedor para el pan y la compra de salsas e ingredientes en Costco. La gestión del inventario se realiza semanalmente para analizar las ventas de la semana.

En conclusión, Venezuelan Hot Dogs tiene una oportunidad única para capitalizar en la comunidad venezolana local y ofrecer hot dogs gourmet con un sabor único y conveniencia al cliente. Con una estrategia sólida de marketing y ventas, así como planes futuros de expansión, esta empresa tiene un gran potencial para crecer en el mercado competitivo de comida rápida gourmet.