

\subsection{Yearly Profit and Loss}\label{sec:title}
\nonumsidenote{\textbf{Summary}: This report presents an analysis of the yearly profit and loss of Alpha Project. It includes a review of the revenues, product sales, cost of goods sold, gross profit, operating expenses, labor costs, rent payments, material costs, maintenance costs, IT expenses, sales and marketing expenses, lease fees and EBITDA margin. The report also provides an overview of the gross profit margin, EBITDA margin and net income margin.}

Alpha Project is a company that has been in operation for five years. During this time it has experienced steady growth in its revenues as well as other financial metrics. This report will provide an analysis of Alpha Project's yearly profit and loss data from 2015 to 2019. 

The first metric that will be analyzed is revenues. Over the five year period Alpha Project saw its revenues increase by 899%, from 601200 US$ in 2015 to 6500492 US$ in 2019. This impressive growth was driven largely by product sales which accounted for all revenue growth over this period; other services contributed nothing to revenue growth during this time frame. 

The next metric that will be analyzed is cost of goods sold (COGS). COGS saw a similar trend over the five year period increasing by 1192% from 237000 US$ in 2015 to 2800188 US$ in 2019. This increase was slightly lower than the corresponding increase in product sales indicating that Alpha Project was able to reduce its costs per unit sold over time thus increasing their overall profitability on each sale made during this period. 

The third metric that will be analyzed is gross profit which is calculated as total revenues minus COGS . Gross profit increased by 1757% from 364200 US$ in 2015 to 3700303 US$ in 2019 indicating a healthy increase due mostly to increases both on product sales as well as reductions on COGS per unit sold over time..  Operating expenses also increased significantly over this same period but not at quite the same rate with only a 567% increase compared to gross profits 1757%. Most notable among these operating expense increases were labor costs which increased by 562%, rent payments which increased by 400%, material costs which increased by 1187%, maintenance costs which also increased 1187% , IT expenses which grew 250 % ,sales and marketing expenses growing 1093 % ,and lease fees increasing 830%. These increases indicate that while Alpha Projects revenue grew significantly their operational overhead also had significant increases resulting mainly from higher salaries paid out ,rental payments made ,materials purchased etc...  

 Finally we can analyze EBITDA margin ,EBIT Margin and Net Income margins . All three margins decreased slightly over the five year period with EBITDA decreasing from 3.16 %in 2015 down to 1.85 %in 2019 while EBIT decreased similarly going down from 1.83 %in