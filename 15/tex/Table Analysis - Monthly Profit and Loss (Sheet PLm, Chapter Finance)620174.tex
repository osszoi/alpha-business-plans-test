

\subsection{Monthly Profit and Loss (m-y)}\label{sec:title}

\nonumsidenote{This section provides a summary of the monthly profit and loss for the Alpha Project. The data shows that revenues and product sales remain consistent at \$50,100 per month. Operating expenses are also relatively consistent at \$28,095 per month with labor costs accounting for most of this expense. Gross profit is \$30,350 per month while EBITDA is \$1,585 and EBIT is \$918. Net income is \$734 per month with a gross profit margin of 60.58\%, an EBITDA margin of 3.16\%, an EBIT margin of 1.83\% and a net income margin of 1.47\%.}

The table provided shows the financial performance of the Alpha Project over twelve months (m-y). The data demonstrates that revenues remain constant each month at \$50,100 while Product Sales also remain steady at this amount each month as well. Other Services do not generate any revenue in any given month, indicating that these services are not currently being offered by the company or their customers simply do not take advantage of them on a regular basis. 

The Cost of Goods Sold totals to be approximately 19,750 each month which leaves the Gross Profit to be around 30,350 each period as well. This indicates that there is likely some form of pricing structure in place for products sold by Alpha Project which allows them to maintain steady profits despite fluctuating costs associated with producing those goods or services sold to customers over time. 

Operating expenses are fairly consistent across all twelve months as well totaling 28,095 each period with Labor Costs making up most (22000)of this expense followed by Rent (2500), Material Costs (500), Maintenance Costs(500), Other Expenses such as Negative Recoveries (125) IT Costs(300), Sales & Marketing Expenses(1500) and Lease Fees(670).  These figures indicate that Alpha Project has been able to keep operating expenses under control despite increasing costs associated with providing goods or services over time due in part to their pricing structure mentioned earlier along with other cost saving measures they have implemented during operation periods throughout the year such as negotiating better terms on leases or outsourcing certain aspects like IT related tasks to third party vendors who can provide better rates than hiring full-time staff members internally would cost them in terms salary/benefits packages etc..  

 After subtracting Operating Expenses from Gross Profits we arrive at Earnings before Interest Tax Depreciation & Amortization (EBITDA) which totals 1,585 each period indicating that after covering all operational costs associated with producing goods/services sold there is still some additional capital left over before paying any interest payments owed on outstanding debts incurred during operations if applicable or allowing for depreciation/amortization expenses related to