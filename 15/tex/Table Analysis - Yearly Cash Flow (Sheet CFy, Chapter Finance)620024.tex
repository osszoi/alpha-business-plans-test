

\subsection{Yearly Cash Flow}\label{sec:title}
\nonumsidenote{This section provides an analysis of the yearly cash flow of Alpha Project. The data shows that Alpha Project has had a net income increase from 8,814 in year 1 to 1,129,810 in year 5. This is coupled with operating cash flow increases from 55,969 in year 1 to 1,165,852 in year 5. Furthermore, the cash balance has increased from 55,969 in year 1 to 2,198,403 in year 5 due to investments and bank loans. } 

The table data provided shows the yearly cash flow for Alpha Project over five years (1-5). In terms of net income (+), there is a significant increase from 8,814 (year 1) to 43,379 (year 2) and then again from 251 871 (year 3) to 635 409 (year 4). This trend continues until it reaches its peak at 1 129 810 (year 5). These results demonstrate that Alpha Project has seen steady growth since its inception and is continuing on this trajectory. 

In addition to net income (+), there are also other factors which contribute towards operating cash flow such as depreciation and amortization (+) and working capital (-). Depreciation and amortization have remained relatively constant throughout the five years at 8000 each while working capital has decreased steadily over time with values ranging from -39155 (year 1) to -28042 (year 5). This indicates that Alpha Project�s working capital is decreasing over time which can be attributed to increased efficiency or cost cutting measures implemented by the company. 

Finally we see that investments (+) made by shareholders have been instrumental in increasing the overall cash balance for Alpha Projects over time. In particular we see that investments were made only during years one and two with values of 150000 and 25000 respectively before dropping off completely for subsequent years. This could indicate a lack of confidence on behalf of shareholders or simply no need for additional funds as operations become more efficient or profits increase significantly enough due to other factors such as sales or marketing initiatives being successful.  
 
Overall it appears that Alpha Projects� financial performance has improved significantly since its inception due primarily to increases in net income coupled with investments made by shareholders during the first two years of operation. As mentioned previously these improvements can be attributed both internal efficiencies as well external factors such as successful sales/marketing campaigns leading directly into higher revenues/profits for the company overall. Going forward it will be interesting to observe if these trends continue or if new initiatives need be taken so ensure continued success going into future years beyond five which have been analyzed here today