

\subsection{Yearly Cash Flow}\label{sec:title}
\nonumsubsection{Summary} This section provides an analysis of the yearly cash flow data for Alpha Project. The table shows that, despite a decrease in net income from year 1 to year 2, operating cash flow increased substantially due to a reduction in wages and salaries. Capital expenditures decreased significantly in years 4 and 5, allowing for significant net cash inflows. Furthermore, investments generated additional cash inflows which allowed for a positive balance at the end of each year. 

The data provided indicates that Alpha Project had positive net income for each of the five years under consideration. In particular, net income rose significantly from year 1 to year 2 before decreasing slightly in years 3 and 4 before increasing again in year 5. This increase was largely due to increases in other operating expenses such as advertising costs and research and development expenses as well as higher sales revenues. 

Operating cash flow also increased over the five-year period under consideration due to a decrease in wages and salaries paid out by Alpha Project during this time frame. Specifically, wages and salaries decreased from $39,155 in Year 1 to $28,042 in Year 5 resulting in an overall reduction of 28\% over the course of these five years. This cost savings was partially offset by increases in depreciation and amortization (D&A) expenses which remained constant at $8,000 per year throughout this period; however, it still resulted in an overall increase of 54\% with respect to operating cash flow when compared with Year 1 figures ($55,969). 

Capital expenditures (CAPEX) also decreased significantly over the course of these five years resulting both from decreases made by Alpha Project itself as well as reductions made by its shareholders (fshare). Specifically CAPEX fell from $150k per annum during Years 1-3 down to zero during Years 4-5 resulting an overall reduction of 100\%. As such there were no capital outlays required during these two final years thus allowing for greater levels of net cash inflows than would have otherwise been possible had CAPEX remained unchanged throughout this period. 

Finally investments made by Alpha Project�s shareholders (fbank) also contributed positively towards its yearly balance sheets through providing additional sources of revenue which could then be used towards furthering its operations or paying out dividends if desired. During Years 1-3 investments amounted to $150k per annum before falling off completely during Years 4-5; however even with this decrease total investment income still amounted to $175k over the course of these five years representing an additional source of revenue above what would have been available without them being present at all. 

Overall it can be seen that Alpha Project has managed its finances effectively resulting both large increases with respect to operating profits as well as significant reductions when it comes capital expenditure requirements thus allowing for greater levels net cash inflow than would have otherwise