

\subsection{Flujo de Efectivo Anual}\label{sec:title}

\nonumsidenote{Resumen: En este capítulo se analizan los datos del flujo de efectivo anual para el Proyecto Alpha. Se muestra que el ingreso neto aumenta significativamente al pasar del primer año al quinto, y que los pagos por capital y gastos de funcionamiento son necesarios para mantener un balance positivo. Además, se observa una inversión significativa en el segundo año, lo que ayuda a compensar la reducción de los ingresos netos en ese momento.} 

El Proyecto Alpha ha estado operando durante cinco años, y su flujo de efectivo anual se muestra en la tabla anterior. El primer elemento destacable es el crecimiento del ingreso neto entre el primer y último año. En particular, ha pasado de 8.814 dólares al principio hasta 1.129.810 dólares al final del período analizado. Esta tendencia indica un fuerte crecimiento para la empresa durante este tiempo y refleja las buenas decisiones tomadas por la administración para expandir sus operaciones comerciales con éxito. 

Los gastos relacionados con depreciación y amortización (D&A) permanecen constantes durante todo el periodo evaluado en 8 mil dólares cada mes o año; sin embargo, los pagos por salarios o sueldos (WC) han disminuido gradualmente desde 39155 dólares hasta 28042 dólares entre los primeros cinco años; esta tendencia puede ser atribuida principalmente al hecho de que la empresa ha logrado mejorar su productividad mediante inversiones adicionales en tecnología o capacitación profesional para sus empleados actuales, lo que le permite reducir costos laborales sin comprometer su nivel general de servicio . 

Además, podemos ver claramente que hay dos componentes principales para mantener un balance positivo: gastar dinero en capital (CAPEX) y obtener flujos financieros positivos (CF). Los pagament