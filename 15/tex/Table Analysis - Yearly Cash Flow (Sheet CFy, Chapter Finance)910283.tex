

\subsection{Yearly Cash Flow} \label{sec:title}
\nonumsubsection{Summary} This chapter presents an analysis of the yearly cash flow for Alpha Project. The table data shows a net income of 8,814 in year 1, 43,379 in year 2, 251,871 in year 3, 635,409 in year 4 and 1,129,810 in year 5. Depreciation and Amortization (D&A) was 8000 each year while Wages and Compensation (-) was -39155 in year 1,-34666 in year 2 ,-68606 in year 3 ,-43651 in year 4 and -28042 in Year 5. Operating Cash Flow (OCF) was 55,969 for Year 1 increasing to 86,045 for Year 2 then to 328477 for Year 3 followed by 687 060 for Year 4 and finally 1165 852 for Year 5. Capital Expenditure from Bank Payment/Loan (CAPEX - fbank) was 0 each years while CAPEX from Shareholders Investment (CAPEX - fshare) was 150000 for Years 1 &2 but 0 subsequently. Net Cash Flow (-94 031 )in Year1 decreased to (-63 955 )in Years 2 with a subsequent increase to 328477forYear3 which kept increasing until it reached 1165 852 at the end of Year5.Cash Balance increased from 55 969 at the beginning of the period to 17 014 at the endofYear2 followed by 345 491 at the end ofYear3 before reaching 1032 551at theendofYear4and finally 2198 403at theendofYear5 after taking into accountCFfromInvestment(+150000inYears1&2).

The analysis reveals that Alpha Project has had consistent growth over a five-year period with increases both net income as well as operating cash flows throughout this period. This is indicative of an overall positive trend which bodes well for future prospects if similar performance can be sustained going forward . 

Depreciation and Amortization expenses have been static over this time frame indicating that there have not been any significant investments made over this time frame or alternatively these investments have been funded through other sources such as shareholder investments or loans taken out from banks or other financial institutions . 

Wages and Compensation expenses have also seen a decrease over this time frame indicating that there may have been some cost cutting measures implemented or alternatively that Alpha Project has become more efficient resulting in fewer employees being required to generate similar levels of output . 

Capital Expenditure from Banks Payments/Loans has remained zero throughout this entire period suggesting that either no major capital purchases were made during this timeframe or they were funded through other sources such as shareholders� investments instead . Additionally it could also indicate that Alpha Project did not need any external financing during this time frame which is indicative of strong financial health . 

