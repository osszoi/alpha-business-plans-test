

Los datos financieros indican que la compañía tiene un sólido equilibrio entre activos y pasivos. Los ingresos netos han aumentado constantemente durante los últimos 12 meses, lo que sugiere una buena salud financiera. Las existencias también muestran un nivel estable, lo que reduce el riesgo de pérdidas por inventario. El flujo de caja es relativamente consistente con los ingresos generados por las ventas. Sin embargo, se recomienda vigilar los pagos comerciales y otros pasivos para mantener el equilibrio en la situación financiera de la compañía.