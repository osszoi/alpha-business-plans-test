

\section{Empresa Alpha Project} \label{sec:alpha-project}
\nonumsidenote{Alpha Project es una empresa de comida rápida con sede en Caracas, Venezuela, que ofrece hot dogs gourmet. El objetivo a corto plazo es vender 4 millones y tener 5 sucursales. La estructura legal es una compañía de responsabilidad limitada y lleva operando durante 4 años. El equipo de gestión está compuesto por Lucas (Presidente), Raul (Vicepresidente) y Andres (Gerente de Operaciones). La empresa cuenta actualmente con 5 empleados y un volumen anual de ventas estimado en 30 mil millones. Su principal competencia son los restaurantes del centro comercial Sograss Mall. Las principales fuentes de presión competitiva son el precio bajo, la diferenciación del producto y la publicidad. El poder negociador de los proveedores es alto ya que solo hay un suministrador para el pan, mientras que el poder negociador de los compradores no existe debido a la falta de alternativas. Los retos principales del sector son la alta competencia, los precios bajos y la cadena de suministro. 

Alpha Project ofrece dos tipos distintosde hot dog: Hot Dog 1yHot Dog 2 . Están diseñados para satisfacer las necesidades del mercado local venezolano gracias a su sabor único e innovador así como por su conveniencia para consumir rápidamente en cualquier momento. Establecerse en el mercado local le da ventaja sobre sus competidores ya que se dirige directamente al segmento local venezolano con un producto adaptado a sus gustos culinarios particulares.. 

El mercado meta se centra en la comunidad venezolana ubicada en Doral, Florida. Se planean utilizar varias herramientas para llegar al mercado meta tales como campañas publicitarias digitales, redes sociales, boca a boca e impresionantes dentro del centro comercial Sawgrass Mall . La estrategia de precios será alta para el sector fast food pero se compensará con buen servicio al cliente y paquetes atractivos . La distribución se real