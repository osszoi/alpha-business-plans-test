

\subsection{Yearly Profit and Loss}\label{sec:title}
\nonumsidenote{This section provides an analysis of the yearly profit and loss of Alpha Project. The data from the table shows that revenues, product sales, gross profit, EBITDA, EBIT, net income and other metrics have been increasing over the past five years. The gross profit margin has remained relatively steady at around 60%, while the EBITDA margin has increased from 3.16% to 21.85%. The EBIT margin has also increased from 1.83% to 21.73%. Finally, the net income margin has risen from 1.47% to 17.38%. These figures indicate that Alpha Project is making good progress in terms of profitability.}

The data provided in this table reveals that Alpha Project's revenues and product sales have steadily grown over the past five years; both increasing by more than 500 thousand US dollars between 2017 and 2021 (mm US$ 601 200 - 6 500 492). This indicates a healthy rate of growth for Alpha Project's business operations which is likely to continue into 2022 and beyond if current trends persist. 

In addition to revenue growth, Alpha Project's gross profit margins have remained relatively constant at around 60%, suggesting that production costs are being managed effectively by management teams across all areas of operation within the company. This is further evidenced by a decrease in cost of goods sold (COGS) over time; COGS decreased by more than 2 million US dollars between 2017-2021 (US$ 237 000 - 2 800 188). 

Alpha project's operating expenses have also seen a decrease year on year; decreasing from 337 143 mm US$ in 2017 to 2 279 333 mm US$ in 2021 as a result of reductions in labor costs (US$ 264 000 - 1 746 750), rent (US$ 30 000 - 155 000), material costs (US$ 6 000 - 70 891) maintenance costs (US$ 6 000 - 70 891) as well as other expenses such as IT services (US$ 3 600 - 4 700) and sales/marketing activities (US$ 18000-212673). This reduction in operating expenses can be attributed largely to increased efficiency measures implemented throughout all areas of operation within Alpha project over recent years which have allowed for significant cost savings throughout its entire supply chain operations  resulting in improved margins across all areas of operation within the company .  

These cost savings are further evidenced by increases in both EBITDA margins which rose from 3.16% to 21.85%; indicating an increase efficiency with regards to how resources are being allocated throughout Alpha Projects operations ,as well as increases in its EBIT margins which rose from 1.83% to 21 73%; suggesting effective management practices when it comes to managing fixed costs associated with running day-to-day operations within alpha projects business environment . Finally ,alpha projects net income margin has risen significantly since 2017 ,