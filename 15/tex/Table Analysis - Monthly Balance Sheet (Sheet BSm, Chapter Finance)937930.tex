

\subsection{Monthly Balance Sheet}\label{sec:title}
\nonumsidenote{Summary: This section provides an analysis of the monthly balance sheet from Alpha Project. It highlights the total assets, current assets, fixed assets and liabilities, as well as the total equity and earnings for each month. Additionally, it shows a comparison between the total assets and liabilities.}
The monthly balance sheet from Alpha Project provides a comprehensive overview of its financial performance over twelve months. The table contains information on the company's total assets, current assets, cash, accounts receivable (A/R), inventories, fixed assets and liabilities. Additionally, it displays details regarding its total liability plus equity, current liability trade payables, other payables and provisions. Finally, long-term debt is also included in this table. 

The data reveals that Alpha Project's total asset value has increased steadily over the course of 12 months starting at $155 589 in month 0 to $163 668 by month 12. In particular there was an increase of $8 079 or 5\% between months 11 and 12 alone which could indicate strong growth potential for this company over time. Concurrently with these increases in asset value there have been corresponding increases in both current asset values as well as fixed asset values across all 12 months which suggests that Alpha Project has been able to invest capital into new projects or expand existing operations successfully during this period of time. 

Furthermore when we look at liabilities we can see that there has been a consistent amount reported each month across all categories including total liability plus equity ($155 589 - $163 668), current liability ($4 855) trade payables ($812) other payables ($3 925) and provisions ($118). This implies that while Alpha Projects' overall asset value is increasing their liabilities are being managed effectively so as not to exceed their capacity for repayment or incur additional debt beyond what they are capable of managing. 

Finally when looking at shareholder equity we can see that it has increased significantly over all twelve months beginning at $150 000 in month 0 up to $158 814 by month 12 indicating strong profitability for shareholders during this period of time despite any potential challenges faced by management such as rising costs associated with new projects or expansions etc.. All together these figures suggest that Alpha Projects is doing well financially despite any external pressures they may be facing due to market conditions or other factors outside their control.