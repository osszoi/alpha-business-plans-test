

Los datos financieros indican que la empresa genera ingresos constantes y sostenibles cada mes. Los costos de ventas representan el 39,42% de los ingresos, lo que nos indica un margen bruto saludable del 60,58%. Las principales áreas de gasto son el costo laboral (44%), alquileres (5%) y materiales (1%). La empresa presenta un EBITDA margin del 3,16%, un EBIT margin del 1,83% y un net income margin del 1,47%. En general, los resultados financieros son buenos y se recomienda mantener estas tendencias para mejorar la rentabilidad.