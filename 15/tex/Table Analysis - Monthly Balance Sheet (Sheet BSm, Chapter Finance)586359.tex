

\subsection{Monthly Balance Sheet}\label{sec:title}
\nonumsidenote{Summary: This paper analyses the monthly balance sheet of Alpha Project, which shows an increase in assets and equity from month to month. The current assets are mainly composed of cash and inventories, while the fixed assets consist of long-term debt. The total liabilities plus equity is equal to the total assets, indicating that Alpha Project is in a healthy financial position.}

The monthly balance sheet of Alpha Project provides valuable insights into its financial performance over time. From the table data, it can be seen that there was an overall increase in both assets and liabilities from month 0 to month 12. In particular, total assets increased from $155,589 in month 0 to $163,668 in month 12. Similarly, total liabilities plus equity rose from $155,589 in month 0 to $163,668 by the end of the period. 

Looking at the composition of Alpha Project's current assets over this period reveals that cash and inventories were the main sources for these funds. Cash increased steadily from $1,647 at beginning of this period to $17,060 by its end; meanwhile inventories remained relatively constant at around 4608 throughout this period. 

Fixed Assets also saw an increase during this same time frame with long-term debt making up most of these funds (from 146000 in Month 0 up to 142000 by Month 12). Total Equity also increased steadily during this same time frame from 150734 at beginning up 158 814 by Month 12 due primarily to earnings increasing over time (from 734 at beginning up 8814 by Month 12). Lastly it can be observed that Current Liabilities were relatively stable throughout this period with Trade Payables and Other Payables making up most of these funds (at 812 and 3925 respectively). 

Overall it appears that Alpha project has been able maintain a healthy financial position as evidenced by their steady increases in both Assets and Liabilities over time as well as their ability to keep Current Liabilities relatively stable despite increases elsewhere on their balance sheet. With such positive trends expected to continue going forward they should remain well positioned for future success provided they are able maintain such fiscal discipline going forward.