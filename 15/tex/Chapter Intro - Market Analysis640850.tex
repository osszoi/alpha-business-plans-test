.

Venezuelan Hot Dogs is a business created by Lucas, with the purpose of selling gourmet hot dogs. It has been in operation for four years and its key members are Lucas as President, Raul as Vice President, and Andres as Operations Manager. Its products consist of two types of hot dogs that meet customer needs through taste ,fastness ,and convenience . The target market is people living in Doral and the competitive advantage lies in quality and service. The legal structure of Venezuelan Hot Dogs is a Limited Liability Company (LLC). Currently it has five employees and was founded four years ago with a small point of hotdogs in New York City. 

The main competitors for Venezuelan Hot Dogs are all food restaurants in Sograss Mall. The size of the market is 30 billion dollars, with high demand due to malls being full every day. Trends within the industry include people eating healthier food and fast food options becoming more common. Challenges facing this industry include high competition, low prices, supply chain issues, price competition from product differentiation or marketing strategies employed by competitors and high bargaining power from suppliers due to limited sources available for certain supplies such as breads used for hot dogs buns; buyers do not have any bargaining power over prices due to lack thereof in fast food industries but there is always a threat of new entrants into this field which can lead to increased competition among existing players on both price points as well as product offerings or services provided by new entrants into this industry space that might be more attractive than what existing companies offer currently . Additionally there may also be threat of substitutes arising from healthier options being made available that could potentially take away customers from traditional fast food outlets like those offering gourmet hot dog products such as those offered by Venezuelan Hot Dogs . 

Venezuelan Hot Dogs sets itself apart from other competitors through its unique recipe adapted for local communities (Venezuelan) ,taste ,and convenience . Its internal strengths lie within its unique recipes ,strong brand recognition amongst local communities ,and skilled workforce while its weaknesses include limited resources ,weak brand recognition outside local communities ,and lack of differentiation between products or services compared to those offered by competitors . External opportunities exist through potential partnerships with other businesses catering towards same target markets or new markets altogether ; technological advancements that can help streamline operations ;or even access to capital investments which can help fuel growth if necessary . Threats come mainly from external sources such as competition arising out of regulatory changes or economic downturns leading customers away from traditional fast foods towards cheaper alternatives . 

The short-term goal for Venezuelan Hot Dogs is to sell 4 million dollars worth goods annually and have five branches while long-term goals involve expanding into other regions. To accomplish these objectives management team should focus on taking advantage of their strengths while addressing weaknesses & threats faced by Venezuelan Hot Dog�s business model so they can capitalize on external opportunities presented in order achieve their desired goals & objectives successfully