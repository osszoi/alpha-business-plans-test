

\subsection{Yearly Profit and Loss}\label{sec:title}
\nonumsidenote{This section looks at the yearly profit and loss of Alpha Project by analyzing their total assets, current assets, cash, accounts receivable (A/R), inventories, fixed assets, liabilities, current liabilities, trade payables, other payables, provisions, long term debt and equity shareholders� earnings. The data shows that Alpha Project has seen an increase in total assets from 207,708 to 2.5 million over five years.}


The table data reveals that Alpha Project had a total asset value of 207.7 thousand in year 0 increasing to 2.6 million by year 5. This indicates that the company is growing steadily as they are able to acquire more resources over time. 
Current Assets also increased from 65 thousand to 2.3 million indicating that the company is making good use of its liquidity by investing it into short-term investments or operations which generate returns quickly such as inventory purchases or accounts receivables (A/R). Cash also increased significantly from 56 thousand to 2.2 million showing an efficient management of finances for day-to-day operations. 
Inventories rose from 9 thousand to 115 thousand suggesting an increase in production output and sales volume over the years while Fixed Assets remained relatively constant between 142 thousand and 284 thousand indicating a stable capital base for long-term investments or projects with minimal depreciation costs being incurred each year. 
Liabilities also saw a steady rise from zero in year 0 up to 1.4 million in year 5 with Current Liabilities increasing from 49 thousand to 329 thousand due mainly to Trade Payables (379) and Other Payables (47 - 319). Provisions also climbed slowly but steadily throughout this period reaching 9578 by year 5 while Long Term Debt was not present at any point during these five years suggesting no external borrowings were needed for funding purposes during this time frame. 
Finally Equity Shareholders� Earnings grew exponentially starting off at 8800 before rocketing up all the way up past two million in year 5 which suggests a healthy return on investment for shareholders who have invested in the company�s success thus far as well as providing confidence that future returns could be even higher if further growth continues along this trajectory going forward into 2021 and beyond! 

In conclusion we can see that Alpha Projects overall financial health has been improving steadily since its inception with Total Assets rising significantly alongside Current Assets & Cash reserves while Inventories & Fixed Assets remain fairly constant throughout this period indicating stability within those areas too! Liabilities have grown gradually but are still well managed considering there was no Long Term Debt present at any stage during these five years whilst Equity Shareholders� Earnings have skyrocketed due likely thanks to prudent management decisions leading up until now so let's hope these positive trends continue into 2021!