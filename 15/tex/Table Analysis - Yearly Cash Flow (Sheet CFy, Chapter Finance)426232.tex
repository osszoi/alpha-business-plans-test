

\subsection{Yearly Cash Flow}\label{sec:cashflow}
\nonumsidenote{This section provides an analysis of the yearly cash flow of Alpha Project, which includes net income, depreciation and amortization (D\&A), working capital (WC), operating cash flow (OCF), capital expenditures from bank and shareholders (CAPEX-fbank/fshare) and cash balance.}

The table above shows a five-year summary of the yearly cash flow for Alpha Project. In year 0, there was no net income as the company had just started operations. However, in year 1, Alpha Project recorded a net income of \$8,814. This figure increased to \$43,379 in year 2 and further rose to \$251,871 in year 3. By year 4, the net income had grown to \$635,409 while it reached its peak at \$1,129,810 in year 5. 

Depreciation and amortization expenses remained constant at \$8000 over all five years while working capital decreased by approximately \$39155 in year 1 before decreasing further to reach negative figures by years 4 and 5 (-\$43651 and -\$28042 respectively). As a result of these changes in income and expenses over time along with CAPEX-fbank/fshare remaining zero throughout the period under review; operating cash flow increased steadily from \$55969 in year 1 to reach its peak at \$1165852 by year 5. 

In addition to this increase in OCF over time; Alpha Project also received an inflow of funds via investments amounting to a total of \$175000 across two years (years 1 & 2). This combined with CAPEX-fbank/fshare being zero resulted in a positive net cash figure for each consecutive financial period starting from -\943101 during Year 1 up until +2198403 by Year 5. Consequently; this enabled Alpha Projects' overall cash balance position to improve significantly from +55969 during Year 1 up until +2198403 during Year 5 indicating that the company was able to generate sufficient funds internally as well as attract external investors who could provide additional resources for expansion purposes. 

In conclusion; Alpha Projects' strong financial performance has enabled it to become one of the leading players within its industry due its ability to generate steady returns on investments while making efficient use of available resources such as working capital which have allowed it maintain positive levels of liquidity throughout all five financial periods under review herewith. 

Summary: This section provides an analysis on Alpha Projects' yearly cash flow which includes their net income growth over time along with their consistent D&A expenses plus decreasing working capital resulting into increasing operating cash flows combined with positive inflows from investments enabling them to achieve high levels or liquidity thus becoming one fo the leading players within their industry.