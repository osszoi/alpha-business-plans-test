\section{Estrategia de Marketing y Ventas} \label{sec:estrategiademarketingyventas}
Los Perros Calientes Venezolanos, una empresa fundada por Lucas con el propósito de vender perros calientes gourmet, se encuentra en la Calle La Colina Edificio 32 Av Fuerzas Armadas Caracas. La compañía ha estado operando durante 4 años y actualmente está compuesta por 5 empleados. Fue fundada cuando Lucas tenía un pequeño puesto de perros calientes en la ciudad de Nueva York que fue muy exitoso.
Los principales competidores son todos los restaurantes de comida en el Sawgrass Mall, así como las grandes cadenas como McDonalds y Wendys. El tamaño del mercado se estima en 30 mil millones de dólares estadounidenses y hay una gran demanda por los productos de Venezuelan Hot Dog debido a su calidad y oferta de servicios. Las principales fuentes de presión competitiva son los precios más bajos, mientras que los proveedores tienen un alto poder de negociación porque solo hay un proveedor de pan. No hay poder de negociación por parte de los compradores, pero sí una alta amenaza de nuevos entrantes que ingresan al mercado todos los días, así como sustitutos como opciones saludables.
Venezuelan Hot Dogs ofrece dos tipos de perros calientes: Hot Dog 1 y Hot Dog 2. Satisfacen las necesidades del cliente a través del sabor, la velocidad y la conveniencia, al mismo tiempo que ofrecen adaptaciones locales para atraer a la comunidad local venezolana. Las fortalezas internas incluyen recetas únicas, mientras que las debilidades incluyen falta de capital disponible para fines de expansión. Las oportunidades externas incluyen personas que buscan nuevos sabores innovadores, mientras que las amenazas externas incluyen una recesión económica y altos niveles de competencia dentro de la industria. Las estrategias implementadas se centrarán en aprovechar las fortalezas, como las recetas únicas, mediante el marketing dirigido a comunidades venezolanas a través de campañas en redes sociales o ventas directas mediante volantes en centros comerciales, etc., con precios establecidos a un nivel más alto que otros restaurantes de comida rápida debido a sus ofertas de calidad en comparación con sus competidores'.
El mercado objetivo para los perros calientes venezolanos consiste principalmente en venezolanos que viven en el área de Doral, a los cuales se puede llegar a través de publicidad en plataformas de redes sociales o ventas directas mediante volantes distribuidos en centros comerciales, etc. La distribución se llevará a cabo en su propia tienda ubicada en Sawgrass Mall, donde los clientes pueden comprar sus productos teniendo en cuenta factores de conveniencia como el diseño del empaque o tiempos de servicio más rápidos que ofrecen otros restaurantes de comida rápida. Las estrategias de diferenciación se centran en atender a la comunidad local de Venezuela ofreciendo recetas únicas que les gustan más que otras opciones y no ofrecen sabores similares ni ingredientes utilizados en sus platos.
Los recursos humanos actualmente consisten en 10 empleados, pero existen planes para contratar amigos/familiares de la comunidad local de Venezuela, mientras que las políticas de recursos humanos giran en torno a los beneficios legalmente exigidos/tiempo libre/evaluaciones de desempeño, etc. Las operaciones involucran instalaciones físicas que consisten en un pequeño espacio utilizado para preparar alimentos ubicado en el área de Doral, con planes existentes para expandir estas instalaciones hasta 6 tiendas más dentro del área de Miami durante el próximo año, seguido por otras 6 por año si el negocio demuestra ser lo suficientemente exitoso. Los procesos operativos incluyen 3 proveedores que proporcionan ingredientes para salchichas junto con 1 proveedor que proporciona productos de pan y salsas/otros ingredientes comprados en tiendas Costco. La gestión del inventario/gestión de la cadena de suministro se realiza semanalmente basándose en datos de ventas recopilados durante períodos anteriores.
Finalmente existen programas de desarrollo y capacitación para empleados enfocados en hacer cumplir los estándares establecidos por la empresa, asegurando que todo el personal cumpla con las mismas reglas al interactuar directamente con los clientes, ya sea en persona o a través de medios en línea como las plataformas de redes sociales utilizadas ampliamente en las comunidades venezolanas cercanas al área de Doral.
