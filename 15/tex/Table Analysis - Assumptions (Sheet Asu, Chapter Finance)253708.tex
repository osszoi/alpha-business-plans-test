

\subsection{Assumptions}\label{sec:assumptions}
The Alpha Project has made several assumptions to analyze the potential revenue and net income after taxes (NIAT) for a five-year period. The growth rate is assumed to be 80\%, inflation is assumed to be 3\%, packing and shipping costs are assumed to be 1.00\%, material costs are assumed to be 1.00\%, maintenance costs are assumed to be 0.50\%, other negative recoveries are assumed to be 0.25\% and sales and marketing costs are estimated at 3.00%. Additionally, it is estimated that taxes will take up 20\% of total revenue. 

Based on these assumptions, the total revenue over the five-year period is projected at \$13,834,026 while the maximum revenue in any one year is projected at \$6,500,492. The NIAT over the five-year period is projected at \$2,069,283 with a maximum NIAT in any one year of \$1,129,810. The minimum investment required for this project is estimated at \$148,353 while its net present value (NPV) using a 10 percent discount rate was calculated as \$2,543,467 with an internal rate of return (IRR) of 186 percent. 

In order for Alpha Project's business plan to succeed it must have adequate funding and resources available in order for its projections to become reality. In addition to financial investments there must also be human capital invested into research and development as well as marketing efforts in order for Alpha Project's products or services to gain traction within their target markets or industries they wish penetrate or expand into new ones altogether . Furthermore , if there are any changes in market conditions such as increased competition , pricing wars , regulatory changes , etc., then adjustments may need to made accordingly . 

Overall , Alpha Projects' assumptions provide insight into what can reasonably expected from their business plan should all factors remain constant throughout the allotted time frame . This information can used by investors when determining whether or not they should invest into Alpha Projects' venture . It also provides guidance for management on how best allocate resources throughout each stage development so that goals set forth can achieved efficiently . 

Summary: The Alpha Project has made several assumptions about growth rates , inflation rates , packing/shipping costs , material costs , maintenance costs , negative recoveries & sales/marketing expenses which were used calculate estimates of total revenues & net income after taxes (NIAT). These estimates suggest that sufficient funding & resources must available in order achieve stated projections & that adjustments may need make should market conditions change unexpectedly during course development .