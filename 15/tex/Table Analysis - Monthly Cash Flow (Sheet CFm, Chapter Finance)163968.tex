

\subsection{Monthly Cash Flow Analysis}\label{sec:title}
\nonumsubsection*{Summary} This section provides an analysis of the monthly cash flow for the Alpha Project. The table data indicates that net income is constant throughout the twelve months, while depreciation and amortization remain unchanged. Working capital requirements are zero for all periods. Operating cash flow is consistent over the year, with no capital expenditure from either shareholders or banks. As a result, net cash is negative in month one but increases to positive values in subsequent months due to investment inflow of $150,000. This leads to a steady increase in cash balance throughout the period. 

The Alpha Project's financial performance can be examined through its monthly cash flow statement. Net income remains constant across all twelve months at $734 million US dollars (mm US$). Depreciation and amortization also remain consistent at $667 mm US$. There are no working capital requirements during this period as indicated by a value of 0 for WC (-). 

Operating Cash Flow (OCF) is calculated using net income plus D&A minus WC which yields an amount of $1,647 mm US$ for each month. Since there are no capital expenditures from either shareholders or banks, CF from Bank Payment/Loan (=) remains at 0 throughout the year and Net Cash increases steadily as an investment inflow of $150,000 occurs each month beginning in month two. 

The final column shows that Cash Balance begins at 1,647 mm US$ in month one and steadily increases each month due to both OCF and investment inflows reaching 17,060 mm US$ by the end of the twelve-month period. This demonstrates that despite initial negative net cash position caused by large investments made during month one; Alpha Project was able to maintain positive cash balances throughout its financial year due to steady operating performance combined with additional investments made during later months.  

Overall it can be seen that Alpha Project has maintained strong financial performance over its twelve-month period thanks to consistent operating results combined with additional investments made during later months resulting in increasing levels of available funds within its accounts over time.