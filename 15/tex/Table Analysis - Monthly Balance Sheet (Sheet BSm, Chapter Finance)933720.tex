

\subsection{Monthly Balance Sheet}\label{sec:title}
\nonumsidenote{Summary: This section provides an analysis of the monthly balance sheet for the Alpha Project. It examines the total assets, current assets, cash, accounts receivable, inventories, fixed assets, liabilities, total liability and equity, current liabilities, trade payables, other payables and provisions. Additionally it looks at long-term debt and total equity as well as earnings and equity shareholders.}

The monthly balance sheet for the Alpha Project is a useful tool to gain insight into its financial health. The table below shows the total assets of the company from month 0 to 12. Total Assets increased steadily over this period from 155 589 in month 0 to 163 668 in month 12. This indicates that there was a steady increase in profits over this period which is encouraging news for potential investors or creditors. 

Current Assets also increased steadily from 6 256 in month 0 to 21 668 in month 12 indicating that more money was coming into the business than going out during this period which is positive for cash flow management. Cash rose significantly from 1 647 in month 0 to 17 060 in month 12 showing that there were sufficient funds available for day-to-day operations of the business. Accounts Receivable stayed constant at zero throughout this period indicating no customers had outstanding debts owed to them by Alpha Project during this time frame which could be due to efficient collection policies or lack of sales activity depending on how one interprets it. Inventories remained constant at 4 608 throughout this period suggesting either slow inventory turnover or strong demand for products being sold by Alpha Project during this time frame since they were able to keep their inventories level despite increasing sales volumes over time. 

Fixed Assets also increased steadily from 149 333 in month 0 up until 142 000 in Month 12 indicating an increasing investment into physical capital such as land or buildings by Alpha Project over time resulting from profits generated by their operations which can be viewed as a sign of growth and strength within their industry sector overall. 

Total Liability + Equity also rose steadily over this same period from 155 589 up until 163 668 while Current Liabilities remained constant at 4 855 throughout all months suggesting that more money was coming into the business than going out during these periods due to efficient management practices within Alpha Projects’ finance department which is a good sign when considering potential investments or loans made towards them moving forward since they have proven themselves capable of managing their finances responsibly thus far with consistent results each reporting cycle on top of growing profits year after year so far according to these figures provided here today. Trade Payables and Other Payables both stayed consistently low while Provisions remained constant at 118 throughout all months further reinforcing our previous conclusions regarding efficient financial management practices adopted by Alpha Projects’ finance department currently employed here today who are doing an excellent job keeping costs down while still maintaining healthy levels of profitability overall according Long Term Debt