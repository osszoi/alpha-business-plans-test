

\subsection{Yearly Cash Flow}\label{sec:title}
\nonumsidenote{This section provides a summary of the yearly cash flow of the Alpha Project company, including net income, depreciation and amortization (D\&A), working capital (WC), operating cash flow, capital expenditure from bank payments/loans (CAPEX-fbank) and CAPEX-fshare, net cash, cash balance and investments. The analysis reveals that the Alpha Project has been able to generate positive net income for each year over the past five years. Additionally, its operating cash flow has been increasing steadily over this period. However, due to high levels of CAPEX-fshare in years 1 and 2 coupled with low levels of investments in subsequent years there is a decrease in total cash balance.}

The table above shows the yearly data for Alpha Project's financial performance over the past five years. In terms of net income (+), it can be seen that all five years have generated positive results ranging from 8,814 to 1,129,810. This indicates that the company is doing well financially as they are generating more than enough revenue to cover their expenses each year. 

Depreciation and amortization (D&A) (+) were kept constant at 8 000 for all five years while working capital (-) ranged from 39 155 in Year 1 to 28 042 in Year 5 indicating an overall reduction in WC over time which could be attributed to better management of resources or increased efficiency within operations. Operating Cash Flow (OCF) was calculated by subtracting WC from D&A resulting in values ranging from 55 969 - 1 165 852 with an overall increase trend between Years 1 - 5 indicating improved operational performance by Alpha Project during this period. 

In terms of Capital Expenditure (CAPEX), two components were considered; fbank (-) which remained constant at zero for all five years and fshare (-) which was highest at 150 000 during Years 1 & 2 before dropping off completely thereafter suggesting that Alpha Project had no need for further investments after these two initial periods or alternatively did not have sufficient funds available to invest any further into their business operations beyond these two periods. 

Net Cash (=) was calculated by subtracting CAPEX-fbank & fshare from OCF resulting in values ranging from -94 031 � 1 165 852 again showing an overall increase trend between Years 1 � 5 but with a slight dip between Years 4 & 5 likely due to higher levels of CAPEX-fshare during those earlier periods when compared with later ones as discussed previously. Finally CF Investments (+) were highest at 150 000 during Year1 before dropping off significantly thereafter suggesting a lack of investment opportunities or insufficient funds available beyond this initial period although it should be noted that 25 000 was still invested during Year 2 although this amount dropped off completely afterwards likely due to same reasons outlined above regarding lack of opportunities/funds