\section{Supuestos}El proyecto Alpha se ha estimado con una serie de supuestos fundamentales. Estas asunciones están relacionadas con el crecimiento, la inflación, los costos de empaquetado y envío, los costos de material, los costos de mantenimiento, otros (recuperaciones negativas), gastos de ventas y marketing y tributación. El resumen de estas suposiciones se muestra en la tabla 1.
La Tabla 1 indica que el crecimiento anual promedio esperado para el proyecto Alpha es del 80\%. Esta tasa representa un nivel moderadamente alto para un proyecto comercial. La inflación esperada durante el período del proyecto es del 3\%. Los costos de empaquetado y envío se estiman en 1\% del total neto facturado mensualmente. Los costos directos por materia prima también se estiman en 1\% del total neto facturado mensualmente. Los costos generales y administrativos incluyen un 0,5\% para mantenimiento general y un 0,25\% para recuperaciones negativas. Se han considerado gastos adicionales de ventas y marketing equivalentes al 3 \% del total neto facturado mensualmente. Por último, las ganancias antes de impuestos se gravarán al 20 \%.
La tabla 1 también muestra un resumen financiero a 5 años con base en estas suposiciones básicas. Se prevé que el ingreso total acumulado durante este período sea de 13,834,026 millones de dólares estadounidenses (USD). De este total acumulado durante los 5 años, 6,500,492 millones de USD corresponderán al mejor año previsto dentro del periodo de cinco años analizados. El resultado NIAT acumulada durante este periodo de cinco años es de 2,069,283 millones de USD , con un máximo NIAT previsto para el mejor año del periodo de cinco años analizados por valor de 1,129,810 millones de USD.
Además, la Tabla 1 indica que la inversión mínima necesaria para el proyecto es de 148,353 mil USD. Finalmente, la Tabla 1 muestra el valor presente neto (NPV) al 10\% para el proyecto que es de 2,543,467.76 USD. La tasa interna de retorno en el proyecto (IRR) se ha calculado como 186\%.
\\nonumsidenote\\{summary} En este apartado hemos realizado un análisis sobre los supuestos básicos en los que se basa la estimación financiera para el proyecto Alpha; entre ellos están el crecimiento anual promedio de 80%, la inflación de re3%, los costes sobre empaque y envío de 1%, los costes sobre materia prima de 1%, los costos de desmantenimiento de 0.5%, otros (recuperaciones negativas) de 0.25% gastos en ventas y marketing o 300% y tribución de 20%. Además, se han calculado los ingresos totales acumulados durante cinco años para el Proyecto Alpha y el NIAT acumulado para este mismo período después del primer año previsto y una inversión mínima necesaria de 148353 mil USD. Finalmente, se ha calculado la NPV@10% para el Proyecto y la IRR como del 186%.
