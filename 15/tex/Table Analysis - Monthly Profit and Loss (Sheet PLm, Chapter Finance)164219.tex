\\nonumsidenote\\{summary} El capítulo de pérdidas y ganancias mensuales del Proyecto Alfa ofrece una visión general de los ingresos, gastos y margen de ganancia neta para cada mes. Los resultados muestran que el proyecto ha tenido un aumento constante en los ingresos durante todo el año, con un margen bruto promedio del 60,58%, un margen EBITDA promedio del 3,16% y un margen EBIT promedio del 1,83%. Además, el proyecto ha logrado mantener su rentabilidad neta promedio en 1.47% durante todo el año.
\section{Pérdida y Ganancia Mensual}

Por medio del presente informe, se presenta la pérdida y ganancia mensual correspondiente al período comprendido entre enero y marzo de 2021. 

En relación a los ingresos, se registró un total de $50,000 durante el mes de enero, mientras que en febrero se obtuvo una suma de $60,000. En marzo, los ingresos disminuyeron a $45,000 debido a la disminución de las ventas.

En cuanto a los gastos operativos, estos ascendieron a $25,000 durante el mes de enero. En febrero hubo un aumento en los gastos debido al pago de bonificaciones por desempeño laboral excepcional alcanzando un total de $35,000. Finalmente en marzo se redujeron los gastos operativos hasta llegar a una suma total de $30,000.

Como resultado final del trimestre analizado se obtuvo una ganancia neta acumulada por un total de $65,000.

Se espera continuar con este nivel positivo para el siguiente trimestre y seguir mejorando nuestros resultados financieros.
El Proyecto Alpha ha registrado $50,100 dólares en ingresos mensuales durante todo el año. Estos ingresos se han generado principalmente a través de la venta de productos ($50,100 dólares), ya que no se han realizado otros servicios ($0 dólares).
Los costos totales asociados con la producción son de 19,750 dólares al mes, lo que significa que hay 30,350 dólares disponibles como ganancias brutas cada mes durante todo el año.
Los gastos operativos totales suman 28.095 dólares por mes, incluyendo 22.000 dólares en salarios; 2.500 dólares en alquiler; 500 dólares en materiales; 500 dólares en mantenimiento; 125 dólares en recuperaciones negativas y 300 dólares para IT, además de 1.500 dólares para publicidad y mercadotecnia, así como 670 dólares por concepto de arrendamiento financiero mensualmente. Estas erogaciones reducen las ganancias antes del impuesto (EBT) hasta los 918 Dólares por mes, lo cual da lugar al impuesto a los $184 Dólarés por mes obteniendo una rentabilidad neta final promedio anual de $734 Dólarés o el 1,47% sobre los ingresos totales.
Establecer estadísticamente los datos anteriores nos permite observar que existe un margen bruto promedio anual del 60,58%; un margen EBITDA promedio anual del 3,16% y un margen EBIT promedio anual del 1,83%. Estas tasas son consideradas buenas dentro de la industria ya que permitirán cubrir los costos operativos sin restarle demasiado valor agregado al Proyecto Alpha.
