

\subsection{Yearly Profit and Loss}\label{sec:title}
\nonumsidenote{This section provides an analysis of Alpha Project's yearly profit and loss data. It examines the revenues, product sales, cost of goods sold, gross profit, operating expenses, labor costs, rent expenses, material costs, maintenance costs, other expenses (negative recoveries), IT expenses, sales and marketing expenses and lease fees. Additionally it looks at EBITDA margin and net income margin.}

The table above shows Alpha Project's yearly financial performance from month 0 to month 5. During this period total revenues increased from \$601200 to \$6500492 with product sales contributing the majority of this increase. Other services were not a significant factor in the revenue growth as they remained constant throughout the period at zero dollars. 

Cost of goods sold also increased over this period from \$237000 to \$2800188 which is largely due to increases in labor costs (from \$264000 to \$1746750) and material costs (from \$6000 to \$70891). Rent expenses were relatively consistent over the 6-month period fluctuating between $30000 - $150000 while maintenance costs followed a similar pattern increasing slightly from $6000 - $70891. 

Gross profit rose significantly during this time frame going from \$364200 at month 0 to $\3700303 by month 5 indicating that Alpha Projects was able to maintain its pricing structure even as production costs rose. Operating Expenses also increased but not as dramatically as gross profits going up by only around 25\% during this time frame. This suggests that Alpha Project was able to reduce overhead or increase efficiency during this 6-month period despite rising production costs. 

 The EBITDA margin for Alpha Project ranged between 3\%-22\% over these 6 months indicating that overall profitability was relatively high compared with industry standards for similar businesses. The net income margin ranged between 1\%-17\% which is lower than most industries but still indicates healthy profits for Alpha Project given their current market position and size relative to competitors in their sector. 

 In conclusion it appears that overall profitability has been increasing steadily for Alpha project since they began operations 6 months ago due largely in part of their ability manage production cost while maintaining pricing structure across all services offered by them despite rising input prices in materials and labor wages/salaries . This trend should continue if they are able to maintain operational efficiencies throughout all departments within the company while continuing aggressive marketing strategies aimed at driving customer acquisition levels up further than what we have seen so far since inception .  

 Summary: This section provides an analysis of Alpha Project's yearly profit and loss data showing increases in revenue driven mainly by product sales paired with a rise in cost of goods sold due mainly increases in labor cost and material cost resulting in steady growths of gross profits over six months along with stable operating expenses