

\section{Alpha Project Business Plan} \label{sec:alpha-project-business-plan}

\nonumsubsection{Executive Summary}\label{sec:executive-summary}
The Alpha Project is a Venezuelan Hot Dogs business that sells gourmet hot dogs. It was founded four years ago by Lucas, and is currently located at Calle la colina edificio 32 Av Fuerzas Armadas Caracas. The target market of the business are people living in the Doral, and its competitive advantage lies in its quality and services. Its short-term and long-term goals are to sell 4 million hot dogs with five branches across the city. 

\nonumsubsection{Company Description}\label{sec:company-description}
The Alpha Project is a Limited Liability Company that has been in operation for four years. The key members of its management team are Lucas as President, Raul as Vice President, and Andres as Operations Manager. Currently it employs five people and had a small point of hotdogs in New York City which was very successful. 

\nonumsubsection{Market Analysis}\label{sec:market-analysis}  
The main competitors of the Alpha Project are all food restaurants in the Sograss Mall, with an estimated size of 30 billion dollars for its market. There is high demand for their products or services due to their convenience, taste and local adaptation to Venezuelan communities. The industry faces strong competition from large chains such as McDonalds or Wendies, while suppliers have high bargaining power due to limited availability of bread supplies. There is also threat from new entrants entering the market every day as well as substitutes offering healthy options instead. 

 \nonumsubsection {Product or Service Description}\label {sec:product -or -service - description }  The Alpha Project offers two types of hot dogs; Hot Dog 1 and Hot Dog 2 which meet customer needs through taste, speediness and convenience compared to other fast food restaurants in the area . They set themselves apart from competitors by offering unique recipes tailored towards Venezuelans living locally , along with great packaging that ensures convenience . Key features include taste , convenience , local adaptation (Venezuelan community) , unique recipe , high prices (for fast food sector). Internal strengths include unique products/services , strong brand recognition , skilled workforce while weaknesses include limited resources , weak brand presence & lack of differentiation . External opportunities include new markets/partnerships/technological advancements while external threats comprise competition/regulatory changes/economic downturns . Strategies can be employed to capitalize on internal strengths /opportunities & address weaknesses/threats such as marketing campaigns targeting Venezuelan communities . 

 \nonumsubsection {Marketing \& Sales Strategy }\label { sec : marketing - sales - strategy }  The target market for the Alpha Project�s products & services consists primarily of Venezuelans living in Doral