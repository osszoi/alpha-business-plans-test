Venezuelan Hot Dogs es una empresa fundada por Lucas con el objetivo de vender perros calientes gourmet. Está ubicada en Caracas y tiene una estructura comercial de Compañía de Responsabilidad Limitada (LLC). La empresa ha estado operando durante los últimos 4 años, dirigida por Lucas (Presidente), Raúl (Vicepresidente) y Andrés (Gerente de Operaciones). En la actualidad, cuenta con 5 empleados y tiene un alto volumen anual de ingresos. El tamaño del mercado al que se dirige Venezuelan Hot Dogs es de 30 billones, con una gran demanda para este tipo de producto debido a la actual tendencia hacia la alimentación saludable y rápida. Los principales competidores de la empresa son los restaurantes que se encuentran en el centro comercial Sawgrass Mall.
Venezuelan Hot Dogs ofrece dos tipos distintos de perros calientes: Perro Caliente 1 y Perro Caliente 2. Destacan por su sabor único e innovador, su atractiva presentación y su adaptación local a Venezuela. Entre sus fortalezas internas se encuentran recetas exclusivas, una marca sólida y personal capacitado. Sin embargo, sus debilidades son la limitada escasez de recursos, una marca débil y la falta de diferenciación entre productos.
 
Por otra parte, existen oportunidades externas como nuevos mercados y asociaciones tecnológicas avanzadas; pero también hay amenazas externas como la competencia regulatoria y los cambios económicos.
La estrategia implementada para capitalizar sobre fortalezas/opciones y abordar debilidades/amenazas incluyen campañas publicitarias, marketing digital, boca a boca y distribución de volantes en el centro comercial Sawgrass Mall. Además, se centra en el mercado de la comunidad venezolana del área de Doral, con precios altos para el sector de comida rápida y un punto de venta en el corner de comida del Sawgrass Mall. También se cuenta con amigos/familiares para la expansión y fuerza laboral hasta 10 empleados a corto plazo, gestión de inventario y cadena de suministro mediante una pequeña área congelada. Las compras son semanales y se analizan las ventas semanales para prevenir problemas de stockouts y de pedidos excesivos. Se cumplen con los horarios de ley de los empleados por hora y se planifica la supervisión y desarrollo personal basado en estándares para mejorar la calidad y el servicio al objetivo de la clientela: Venezuelan Hot Dogs.
En resumen, Venezuelan Hot Dogs es un negocio exitoso dedicado a la venta gourmet de perros calientes que tiene la intención de expandirse hasta 5 sucursales pronto. Está ubicado en Caracas y cuenta con un equipo directivo conformado por 3 miembros clave: Lucas (Presidente), Raúl (Vicepresidente) y Andrés (Gerente de Operaciones). Ofrecen dos variedades distintivas -Perro Caliente 1 y 2- destacadas por su sabor único e innovador, así como por su presentación atractiva y adaptación al mercado local de Venezuela. Cuentan con fortalezas internas, tales como recetas creativas y una marca fuerte, así como debilidades tales como la escasez de recursos, cuota de mercado débil, falta de diferenciación entre productos, opciones limitadas para presentarse en nuevos mercados, tecnología avanzada, competencia regulatoria y cambios macroeconómicos. Para enfrentar estos desafíos, implementan estrategias de comunicación, publicidad, marketing digital, boca a boca, volantes distribuidos en Sawgrass Mall y el objetivo de mercado de la comunidad venezolana del área de Doral. Tienen precios altos para un fast food y su punto de venta interior es en local food corner de Sawgrass Mall. Amigo y familiares planean expandirse con hasta 10 empleados a corto plazo. Gestionan el inventario en una cadena de suministro pequeña, área congelada, compras semanales y analizan las ventas semanales para prevenir problemas de stockouts y sobrepedidos. Se trabaja en cumplimiento con horas laborales y supervisión con planes de desarrollo personal basado en estándares para mejorar la calidad del servicio ofrecido a la clientela. El objetivo de Venezuelan Hot Dogs es expandirse.
