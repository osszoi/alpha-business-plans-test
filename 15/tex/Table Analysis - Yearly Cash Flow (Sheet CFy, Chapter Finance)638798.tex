

\subsection{Yearly Cash Flow}\label{sec:title}
\nonumsidenote{This section provides a summary of the yearly cash flow analysis of Alpha Project. The analysis reveals that net income has increased steadily over five years, while operating cash flow and cash balance have grown significantly. In addition, the company has been able to maintain a positive net cash position through capital expenditure and investment activities.}

The table provided presents data on Alpha Project�s yearly cash flow from year 1 to year 5. Net income for each year is represented in row 1, with an increase from 8,814 in year 1 to 1,129,810 in year 5. This indicates steady growth for the company over time. Row 2 shows D&A expenses, which remain constant at 8000 per year throughout the five-year period. Row 3 shows WC expenses which decrease over time from 39155 in year 1 to 28042 in year 5 suggesting improved efficiency and cost control measures implemented by the company over this period. 

Row 4 shows operating cash flow which increases significantly from 55,969 in year 1 to 1,165,852 in year 5 indicating successful operations and efficient management of resources by Alpha Project during this period. Rows 5 and 6 show CAPEX - fbank and CAPEX - fshare respectively which remain equal at 0 for all years as no bank payments or loan were made or investments made via shareholding during this period.  Row 7 shows CF from Bank Payment / Loan which also remains equal at 0 for all years indicating no loan payments or bank payments were made during this period either. 

Row 8 shows net cash position which remains positive throughout all five years ranging from -94031 in Year 1 increasing steadily to 1165 852 by Year 5 indicating successful financial management within Alpha Project resulting in healthy financial performance across all five years studied here. Finally row 9 displays CF from Investment activity with 150000 being invested initially followed by 25000 being invested only once during Year 2 with no further investments made till end of Year 5 demonstrating conservative approach towards investments taken up by Alpha Project over these five years under study here.. 

 Lastly row 10 displays Cash Balance increasing significantly from 55 969 (Year1) to 2198403 (Year5) reflecting growing liquidity within the organization due its efficient financial management practices as discussed above .  

 All these factors combined together indicate that Alpha project has been able to achieve strong financial performance across a span of five years under study here through its efficient financial management practices such as judicious investment decisions , cost control measures , effective use of resources etc . This demonstrates sound business acumen exercised by Alpha project�s leadership team .