.

Venezuelan Hot Dogs is a fast food restaurant located in Caracas, Venezuela, founded by Lucas four years ago. The company offers two types of hot dogs�Hot Dog 1 and Hot Dog 2�to customers living in the Doral area. Venezuelan Hot Dogs has five employees and its competitive advantage lies in its quality and services. Its legal structure is Limited Liability Company (LLC). The main competitors of Venezuelan Hot Dogs are other restaurants in the Sawgrass Mall; their target market consists of people living in the Doral area with an estimated market size of 30 billion dollars. The current market demand for their products and services is high due to full malls, but there is also high competition and low prices that pose challenges for the industry. Sources of competitive pressure include price competition, product differentiation, and marketing strategies. Suppliers have high bargaining power while buyers have no bargaining power; meanwhile new entrants are entering the market every day as well as new healthy alternatives that threaten traditional fast foods such as hot dogs. 

Venezuelan Hot Dogs offer two types of hot dogs that meet customer needs through taste, speed, convenience, and local adaptation for Venezuelan customers who want this flavor profile. Their key features include taste, convenience, local adaptation with a unique recipe that sets them apart from competitors� offerings; internal strengths include their unique recipe while weaknesses are limited resources and lack of capitalization on external opportunities like partnerships or technological advancements. Strategies implemented by Venezuelan Hot Dogs capitalize on strengths (unique recipe) while addressing weaknesses (lack of capital) by focusing on marketing campaigns targeting Venezuelans living nearby as well as pricing strategies that set them apart from traditional fast food restaurants at a higher price point than expected from most restaurants within this sector . Distribution takes place at their physical facility located within Sawgrass Mall alongside plans for expansion into six more stores over the next year throughout Miami; inventory management involves weekly purchases based on sales analysis alongside HR policies adhering to laws regarding hourly employees with benefits such as time off or performance evaluations being offered accordingly . 

In conclusion, Venezuelan Hot Dogs is a successful business venture offering two types of hot dog products to customers living in Caracas' Doral area with an estimated market size worth 30 billion dollars annually. The business has five employees and operates under an LLC legal structure giving it limited liability protection against potential losses or liabilities incurred during operations. Its competitive advantage lies in its quality products which feature unique recipes tailored to meet customer needs through taste, speed, convenience and local adaptation for Venezuelans wanting these flavors profiles not found elsewhere within this sector or regionally speaking amongst competitors alike.. Short-term goals involve selling 4 million hot dogs while long-term goals focus on becoming a regional leader within the fast food industry via expansion into six more stores throughout Miami over the next year alongside marketing campaigns targeting Venezuelans living nearby coupled with pricing strategies setting them apart from traditional fast food establishments at higher price points than expected from most restaurants operating within this space..