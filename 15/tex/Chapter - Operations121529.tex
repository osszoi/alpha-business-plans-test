\section{Operaciones} \label{sec:operaciones}Venezuelan Hot Dogs es una empresa fundada por Lucas con la intención de vender perros calientes gourmet. Está ubicada en la Calle La Colina Edificio 32, Av. Fuerzas Armadas en Caracas. El objetivo a corto plazo es vender 4 millones y tener 5 sucursales. La estructura legal de la empresa es una Compañía de Responsabilidad Limitada (LLC). La empresa lleva operando desde hace 4 años, los miembros clave del equipo directivo son Lucas - Presidente, Raúl - Vicepresidente y Andrés - Gerente de Operaciones. Actualmente hay 5 empleados y el volumen anual de ingresos es alto.
La historia comenzó cuando Lucas tuvo un pequeño puesto para vender perros calientes en Nueva York que fue muy exitoso. Los principales competidores son todos los restaurantes localizados en el centro comercial Sawgrass Mall. El tamaño del mercado es de 30 billones y hay mucha demanda para este tipo de producto, ya que todos los centros comerciales están llenos. Las tendencias actuales se dirigen hacia la alimentación saludable y rápida, lo cual representa un gran reto para el sector, ya que habrá mucha competencia con precios bajos, además de problemas en la cadena de suministro. Sin embargo, existen grandes oportunidades debido a las nuevas opciones saludables disponibles en el mercado, además del poder adquisitivo latinoamericano que busca nuevos sabores innovadores y tradicionales latinos como los perros calientes venezolanos ofrecidos por Venezuelan Hot Dogs.
Venezuelan Hot Dogs ofrece dos tipos diferentes de perros calientes: Perro Caliente 1 y Perro Caliente 2. Ambos satisfacen las necesidades del cliente gracias a su sabor único e innovador, así como a su conveniencia al ser servidos más rápidamente que otros restaurantes fast food tradicionalmente conocidos. Además, se destacan por su presentación atractiva y adaptación local para Venezuela. Las fortalezas internas incluyen recetas únicas, marca fuerte y personal capacitado. Sin embargo, existen debilidades internas como la escasez limitada de recursos, marca débil y falta de diferenciación entre productos. Existen oportunidades externas tales como mercados nuevos y asociaciones tecnológicas avanzadas; sin embargo, las amenazas externas incluyen competencia regulatoria y cambios económicos.
Las estrategias implementadas para capitalizar sobre las fortalezas y oportunidades y abordar debilidades y amenazas incluyen campañas publicitarias, marketing digital, boca a boca, así como flyers distribuidos en el centro comercial Sawgrass Mall. El mercado objetivo es la Comunidad Venezolana del área Doral, con precios altos para sectores de fast food. Los productos y servicios son distribuidos en punto de venta dentro del local food corner del Sawgrass Mall, para diferenciarse de la competencia, focalizando en la Comunidad Venezolana y gran embalaje con conveniencia para servirse más rápido que otros restaurantes fast food tradicionalmente conocidos y contando con amigos y familiares para la expansión y la fuerza laboral hasta 10 empleados a corto plazo. Para gestionar inventario y cadena de suministro, se utiliza una pequeña área para el congelado y las compras se realizan de manera semanal analizando las ventas semanales para prevenir problemas de stock outs y sobre pedidos. A veces las horas de trabajo cumplen la ley hourly employees con planes de supervisión y desarrollo personal basados en sus estándares para mejorar calidad y servicio ofrecido a la clientela objetivo de Venezuelan Hot Dogs.
