

\subsection{Yearly Profit and Loss}\label{sec:title}
\nonumsidenote{Summary: This section provides an analysis of the yearly profit and loss data for Alpha Project. It shows the revenues, product sales, cost of goods sold, gross profit, operating expenses, labor cost, rent, material cost, maintenance cost, IT costs, sales and marketing costs and lease fees. The results show that revenues have increased from mm US$ 601200 in month 0 to mm US$ 6500492 in month 5 while gross profit margin has decreased from 60.58\% to 56.92\%. EBITDA margin has also decreased from 3.16\% to 21.85\%. Net income margin has decreased from 1.47\% to 17.38\%.}

The table provided is a yearly profit and loss report for Alpha Project which outlines their financial performance over five months (mm US$). The report includes revenues (column 2), product sales (column 3), other services (column 4), cost of goods sold (COGS) (column 5), gross profit (GP) (column 6), operating expenses (OE) (column 7), labor cost(LC)  column 8), rent(R) column 9), material cost(MC)(column 10)), maintenance costs(MTC)( column 11)), other negative recoveries(ONR)( column 12)), IT costs(ITC)( column 13)) , sales and marketing costs(SMC)( column 14))and lease feee(LF)(Column 15)). 

The first observation is that revenue has steadily increased over the five months period from mm US$ 601200 in month 0 to mm US$ 6500492 in month 5 indicating growth in the company's operations over this time frame. Product sales have followed a similar trend with an increase of 1083287mm US$. Other services however have remained at zero throughout the period suggesting they are not generating any significant income through this avenue yet or it is not being reported here . 

In terms of COGS there has been a steady increase as well with a total increase of 2531 188mmUS$. This can be attributed to higher production levels due to increasing demand for products as indicated by rising revenue figures . GP however does not follow this trend as it decreases slightly from 364 200mmUS$ in month 0 to 3700303mmUS$in Month 5 indicating that even though revenue increases COGS increases at a faster rate leading to lower GP margins . 

Operating expenses also increase significantly during the period but remain relatively stable when compared with GP margins . LC is one of the largest contributors accounting for almost half OE with an increase from 264000mmUS$in Month0to174675ommUS $inMonth5 due tpincreased production levels . Rent follows closely behindwithanincreasefrom30000mminMonth0to155000mminMonth5suggest