

\section{Estrategia de Ventas y Marketing para Alpha Project}\label{sec:estrategia-ventas-marketing}
\nonumsubsection{Resumen Ejecutivo} 
Alpha Project es una empresa que ofrece hot dogs gourmet a la comunidad venezolana en el área de Doral. Nuestra ventaja competitiva es la calidad y los servicios, con el objetivo a corto plazo de vender 4 millones y tener 5 sucursales. 

\nonumsubsection{Descripción de la Compañía} 
Alpha Project está constituida como una Sociedad Limitada desde hace 4 años, contando con 5 empleados. El equipo directivo está formado por Lucas (Presidente), Raul (Vicepresidente) y Andres (Gerente de Operaciones). 

\nonumsubsection{Análisis del Mercado} 
Los principales competidores son todos los restaurantes ubicados en el centro comercial Sawgrass Mall, siendo nuestro mercado potencial de 30 billones. La demanda actual es alta ya que todos los centros comerciales están llenos. Los principales retos son la alta competencia, precios bajos y cadena de suministro. Las fuentes principales de presión competitiva son el precio, diferenciación del producto y marketing. La fuerza negociadora tanto para los proveedores como para los compradores es baja debido a su escasa variedad en cuanto al pan disponible para preparar hotdogs. Finalmente, hay un alto riesgo por parte nuevas entradas al mercado así como sustitutos saludables que amenazan con disminuir las ventas.  

 \nonumsubsection{Descripción del Producto o Servicio} 
Alpha Project ofrece principalmente dos tipos de Hot Dogs: Hot Dog 1 y Hot Dog 2; cuyos beneficios clave se basan en su sabor único adaptado a la cultura Venezolana, rapidez e innovación culinaria combinada con conveniencia. Nuestras fortalezas internas incluyen recetas exclusivas mientras que nuestras debilidades se encuentran relacionadas principalmente con limitaciones financieras; existiendo muchas oportunidades externas tales como