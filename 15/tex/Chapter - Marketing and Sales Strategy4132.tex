

\section{Marketing and Sales Strategy}\label{sec:marketing-sales}
Venezuelan Hot Dogs is a business established by Lucas, located in Calle la colina edificio 32 Av Fuerzas Armadas Caracas. The business offers two types of gourmet hot dogs – Hot Dog 1 and Hot Dog 2 – to the people living in the Doral. Venezuelan Hot Dogs has been operating for four years and has five employees. It was founded when Lucas had a small point of hotdogs in New York City which was very successful. 

The main competitors are all the food restaurants in the Sograss Mall, with an estimated market size of 30 billion dollars. The current market demand for Venezuelan Hot Dogs’ products or services is high, as all malls are full. There are trends in the industry towards healthy eating and fast food that Venezuelan Hot Dogs must take into account when creating their marketing strategy. The main sources of competitive pressure come from lower prices and product differentiation, while suppliers have high bargaining power due to there being only one bread supplier available to them. Buyers do not have any bargaining power as they cannot negotiate prices with Venezuelan Hot Dogs. The threat of new entrants into the industry is increasing every day while substitutes such as healthier options pose a threat to traditional fast food outlets such as Venezuelan Hot Dogs. 

Venezuelan Hot Dogs offer two types of gourmet hot dogs –Hot Dog 1 and Hot Dog 2– which meet customer needs through taste, convenience and speediness; however it sets itself apart from other competitors through its unique recipe tailored specifically for local communities within Venezuela who crave these flavours abroad. Its internal strengths include its unique recipe, strong brand recognition and skilled workforce while weaknesses include limited resources, weak brand awareness and lack of differentiation from other competitors on the market today. External opportunities exist within new markets that can be tapped into by forming partnerships or taking advantage of technological advancements whilst external threats come mainly from competition, regulatory changes or economic downturns which could affect business operations significantly if not managed properly.. 

To capitalize on these strengths and opportunities whilst addressing weaknesses and threats posed by external factors, Venezuelan Hot Dogs plan on implementing strategies such as focusing their marketing campaigns around targeting Venezuelans living within the Doral area via social media platforms like Instagram or Twitter along with digital campaigns alongside word-of-mouth advertising techniques like fliers distributed at local malls etc., setting their pricing strategy higher than similar fast food outlets but still competitively priced enough to attract customers’ attention; distributing their products at their own sales point located inside Sawgrass Mall; differentiating themselves through packaging convenience specific to Venezuelans living abroad; hiring friends/family members within Venezuela who understand this culture better than anyone else; expanding physically by opening 6 stores over next year followed up annually with 6 more stores each year thereafter; managing inventory efficiently by analyzing sales data weekly before purchasing supplies accordingly; setting up HR policies based upon law regulations regarding hourly