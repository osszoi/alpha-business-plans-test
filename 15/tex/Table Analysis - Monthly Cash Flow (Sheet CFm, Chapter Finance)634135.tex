

\subsection{Monthly Cash Flow Analysis for Alpha Project}\label{sec:title}
\nonumsidenote{This section of the business plan provides an in-depth analysis of the cash flow generated by Alpha Project over a twelve month period. The analysis will include an examination of net income, depreciation and amortization (D\&A), working capital (WC), operating cash flow, capital expenditure from both bank payments and shareholders, net cash, and cash balance.} 

The table above shows the monthly cash flow data for Alpha Project. The first row lists the months, while columns one through twelve represent each respective month. Net Income is shown in column two with a value of 734 mm US$. This indicates that Alpha Project earned 734 million dollars in net income during the first month. D\&A follows with a value of 667 mm US$, indicating that Alpha Project spent 667 million dollars on depreciation and amortization during this same period. WC is listed as -246 mm US$ which indicates that there was 246 million dollars less in working capital at the end of this period than at its beginning. Operating Cash Flow is then calculated to be 1,647 mm US$ based on these values. Capital Expenditure from both banks payments/loans and shareholders are shown as 0 mm US$. CF from Bank Payment/Loan is also 0mm US$, meaning that no money was borrowed or loaned during this time frame. Net Cash is then calculated to be -148,353 mm US$, while CF from Investment was 150000mm US$. Finally, Cash Balance is calculated to be 1,647mm US$. 

Based on this data it appears that overall Alpha Projects had a positive monthly cash flow throughout all twelve months represented in the table above; however there were some fluctuations within those months due to varying levels of D\&A expenses versus Net Income earned as well as differences between Working Capital amounts at different points throughout these periods. As evidenced by the zero values for both Capital Expenditure from Banks Payments/Loans and Shareholders as well as CF from Bank Payment/Loan it can be assumed that no additional investments were made into either category during this time frame; however it should also be noted that despite not having any additional investments made into these categories there still remained a positive monthly cash flow throughout all twelve months represented in the table above. 

Overall it appears that Alpha Projects has been able to maintain consistent levels of financial stability over its twelve month period represented by this data set; however further investigation should be done into why certain levels of expenses have remained consistent while other have fluctuated over time so potential areas where savings could be achieved can potentially identified if needed going forward. Additionally further research should also look into what long-term investments may need to be made over future periods such as investing more into shareholder equity or taking out loans for large purchases so future financial stability can continue to remain consistent going forward if needed