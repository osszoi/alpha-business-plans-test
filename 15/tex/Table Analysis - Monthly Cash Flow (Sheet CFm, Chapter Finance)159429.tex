
In this chapter, the monthly cash flow of the Alpha project will be analyzed. The revenues, operating and non-operating expenses, as well as the capital flows from and to the bank are presented. The analysis shows that Alpha project is a profitable investment with a positive cash balance throughout the period.

                El flujo de caja mensual es una herramienta útil para ayudar a los propietarios de negocios a entender el dinero que entra y sale durante un período determinado. Esto le permite ver qué ingresos se generan, cuáles son los gastos necesarios y cuál es el saldo neto. Al analizar esta información, puede tomar decisiones informadas sobre la mejor manera de administrar su presupuesto. 
            
                El flujo de caja mensual también le ayuda a identificar patrones en sus finanzas, lo que le permite planificar mejor futuras inversiones y prevenir problemas financieros antes de que ocurran. También puede usarlo para monitorear los resultados de sus operaciones comerciales y hacer ajustes si es necesario. 
            
                Para crear un flujo de caja mensual, primero deben recopilarse todos los datos relevantes del mes pasado. Esto incluye ingresos totales, gastos totales e impuestos pagados durante el período. Luego hay que calcular el saldo neto al final del mes restando los gastos totales de los ingresos totales. Por último, se anotan las tendencias observadas en la información recopilada para compararlas con el mes anterior y ver si hay alguna diferencia significativa en las finanzas del negocio. 
            
                El flujo de caja mensual es una herramienta valiosa para ayudar a los propietarios de negocios a comprender mejor sus finanzas y tomar decisiones acertadas sobre ellas. Al analizar este documento con regularidad, pueden evitar problemas financieros antes de que ocurran y maximizar sus ganancias potenciales al mismo tiempo.

                El Proyecto Alpha genera un flujo de caja mensual. Estas son las entradas y salidas en dólares estadounidenses (US$) para cada mes durante los primeros 12 meses del proyecto. La tabla muestra que el Proyecto Alpha genera un ingreso neto promedio mensual de US$734,000. Los gastos directos e indirectos (D&A) totales suman US$667,000 por mes en promedio, mientras que la variación neta del trabajo capital (WC) es negativa por US$246,000 al principio y luego se estabiliza en 0 a partir del primer mes. Estas entradas y salidas generan un flujo de caja operativo promedio total de US$1,647 mil al principio del proyecto que luego se incrementa hasta alcanzar los US$17 mil al final del periodo considerado.

Además del flujo de caja operativo descrito anteriormente, hay otros dos componentes importantes para determinar el saldo final: los gastos en capital provenientes tanto desde bancos como desde accionistas; y la inversión realizada por parte de los accionistas dentro del proyecto. Respectivamente, estas entradas/salidas representan US$150 mil desde los accionistas en forma única al principio; además, no hay ningún pago o préstamo recibido/realizado a través del banco durante este periodo, lo que significa que no hay ningún movimiento para este concepto en particular durante este primer año de desarrollo del proyecto Alpha.

                Finally, due to all these elements, the cash balance increased constantly and positively reaching the final value of US$17k at the end of the first four months included in this particular study. This shows the potential of the project to obtain excellent scores and generate efficient positive changes in the long run.

Por lo que se ha expuesto anteriormente en esta Sección, podemos sacar ciertas conexiones que el Proyecto Alpha presenta un flujo de caja monthly promedio positivo tanto de sus operaciones como de dinero extra proveniente de su propia inversión o externamente de un préstamo bancario. Esto muestra el potencial para generar ingresos por encima a los gastos en cualquier momento y durante todo el período considerado. Adicionalmente a ello, se debe tomar en cuenta que existe la oportunidad para aumentar el capital de seguridad mediante diferentes mecanismos tales como la inversión de accionistas o flujos extra provenientes de préstamos o investigaciones similares.

                In conclusion, we can maintain that the Alpha Project presents a positive monthly cash flow both in terms of expenses and average monthly income supplied by the project and, to a considerable extent, by shareholders' special inflows that increase its total cash amount.
