

This chapter provides an overview of Venezuelan Hot Dogs' financial plan, including the company's current size and capacity, pricing strategy, operational processes and plans for managing inventory and supply chain management. Venezuelan Hot Dogs is a Limited Liability Company founded four years ago by Lucas with 5 employees and an annual revenue of $30 billion. The main competitors are all the food restaurants in the Sawgrass Mall with high competition, low prices and supply chain as well as big chains like McDonalds or Wendies. There is high demand for their products due to people eating healthy food while also trends towards local adaptation. The target market are members of local Venezuelan community living in Doral area who will be reached through advertising campaigns on social media platforms such as Instagram or Twitter; direct sales; digital campaigns; word-of-mouth referrals; fliers in malls etc. Pricing strategy will focus on higher prices than fast food sector average due to quality ingredients used while product distribution will take place at one location within Sawgrass Mall for now with plans to expand into 6 more locations next year across Miami area following same model afterwards if successful. 

The data in Table 1 provides a detailed look at Alpha Project's yearly cash flow from Year 1 to Year 5. In Year 1, Alpha Project had a net income of 8,814 dollars and an operating cash flow of 55,969 dollars which was offset by D&A expenses of 8,000 dollars being offset by wages and salaries paid out of 39155 dollars along with 150k from financing through shares and 0k through bank payments/loans resulting in negative net cash balance (-94k) but was able to offset it with investments totaling 150k resulting in ending Cash Balance for Year 1 at 55,969 dollars. Subsequent years (Year 2-5) saw increasing Net Income (+43k Y2; +252k Y3; +635K Y4; +1M+Y5) & Operating Cash Flow (+86K Y2; +328K Y3; +687K Y4; +1M+Y5). This trend is attributed mainly to increasing D&A expenses (all 8K per year) which are more than offset by larger Wage & Salaries payments (-34666,-68606,-43651,-28042 respectively). Capital Expenditures decreased significantly over time (from 150K both via shares and bank loans/payments down to 0 across all categories). As such Net Cash Balances improved greatly over time (from -94K up until 2M+ by end of period). 

Overall these figures paint a positive picture for Alpha Projects' financial health as its Net Income increases steadily while its Operational Cash Flow improves exponentially thanks largely due to decreasing CAPEX costs over time despite large Wages & Salaries payments each year resulting ultimately in very impressive profits throughout the period considered here.