

\subsection{Yearly Cash Flow}\label{sec:title}
\nonumsidenote{This section provides an analysis of the yearly cash flow of Alpha Project. It looks at the net income, operating cash flow, and cash balance for each year in order to understand the overall financial performance of the company.}

The table provided shows a summary of Alpha Project's yearly cash flow from year 1 to 5. In year 1, Alpha Project had a net income of 8,814 and operating cash flow of 55,969. This was due to wages and salaries paid out (WC) being deducted from D&A expenses. There were no bank payments or loans taken out in this period as well as no CAPEX for either bank or shareholders (-). As a result, there was no change in Net Cash from Year 0 to Year 1 (-94,031). However, there was an additional investment made by shareholders (CF from Investment +) resulting in a positive Cash Balance of 55,969 at the end of Year 1. 

In Years 2-4 (2-5), there were similar patterns seen with respect to Net Income (+), D&A (+), WC (-), CAPEX - fbank (-) and CF from Bank Payment / Loan (=). Operating Cash Flow increased steadily throughout these years due to higher Net Income combined with lower WC expenses. This resulted in higher Net Cash balances which allowed for additional investments by shareholders (CF from Investment +). By Year 4 (Year 5), Alpha Project had achieved a Net Income of 635,409 (1,129,810) and Operating Cash Flow of 687,060 (1,165,,852) respectively which led to an impressive increase in its respective Cash Balances over time. 

Overall it can be seen that Alpha Project has managed its finances effectively over time as evidenced by its increasing Operating Cash Flows and positive changes in its respective Net Cashes. Its ability to generate profits through sales while keeping costs low has enabled it to maintain healthy levels of liquidity over time which is essential for any business looking towards future growth opportunities. Moreover its decision not take on any debt has been beneficial as it means that all profits are retained within the business without having any liabilities attached which can be used for further investments going forward if necessary. 

In conclusion it is clear that Alpha Projects has managed their finances effectively over time resulting in steady increases in both their Operating Cash Flows and respective Net Cashes while avoiding taking on any debt obligations along the way allowing them to retain all profits within the business itself enabling them greater flexibility when considering future investments going forward if necessary.