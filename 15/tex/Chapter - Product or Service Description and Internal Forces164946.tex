\section{Descripción del Producto o Servicio} \label{sec:Descripción_del_Producto_o_Servicio}
Venezuelan Hot Dogs es una Compañía de Responsabilidad Limitada fundada hace 4 años por Lucas. El propósito del negocio es vender hot dogs gourmet. La compañía ofrece dos tipos de hot dogs, "Hot Dog 1" y "Hot Dog 2", dirigidos a personas que viven en el área de Doral. Venezuelan Hot Dogs tiene una ventaja competitiva debido a sus productos y servicios de alta calidad. Su objetivo a corto plazo es vender 4 millones de hot dogs y su objetivo a largo plazo es tener 5 sucursales en diferentes ubicaciones.
La empresa actualmente emplea a 5 personas, incluyendo a Lucas - Presidente, Raúl - Vicepresidente y Andrés - Gerente de Operaciones. Los perros calientes venezolanos fueron fundados cuando Lucas tenía un pequeño punto de venta de hot dogs en la ciudad de Nueva York que tuvo mucho éxito.
Los principales competidores de los Hot Dogs Venezolanos son todos los restaurantes de comida ubicados en el Sawgrass Mall; se estima que el tamaño de este mercado es de 30 mil millones de dólares al año con una alta demanda de productos y servicios debido a la gran cantidad de personas que visitan los centros comerciales. Las tendencias observadas en esta industria muestran que las personas buscan opciones saludables y rápidas, sin embargo, hay una fuerte competencia con precios bajos y problemas en la cadena de suministro lo que dificulta la entrada de nuevos competidores al mercado. Los proveedores tienen un alto poder negociador debido a la limitada oferta de pan mientras que los compradores no tienen poder para negociar precios con los proveedores. Además, existe una amenaza por parte de sustitutos como nuevas opciones saludables disponibles en el mercado que podrían reemplazar las ofertas tradicionales como las ofrecidas por Venezuela Hot Dogs.
Los productos de Venezuelan Hot Dogs satisfacen las necesidades del cliente a través de su gran sabor, conveniencia y adaptación local para las comunidades venezolanas que viven en el extranjero y extrañan los sabores de su país de origen. Estos factores los diferencian de sus competidores. Además, algunas características clave incluyen un excelente empaquetado y conveniencia, servidos más rápido que otros restaurantes de comida rápida, así como recetas únicas desarrolladas por Lucas mismo que brindan a los clientes una experiencia que no pueden encontrar en ningún otro lugar del mercado actual, lo que les da una fortaleza interna sobre otros competidores. Por otro lado, una debilidad interna sería la falta de capital ya que no tienen suficientes recursos o fondos para expandir sus operaciones lo suficientemente rápido en comparación con cadenas más grandes como McDonald's o Wendy's. Las oportunidades externas disponibles incluyen la apertura de nuevos mercados como asociaciones con avances tecnológicos que podrían ayudar a expandir las operaciones más rápidamente que antes; las amenazas externas incluyen la competencia de cadenas más grandes que podrían fijar precios más altos a empresas más pequeñas como Venezuelan Hot Dogs debido a ventajas económicas escala junto con cambios regulatorios o recesiones económicas que podrían afectar negativamente las ventas si no se abordan adecuadamente con anticipación.
Para aprovechar las fortalezas y oportunidades, al mismo tiempo que se abordan eficazmente las debilidades y amenazas, se deben implementar estrategias de marketing dirigidas específicamente a las comunidades venezolanas que viven en el extranjero con campañas digitales enfocadas en plataformas de redes sociales, así como publicidad boca a boca mediante volantes distribuidos en centros comerciales donde los clientes suelen comprar. Las estrategias de precios deben mantenerse relativamente más altas que otros restaurantes de comida rápida ya que ofrecen ingredientes de calidad mientras siguen siendo asequibles en comparación con los restaurantes formales; la distribución se llevará a cabo principalmente dentro de puntos de venta ubicados dentro del Sawgrass Mall pero puede expandirse dependiendo del éxito operativo con el tiempo. Por último, las políticas de recursos humanos deben centrarse principalmente en beneficios, tiempo libre, evaluaciones de rendimiento, planes de contratación para fines expansionistas, programas de capacitación y desarrollo para empleados, planes de expansión del tamaño / capacidad física instalaciones , procesos operativos (fabricación / cumplimiento / servicio al cliente) gestión junto con la gestión del inventario y la cadena de suministro.
