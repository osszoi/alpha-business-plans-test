

\subsection{Yearly Cash Flow}\label{sec:title}
\nonumsidenote{This section provides an analysis of the yearly cash flow of the Alpha Project. It starts by summarizing the key elements of the table data and then proceeds to explain how these values can be used to assess the financial performance of the company.}

The table data provided in this section shows a detailed breakdown of Alpha Project's net income, depreciation and amortization (D\&A), working capital (WC), operating cash flow, capital expenditure (CAPEX) payments, and cash balance for each year over a five-year period.  In particular, it is possible to observe that Alpha Project had net income figures ranging from $8,814 in Year 1 to $1,129,810 in Year 5; D\&A expenses stayed constant at $8000 per year; WC decreased from -$39155 in Year 1 to -$28042 in Year 5; Operating Cash Flow increased from $55969 in Year 1 to $1165852 in Year 5; CAPEX payments were mostly zero but totaled up to $300000 between Years 1 and 2; CF from Bank Payments/Loans was always zero; Net Cash decreased from -$94031 in Year 1 to +$1165852 in Year 5; CF from Investment was positive only for Years 1 and 2 at +$150000 and +$25000 respectively; while Cash Balance increased steadily throughout the years reaching a total of +2198403 by end of year 5. 

These figures provide valuable insight into Alpha Project's financial performance during this period. First off, it is clear that there has been an overall increase on all fronts: Net Income has grown exponentially as well as Operating Cash Flow which is directly related with profits generated by sales or services rendered. This means that despite having some negative Working Capital figures during certain periods due possibly to high inventory or accounts receivable balances not being paid on time, overall operations have been profitable enough for those losses to be compensated by other sources such as investment returns or loans taken out against future revenue streams. Furthermore, although CAPEX payments have been relatively low compared with other years due possibly lack of need for investments on fixed assets such as buildings or machinery during this period they do represent a significant part of total expenses since they are non-recurring costs related with long-term projects that will generate value down the line. Finally yet importantly it is also possible observe that Net Cash has gone from negative values at start up phase when most likely funds were coming almost exclusively through external sources such as investors or loans (with no bank payment activity) towards more positive values indicating a higher ability for self-sustainability via internal generation of funds after few years into operation which implies greater control over decisions made regarding growth strategies without being tied down too much by external factors such as loan conditions imposed by creditors or terms set forth by investors. 

In