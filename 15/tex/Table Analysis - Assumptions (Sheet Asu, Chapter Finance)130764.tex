

\subsection{Assumptions}\label{sec:assumptions}
\nonumsidenote{This section summarizes the assumptions used to create a business plan for the Alpha Project. It includes growth, inflation, packing and shipping costs, material costs, maintenance costs, other negative recoveries, sales and marketing expenses, taxes incurred and results summary.} 
The Alpha Project is a business venture that requires strategic planning in order to generate maximum revenues. In order to accurately forecast the potential success of this project we have outlined several assumptions in our analysis table \ref{tab:alpha}. The first set of assumptions include growth at 80 percent as well as an inflation rate of 3 percent. Furthermore we have estimated that our packing and shipping costs will be 1.00 percent while material cost will be 1.00 percent and maintenance cost 0.50 percent with other negative recoveries totaling 0.25 percent. We have also assumed that sales and marketing expenses will total 3.00 percent with taxes incurred at 20%. 

The results summary from our analysis table \ref{tab:alpha} indicate that total revenue over 5 years is projected to be 13,834,026 US dollars with a maximum revenue of 6500492 US dollars over the same period. Additionally total NIAT (net income after tax) over 5 years is projected to be 2,069283 US dollars with a maximum NIAT of 1129810 US dollars over the same period. Minimum investment required for this project has been calculated to be 148353 US dollars while NPV@10% stands at 2543267 USD 76 and IRR (internal rate of return) at 186%. 

Based on these assumptions it can be concluded that the Alpha Project has great potential for success if implemented correctly due to its high projected returns on investments within only five years time frame.. This makes it an attractive option for potential investors who are looking for lucrative opportunities in businesses with short-term profitability prospects.. However before investing into this venture it is important to consider all risks associated with such projects including market volatility or unforeseen economic conditions which may adversely affect profits earned from this venture.. Additionally it would also be beneficial to conduct further research into competitors within the industry as well as any government regulations or policies which may need to be adhered too when setting up such ventures.. All these considerations should help ensure successful implementation of this project thereby leading towards greater returns on investments in future years ahead..