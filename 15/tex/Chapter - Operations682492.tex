\section{Operaciones} \label{sec:operaciones}Venezuelan Hot Dogs es una empresa creada por Lucas con la intención de vender perros calientes gourmet. La dirección de la empresa es Calle La Colina, Edificio 32, Av. Fuerzas Armadas, Caracas. Esta empresa tiene como objetivo a corto plazo alcanzar ventas por 4 millones y tener 5 sucursales, mientras que a largo plazo quiere expandirse en el área de Miami. Esta empresa está constituida como una compañía de responsabilidad limitada y lleva operando desde hace 4 años. El equipo directivo lo forman Lucas (Presidente), Raúl (Vicepresidente) y Andrés (Gerente de Operaciones). Actualmente cuenta con 5 empleados y su facturación anual ronda los 30 millones.
La historia de los perros calientes venezolanos se remonta a cuando Lucas abrió un pequeño puesto de venta en Nueva York, que resultó ser todo un éxito entre los lugareños. Los principales competidores son todos los restaurantes del Sawgrass Mall, donde se encuentra ubicado el negocio, principalmente debido al gran flujo turístico que generan este tipo de centros comerciales.
Los productos ofrecidos son dos tipos distintos de perro caliente: hot dog 1 y hot dog 2, ambos diferenciados por su sabor único gracias a las recetas secretas creadas por Lucas para satisfacer las demandas del mercado local venezolano. Estos productos ofrecen sabores únicos, rapidez en su preparación y servicios adicionales para hacerlos más atractivos para el público objetivo, que son aquellas personas residentes en el área de Doral o visitantes turistas. Además, poseen la ventaja competitiva frente a otros restaurantes debido al sabor exclusivo e innovador, así como la adaptación al mercado local venezolano que caracteriza al establecimiento.
El sector presenta una altísima competencia debido a la presión ejercida por cadenas tan grandes como McDonald's o Wendy's, así como los bajos precios impuestos por ellas mismas. Sin embargo, debido al sabor único de nuestros productos, no hay sustitutivos directamente relacionados, siendo este un aspecto positivo a considerar por parte de Venezuela Hot Dogs para su ventaja competitiva frente a otras empresas similares encargadas de ofrecer otros consumibles rápidos o similares. Aunque el poder negociador de proveedores sigue siendo muy alto y el poder negociado de compradores sigue siendo nulo al tratar de cambiar un producto por otro después de comprar el primero.
Para llegar al mercado objetivo, Venezuelan Hot Dogs pondrá en funcionamiento una gran campaña publicitaria tanto online como en los medios tradicionales y una promoción de boca a boca entre los habitantes que forman su distrito objetivo, con el fin de llegar a esa audiencia que busca experiencias, sabores únicos e innovadores y que representa una oportunidad para el crecimiento de la empresa en este punto geográfico.
La apreciación de los productos es muy alta para el sector fast food, pero lo justo buscando que sea conveniente para los clientes y asegurando que la empresa gane el beneficio real necesario para continuar en funcionamiento y expandirse.
En cuanto a la área de Recursos Humanos, el plan actual consiste en favorecer el contrato de amigos y familiares de la Comunidad Venezolana con el fin de que se integren con facilidad al equipo de trabajo actual y aumentar el número de trabajadores tanto como part-timers como full-time employees, todos ellos garantizando las leyes sociales y beneficios correspondientes a empleados regulares. También se prevé la formación y capacitación del trabajador para mejorar sus habilidades y conocimientos siempre orientadas al servicio que ofrece la empresa.
Enloquecerse referente al área operativa, la historia comienza cuando la empresa ubica su primer punto de venta en Sawgrass Mall, una ubicación de gran flujo blanco que se genera por la ubicación turística del mismo, lo cual hace que la empresa utilice este espacio físico como el supermercado.
