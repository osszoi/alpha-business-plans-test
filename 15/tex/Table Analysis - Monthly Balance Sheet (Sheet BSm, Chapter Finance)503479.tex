

\subsection{Monthly Balance Sheet}\label{sec:title}
\nonumsidenote{Summary: This section examines the balance sheet data of Alpha Project for the months of 0-12. It reveals that total assets and liabilities have increased over time, while current assets are mainly cash and inventories. Total equity has also increased over time, with earnings and equity shareholders accounting for this increase.}

The table below shows the monthly balance sheet data of Alpha Project from month 0 to month 12. The data reveals that total assets have steadily increased from month 0 to 12, rising from $155,589 to $163,668 respectively. This is likely due to Alpha Project's successful operations during this period as well as its ability to raise additional capital through debt or equity financing. 

The current assets make up a significant portion of total assets at each point in time. These are mainly composed of cash ($1,647 - $17,060) and inventories ($4,608). Cash is likely used for day-to-day operations such as paying suppliers or employees while inventories are goods purchased by Alpha Project which will be sold at a later date. 

Fixed Assets also make up a large portion of total Assets at each point in time (from $149,333 -$142,000). These include tangible items such as land and buildings which are used by Alpha Projects in its operations but do not generate immediate revenue. 

Total Liability + Equity is equal to Total Assets throughout the period indicating that all liabilities are accounted for on the balance sheet. Current Liabilities consist mostly of Trade Payables ($812-$812) and Other Payables ($3,925-$3,925). Long Term Debt does not appear on the balance sheet suggesting it is not being utilized by Alpha Project during this period. 

Total Equity has also risen steadily from month 0 to 12 (from $150734 -$158814). This increase can be attributed primarily to Earnings ($734-$8814) and Equity Shareholders ($150000-$150000). Earnings represent profits generated by Alpha Projects' operations while Equity Shareholders represent investments made into the company either through debt or equity financing sources such as venture capital or angel investors.  

Overall these figures suggest that Alpha Project has experienced steady growth during this period with increasing levels of both assets and liabilities along with increasing levels of earnings and equity shareholders investments into the company via debt or equity financing sources such as venture capital or angel investors.. Additionally there appears no reliance on long term debt which could lead to higher risks if taken on in future periods