Venezuelan Hot Dogs es una empresa dedicada a la venta de hot dogs gourmet ubicada en la Calle La Colina, Edificio 32, Av Fuerzas Armadas Caracas. Ofrecemos dos tipos de perros calientes para satisfacer las necesidades de los clientes en cuanto a sabor, rapidez y conveniencia. Nuestro mercado objetivo son personas que viven en el área de Doral. Nuestra principal ventaja competitiva es nuestra calidad y servicios. Nuestros objetivos a corto plazo son vender 4 millones de dólares y tener 5 sucursales; mientras que nuestro objetivo a largo plazo es convertirnos en una marca reconocida en toda Venezuela. Somos una Compañía de Responsabilidad Limitada con cinco empleados trabajando actualmente.
Nuestra industria se ve afectada por altos niveles de competencia, bajos precios y problemas en la cadena de suministro como principales desafíos. Sin embargo, nos destacamos debido a recetas únicas que adaptan sabores locales combinados con un gran empaque servido más rápido que otros restaurantes. Las fortalezas internas incluyen productos o servicios únicos junto con un fuerte reconocimiento de marca. Sin embargo, las debilidades incluyen recursos limitados como la falta de inversión capital para abordarse mediante campañas publicitarias dirigidas al público venezolano, así como estrategias precio-calidad estables pero accesibles para los clientes exigentes.

Por otro lado, las oportunidades externas incluyen posibles asociaciones o avances tecnológicos que podrían mejorar la eficiencia. Mientras tanto, las amenazas provienen principalmente del deterioro económico o cambios regulatorios si no se preparan adecuadamente mediante planes contingentes correspondientes.
En cuanto a políticas laborales, contamos con beneficios legales para el personal, así como planes formativos basados en estándares propios. Además, planeamos contratar amigos y familiares dentro del país para ofrecerles opciones laborales adecuadas. En cuanto a operaciones físicas, tenemos dos tiendas pequeñas ubicadas dentro del Sawgrass Mall, con planes futuros apuntando a incrementar 6 tiendas más durante el próximo año distribuidas por toda el área de Miami. Finalmente, la gestión del inventario se centra en análisis semanales para compras pertinentes mientras que los procesos de cadena de suministro abarcan 3 proveedores de salchichas y 1 pan junto con salsas compradas cada semana en Costco.
