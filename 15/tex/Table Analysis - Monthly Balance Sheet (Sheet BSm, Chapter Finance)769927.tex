

\subsection{Monthly Balance Sheet}\label{sec:title}
\nonumsidenote{This section provides an analysis of the monthly balance sheet data for Alpha Project. The total assets and liabilities have increased steadily over the period, with current assets increasing at a faster rate than fixed assets. Cash has increased significantly over the period, while accounts receivable remain unchanged. Inventories have remained relatively constant, while current liabilities and trade payables have also remained steady. Total equity has increased steadily over the period, driven by an increase in earnings.} 

Alpha Project's monthly balance sheet data shows a steady increase in total assets and liabilities over the 12-month period from January to December 2020. Over this time, current assets have grown at a faster rate than fixed assets, with cash increasing significantly while accounts receivable remain unchanged. Inventories have remained relatively constant across this period as well, while both current liabilities and trade payables have been consistent throughout 2020. Total equity has seen a steady growth trend during this time frame due to an increase in earnings reported by Alpha Project during this time frame. 

The most notable change in Alpha Project's balance sheet is its large jump in cash reserves from January to December 2020. This is likely due to improved financial management practices implemented by Alpha Project that allowed it to better manage its cash flow and keep more money on hand for operating expenses or other investments opportunities that may arise throughout the year. Additionally, Alpha Project's inventories stayed relatively flat throughout 2020 despite increased demand for their products or services which suggests that they were able to effectively manage their inventory levels without having to purchase additional stock or materials during this time frame which helped them keep operational costs down as well as maintain healthy margins on their products or services offered during this same timeframe 

In addition to these changes on the asset side of Alpha Projects' balance sheet there were also some changes on the liability side of things as well such as increases in long term debt when compared against short term debt which suggest that Alpha Projects was able to secure longer term financing options at lower interest rates when compared against short-term financing options such as lines of credit or business loans taken out throughout 2020 which could help them reduce their overall debt burden going forward if they are able to service these debts efficiently over time  . 

 Overall it appears that Alpha Projects was able to successfully manage its finances through careful budgeting and cost control measures implemented throughout 2020 resulting in improved liquidity ratios when compared against previous years� performance suggesting that they are better positioned financially going into 2021 then they were before . 

 Summary: This section provided an analysis of monthly balance sheet data for Alpha Project showing how total assets and liabilities had grown steadily over 12 months from January 2020 - December 2020 due mainly too improved financial management practices implemented by company allowing them better manage cash flow , maintain healthy inventory levels , secure longer term financing options at lower interest rates , all resulting in improved liquidity ratios positioning company better financially going into 2021