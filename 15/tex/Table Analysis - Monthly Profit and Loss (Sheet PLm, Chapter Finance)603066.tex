

\subsection{Monthly Profit and Loss}\label{sec:title}
This section presents an analysis of the monthly profit and loss data for Alpha Project in 2020. The table shows that the company had a total revenue of \$50,100 each month, with product sales accounting for all of it. Cost of goods sold was \$19,750 each month while operating expenses totaled \$28,095. Labor costs accounted for the majority of this expense at \$22,000 per month. Other expenses included rent (\$2,500), material cost (\$500), maintenance cost (\$500), other negative recoveries (\$125) IT costs (\$300) and sales and marketing expenses (\$1,500). Lease fees amounted to \$670 per month. 

The gross profit was \$30,350 each month while EBITDA was \$1,585 per month. Depreciation and amortization were at a steady rate of \$667 each month while EBIT was slightly lower at \918 due to interest payments totaling zero every month during 2020. After current taxes were taken out net income came to a total of \734 every month in 2020 for Alpha Project. 

Gross profit margin was 60.58 percent which is relatively high indicating that there is potential to increase profits by reducing costs or increasing revenues without incurring additional overhead costs such as labor or rent etc.. EBITDA margin was 3.16 percent while EBIT margin stood at 1.83 percent which are both within normal range when compared with other companies in similar industries; however they could be improved upon by cutting down on unnecessary expenses such as lease fees or material cost etc.. Net income margin was 1.47 percent which indicates that there is room for improvement in profitability through better management practices or increased efficiency measures etc.. 

In summary, Alpha Project had a total revenue of $50,100 each month during 2020 with product sales accounting for all its income; however it also incurred significant operating expenses resulting in only a modest net income margin of 1.47%. With careful management and strategic changes in operations such as reducing unnecessary costs or increasing efficiency measures the company can improve its bottom line profits significantly over time without having to incur any additional overhead costs such as labor or rent etc..