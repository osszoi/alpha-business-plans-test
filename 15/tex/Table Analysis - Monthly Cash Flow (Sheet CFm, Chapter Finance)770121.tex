

\subsection{Monthly Cash Flow}\label{sec:title}
\nonumsidenote{\textbf{Summary}: The following report provides an analysis of the monthly cash flow for Alpha Project. It includes a discussion of the net income, depreciation and amortization, working capital, operating cash flow, capital expenditures from bank payments or loans, net cash, and cash balance over twelve months.}

The table data provided contains information related to Alpha Project's monthly cash flow over twelve months. This report will analyze each element in order to gain insight into the company's financial performance. 

First, it is important to note that the columns represent months (m) and the rows represent years (y). The first row represents mm US$, which is a unit of measurement for currency. 

Starting with Net Income (+), we can see that Alpha Project had a consistent net income over all twelve months at 734 mm US$ each month. Depreciation and Amortization (+) followed suit with 667 mm US$ per month throughout the year. Working Capital (-) was 246 mm US$ in month 0 but dropped to 0 in subsequent months due to loan repayments or other investments made by Alpha Project. 

Operating Cash Flow was 1,647 mm US$ in month 0 but decreased slightly every month thereafter due to payments towards Capital Expenditure - fbank (-) or Capital Expenditure - fshare (-). Both of these components were zero for all 12 months except for when Alpha Project invested 150000 mm US$ into its own shares during month 1. As expected, this large investment caused a decrease in Net Cash from 148353 mm US$, where it started at in month 0 down to 1401 mm US$ by month 12 as more money was spent on investments than earned through operations during those 12 months. Finally, CF from Investment (+) increased Cash Balance from 1647mmUS$ up to 1706mmUS$. 

Overall, it appears that Alpha Projects finances have been stable throughout the year despite making large investments into its own shares early on in 2020; however there is still potential room for improvement as their operating profits are not enough cover their expenses without relying on outside sources such as loans or investments from shareholders.