

\subsection{Yearly Cash Flow}\label{sec:title}
\nonumsubsection*{Summary} This section will analyze the yearly cash flow of Alpha Project, focusing on net income, depreciation and amortization (D\&A), working capital (WC), operating cash flow (OCF), capital expenditure from bank payments/loans (CAPEX-fbank) and CAPEX from shareholder investments (CAPEX-fshare). The analysis will also include a summary of the main ideas discussed. 

Net Income for Alpha Project has increased significantly over the past five years, growing from 8,814 in year 1 to 1,129,810 in year 5. This is likely due to an increase in sales or other factors that have allowed the company to generate more income. 

Depreciation and Amortization expenses have remained steady at 8000 each year. Working Capital has decreased over time from 39155 in year 1 to 28042 in year 5. This indicates that Alpha Project is becoming more efficient with their use of resources as they become more established. 

Operating Cash Flow has increased significantly over time as well; it grew from 55,969 in year 1 to 1,165,852 in year 5. This suggests that Alpha Project is becoming increasingly profitable and able to generate more cash flow than before. 

Capital Expenditure from Bank Payments/Loans has been zero for all five years while Capital Expenditure from Shareholder Investments was 150000 at first but then reduced by 25000 for each subsequent year until it became zero by Year 4. This could indicate that shareholders are less interested in investing money into the company after seeing its success or that there were no longer any projects requiring additional funding after Year 3. 

Finally Net Cash Flow was negative at -94031 during Year 1 but then shifted towards positive values until reaching a peak of 1165852 during Year 5; this suggests that Alpha Project has become increasingly successful over time and is now generating significant amounts of cash flow each year. 

Overall these figures show that Alpha Project has grown significantly since its inception; net income and operating cash flow have both increased drastically while working capital costs have decreased steadily indicating improved efficiency within the company�s operations. Furthermore shareholder investments have dropped off after three years suggesting they may be content with their returns thus far or simply feel no need to invest further funds into new projects due to current success levels achieved by the business so far