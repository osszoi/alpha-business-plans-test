Venezuelan Hot Dogs es una empresa que se dedica a la venta de hot dogs gourmet en el área de Doral, con la misión de ofrecer un producto de alta calidad y servicio excepcional a sus clientes. La compañía ofrece dos tipos de hot dogs y su mercado objetivo son las personas que viven en Doral.

La ventaja competitiva de la empresa radica en la calidad y el servicio que ofrece, con objetivos a corto plazo para vender 4 millones de hot dogs y tener cinco sucursales, mientras que los objetivos a largo plazo incluyen expandirse por todo Miami.

Venezuelan Hot Dogs es una compañía fundada hace cuatro años, con los miembros clave del equipo directivo incluyendo al presidente Lucas, al vicepresidente Raul y al gerente de operaciones Andres. Actualmente tienen cinco empleados y su historial incluye haber tenido un pequeño punto de venta en Nueva York.

Los principales competidores son todos los restaurantes ubicados en el centro comercial Sograss, pero creen que hay un mercado potencial valorado en $30 mil millones para ellos. La demanda actual del mercado es alta ya que los centros comerciales están llenos.

Las tendencias actuales muestran una preferencia por alimentos saludables y rápidos. Las principales fuentes de presión competitiva son los precios bajos y la cadena de suministro inestable. Los principales competidores son las grandes cadenas como McDonald's o Wendy's.

El poder negociador con los proveedores es alto debido a que solo cuentan con un proveedor para el pan, mientras que el poder negociador de los compradores es bajo. La amenaza de nuevos participantes en el mercado es alta, pero la amenaza de sustitutos es baja.

Los productos de la empresa se destacan por su sabor y conveniencia, con una receta única adaptada a la comunidad venezolana local como fortaleza interna. La falta de capital es una debilidad interna. Las oportunidades externas incluyen un mercado ávido por sabores innovadores y una gran comunidad venezolana en la zona, mientras que las amenazas externas incluyen una recesión económica y alta competencia.

Para capitalizar sus fortalezas y oportunidades, planean implementar campañas de marketing enfocadas en la comunidad venezolana a través de redes sociales, publicidad digital y boca a boca. Planean distribuir sus productos desde su punto de venta en el centro comercial Sawgrass.

Para diferenciarse de sus competidores, se centrarán en ofrecer empaques convenientes además del servicio más rápido posible para un restaurante fast food. Actualmente emplean 10 personas con planes futuros para expandirse alrededor del área Doral abriendo seis tiendas adicionales durante el próximo año antes de expandirse aún más por Miami.

Los procesos operativos actuales incluyen tres proveedores diferentes para las salchichas, un proveedor para el pan y compran todos los demás ingredientes necesarios en Costco. Planean administrar su inventario analizando las ventas semanales para ajustar sus compras según sea necesario.