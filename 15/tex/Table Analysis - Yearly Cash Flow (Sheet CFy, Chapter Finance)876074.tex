

\subsection{Yearly Cash Flow}\label{sec:title}
\nonumsubsection*{Summary:} This chapter provides an analysis of the yearly cash flow for Alpha Project from year 1 to 5. Net income increased significantly over the five-year period, while operating cash flow grew steadily and net cash position improved. Capital expenditures were funded mainly through debt financing in years 1 and 2, but shifted to equity financing in years 3, 4 and 5. 

The table provided shows a detailed view of the yearly cash flow for Alpha Project from year 1 to 5. The first column contains the labels for each row of data, with �Year� as the first entry followed by �Net Income (+)�, �D&A (+)� (depreciation and amortization), �WC (-)� (working capital), �Operating Cash Flow�, �CAPEX - fbank (-)� (capital expenditure funded by bank payment/loan), �CAPEX - fshare (-)� (capital expenditure funded by share issuance), �CF from Bank Payment / Loan (=)� (cash flow from bank payment/loan), �Net Cash�,  �CF from Investment (+)�(cash flow from investment), and finally  "Cash Balance". 

Beginning with Year 1, Net Income was $8 814 which is relatively low compared to subsequent years. Depreciation & Amortization expenses were $8000 in all five years. Working capital decreased every year except Year 3 where it increased slightly before decreasing again in Years 4 & 5; this suggests that Alpha Project has had difficulty managing its working capital efficiently throughout this period. Operating Cash Flow was $55 969 in Year 1 which then increased steadily up until Year 4 where it reached its peak at $687 060 before decreasing slightly to $1 165 852 in Year 5; this indicates that Alpha Project's operations have become more profitable over time as shown by their increasing Operating Cash Flow figures. Capital Expenditures were mainly financed through debt financing during Years 1 & 2 ($150 000 each). During Years 3-5 CAPEX was funded primarily through equity financing ($0 each year). The result is a positive Net Cash position each year indicating that Alpha Project has been able to generate enough revenue during these periods to cover all costs associated with running the business including CAPEX investments without needing additional external funding sources such as loans or credit lines. Finally we see that overall Cash Balance increases steadily throughout these five years reaching a peak of $2 198 403 at the end of Year 5 indicating that overall Alpha Projects financial health has been improving over time despite some hiccups along the way such as inconsistent working capital management practices.  

In conclusion we can see that since its inception Alpha Projects financial performance has been improving consistently over time due largely to increasing net income figures combined with efficient