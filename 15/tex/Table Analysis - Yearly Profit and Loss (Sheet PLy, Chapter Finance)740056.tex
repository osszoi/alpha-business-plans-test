

\subsection{Yearly Profit and Loss}\label{sec:title}
\nonumsidenote{This section provides an analysis of the yearly profit and loss for the Alpha Project. It includes a summary of the gross profit, operating expenses, labor cost, rent, material cost, maintenance cost, IT costs, sales and marketing costs, lease fees as well as EBITDA margin and net income margin.}
The table provided contains data on monthly revenues from product sales and other services for the Alpha Project along with associated costs. The total revenue for each month is shown in column 2 (mm US$) while columns 3-5 show the breakdown between product sales and other services. Columns 6-13 provide information about various costs including Cost of Goods Sold (COGS), Operating Expenses (OE), Labor Cost (LC), Rent (Rent), Material Cost (MC), Maintenance Cost (MCt) Others (neg. recoveries) IT Sales and Marketing Lease Fee. Finally columns 14-17 show Earnings Before Interest Taxes Depreciation & Amortization (EBITDA) EBIT Interest EBT Current Tax Net Income Gross Profit Margin EBITDA margin EBIT margin Net income margin respectively. 

From this data we can see that overall revenue has increased steadily over the five months with a peak in May at 6500492 mm US$. This increase was largely due to an increase in product sales which accounted for almost all of this growth; other services remained relatively constant at 0 mm US$. COGS also increased alongside revenue reaching 2800188 mm US$ in May indicating that more products were being sold but at a lower price point than earlier months. Operating expenses also rose steadily over these five months to 2279333 mm US$, however they still remained below COGS meaning that there was still some degree of profitability even after accounting for OE. 

Labor cost was one of the largest contributors to OE increasing from 264000 mm US$ in January to 1746750 mm US$ by May; rent followed a similar trend increasing from 30000 mm US$ to 155000 mmUS$. Material cost MCt maintained a steady level throughout while IT Sales & Marketing Lease Fee experienced minor fluctuations with no clear pattern emerging from this data set. Overall OE was slightly higher than LC suggesting that overhead expenses such as rent utilities etc may have been higher than expected or budgeted for during these five months though it is difficult to draw any definitive conclusions without further information about these specific expenses or about budgeting practices within Alpha Project itself. 

Finally we can analyze margins earned by Alpha Project during these five months by looking at columns 23-26 which represent Gross Profit Margin EBITDA Margin EBIT Margin & Net Income Margin respectively These margins indicate how much profit is being made relative to revenue generated during each month For example Gross Profit Margin shows us what percentage of total revenue remains after subtracting COGS As