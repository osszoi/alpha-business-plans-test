Alpha Project es una empresa que se dedica a la venta de perros calientes gourmet. Es liderada por Lucas, quien cuenta con 4 años de experiencia en el negocio, y tiene 5 empleados. El objetivo principal de Alpha Project es vender 4 millones de perros calientes al año y abrir 5 sucursales en los próximos 3 años. La estructura legal que rige a la compañía es una Limited Liability Company (LLC). Su competencia directa son todos los restaurantes ubicados en el Sawgrass Mall. Sin embargo, hay muchas otras cadenas, como McDonalds, Wendies, entre otros, que también influyen en el mercado. Se estima que el tamaño del mercado potencial es de 30 billones anualmente.
Los principales desafíos a los que se enfrentan las empresas del sector son la fuerte competencia existente, los bajos precios y el manejo adecuado de su cadena logística. Mientras tanto, su fuente principal de presión competitiva proviene del bajo precio ofrecido por sus rivales directos e indirectos. Por otro lado, el poder negociador tanto de los proveedores como de los consumidores se considera sumamente positivo para Alpha Project, ya que cuentan con un único proveedor para sus panes y no hay ninguna barrera para acceder al mercado local.
Los productos ofrecidos por Alpha Projects incluyen dos variedades diferentes: Hot Dog 1 y Hot Dog 2, preparadas siguiendo recetas exclusivas diseñadas para satisfacer las necesidades particulares del público local venezolano. Esto les permite contar con un amplio margen dentro del segmento fast food, gracias a su sabor único e innovador, sin descuidar aspectos clave tales como rapidez y conveniencia. Paralelamente, la marca cuenta con fortalezas internas, como su posicionamiento regional, excelente presentación visual y habilidad para adaptarse al gusto lingüístico del país, mientras que sus debilidades principales son la carencia de capital y limitaciones internas a corto plazo para cubrir necesidades operacionales y administrativas.
Las oportunidades externas incluyen políticas contrarrestativas contra la transformación empresarial, así como la explotación exitosa mediante condiciones promocionales electrónicas. Mientras tanto, las amenazas externas se refieren a los ingresantes constantes al sector, así como a alternativas saludables recientemente creadas por otros actores importantes del mercado.
Por último, respecto a la mercadotecnia de venta, Alpha Project busca dirigirse al target market de Doral (venezolanos en Miami) mediante campañas publicitarias digitalizadas, trabajo de voz a voz, campañas de volantes dentro del mall para promover productos, fijando precios superiores sin perder calidad en el servicio al cliente. Distribuirse en un mayor número de direcciones y desarrollar futuras sucursales durante el próximo año, agregando 6 sucursales adicionales a partir del tercer y cuarto año.
